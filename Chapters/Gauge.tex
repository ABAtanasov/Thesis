\chapter{Gauge Theory\label{ch:gauge}}

	Gauge theory will play a central role in understanding the geometric Langlands correspondence physically. The role of the group $G$ in the Langlands correspondence is played by the gauge group in the physical theory. 

	\section{Fiber Bundles}
	\subsection{Definitions and Examples}
		We will be working on a manifold $M$ (not necessarily Riemannian). In the first definition, we can assume $M$ is just a topological space. 
	\begin{defn}[Fiber Bundle]
		We define a \textbf{fiber bundle} on a topological space
		\begin{itemize}
			\item A topological space $E$ called the \textbf{total space}
			\item A topological space $M$ called the \textbf{base space}
			\item A topological space $F$ called the \textbf{fiber} 
			\item A \textbf{projection map} $\pi: E \to M$ that is surjective so that $\pi^{-1}(p) := E_p$ is homeomorphic to $F$. This is \textbf{the fiber over $p$}.
			\item For each $x \in E$ there is an open neighborhood $U \subseteq M$ of $p = \pi(x)$ so that there is a homeomorphism $\psi$ from $U \times F$ to $\pi^{-1}(U)$ in such a way that projection $p_1$ onto the first factor of $U \times F$ gives $\pi$
			\[
			\begin{tikzcd}
				U \times F \arrow[rd,swap,"p_1"] \arrow[rr,"\psi"] && \pi^{-1}(U) \arrow[dl,"\pi"]\\
				& U &
			\end{tikzcd}
			\]
		\end{itemize}
	\end{defn}
	Fiber bundles generalize the notion of cartesian products of two spaces $M$ and $F$ by allowing for the same local product structure but much more interesting global ``twisted structure''. 
	
	
	In physics, especially when calculations are to be performed, manifolds are often described in terms of a set of coordinate charts $U_\alpha$ that are homeomorphic to $\mathbb R^n$ with $n = \dim M$ and $\alpha \in I$ is an index in some indexing set, not necessarily finite\footnote{But in the case of $M$ compact, $I$ can always be made finite.}. A covering of $M$ in terms of coordinate charts 
	\[
		M = \bigcup_{\alpha \in I} U_\alpha.
	\]
	together with homeomorphisms $\psi_\alpha: U_\alpha \to \mathbb R^n$ is called an \textbf{atlas} for $M$. In order to make sense of $M$ in terms of an atlas, we define \textbf{transition maps} between different $U_\alpha$ by $\varphi_\beta \circ \varphi_{\alpha}^{-1}$.
	
	By using transition maps, we can transport data locally defined on $U_{\alpha}$ to other parts of $M$ by ``moving it across'' other $U_\beta$. This data often comes from the fiber bundles over $M$. This gives us an ability to ``glue together'' locally trivial bundles on the $U_\alpha$ to construct a globally nontrivial fiber bundle. For the fiber bundles of interest to us, there will be a group $G$ of automorphisms that acts on the fibers when comparing the data across different $U_\alpha$. We will later refer to $E$ as an \textbf{associated bundle} to $G$.
	We define this more precisely:
	\begin{defn}[Coordinate Bundle]
		A \textbf{coordinate bundle} consists of 
		\begin{itemize}
			\item A fiber bundle, defined as before
			\[
			\begin{tikzcd}
				F \arrow[r] & E  \arrow[d,"\pi"] \\ 
				& M
			\end{tikzcd}
			\]
			\item A group $G$, called the \textbf{structure group} of $E$ acting effectively on each fiber\footnote{A $G$-action is effective if only the identity element acts trivially i.e. $\forall g\in G \, \exists f \in F \mid gf \neq x$. The reason for this is that if $G$ did not act effectively, then elements that act trivially would give a normal subgroup $N$. Upon passing to the quotient we would get an effective action of $G/N$ on $F$.}.
			\item A set of open coverings $\{ U_\alpha \}_{\alpha \in I}$ of $M$ with diffeomorphisms $\phi_\alpha: U_\alpha \times F \rightarrow \pi^{-1} (U_i)$ called \textbf{local trivializations} so that the following diagram commutes. 
			\[
			\begin{tikzcd}[row sep=2.5em]
			U_\alpha \times F \arrow[dr,swap,"p_1"] \arrow{rr}{\psi_\alpha} && \pi^{-1}(U_\alpha) \arrow{dl}{\pi} \\
			 & U_\alpha \\
			\end{tikzcd}
			\]
			\item For each $p \in U_\alpha \cap U_\beta$, $\psi_\beta^{-1} \psi_\alpha$ act continuously on the fiber $\pi^{-1}(p)$, coinciding the action of an element of $G$.
		\end{itemize}
		% diffeomorphism\footnote{This is a diffeomorphism in this case because we are considering a $C^\infty$ fiber bundle. If we were considering continuous, differentiable, or holomorphic fiber bundles this map would be continuous, differentiable, or holomorphic respectively.}
		
		% $\psi_\alpha$ captures the explicit isomorphism data between the fiber object $F$ and the fiber at a point $F_p$. $\psi_{\alpha, F_p} (f) = \psi_\alpha (p, x)$
	\end{defn}
		
	In gauge theory, $G$ is taken to be a \textbf{Lie group} called the \textbf{structure group} of $E$.
	\begin{defn}
		A Lie group is a group that is also a differentiable manifold so that the group operations of multiplication and inversion are compatible with the  differentiable structure. 
	\end{defn}
	A basic working knowledge of Lie theory is assumed, however we will go over relevant aspects of Lie groups in the following sections of this chapter.
	
	\begin{nb}
		In the above, we described $\varphi_\alpha$, $\tau_{\alpha \to \beta}$, and $\psi_\beta^{-1} \psi_\alpha$ as \emph{homeomorphisms}, which are indeed morphisms in the category of topological spaces. If we wish to work in other categories, such as $C^r$-differentiable, smooth, analytic, or complex manifolds, then the transition functions would have to be $C^r$-differentiable, smooth, convergent Taylor series, or holomorphic respectively. 
		If we were working in the category of algebraic varieties, the corresponding maps we consider would have to be \emph{regular}.
	\end{nb}	
		%
		% From this we can define the vertical component of a point in the total space: $f_\alpha: E \rightarrow F$ by $f_\alpha = \psi^{-1}_{\alpha, \pi}$.
		
		At the fiber over each point, since we can identify $\psi^{-1}_{\beta, p} \circ \psi_{\alpha, p}$ with an element in $G$, we write $g_{\alpha, \beta}: U_{\alpha \beta} \rightarrow G$ to denote the $G$ action fiberwise on the overlap of the two bundles over $U_\alpha, U_\beta$. This translates data from one coordinate patch into the other.
		\begin{prop}
			$g_{\alpha \beta}$ satisfies 
			\begin{itemize}
				\item (identity) $g_{\alpha \alpha} = 1$
				\item (inversion) $g_{\alpha \beta} = g^{-1}_{\beta \alpha}$
				\item (cocycle condition) On $U_\alpha \cap U_\beta \cap U_\gamma$ $g_{\alpha \beta} g_{\beta \gamma} = g_{\alpha \gamma}$
			\end{itemize}
			\label{prop:cocycle}
		\end{prop}
		% One way to look at fiber bundles (c.f. \cite{lindenhovius2011}) is to define on the restriction of $E$ to each $U_\alpha$, functions $f_\alpha: U_\alpha \to F$ given by $f_\alpha(x) := p_2 p_1^{-1} (p)$
%
% 			Moreover, pick $x \in U_\alpha \cap U_\beta$ and $F_\alpha, F_\beta$ be the fibers over $x$ of $E$ as a bundle on $U_\alpha, U_\beta$ respectively.
% 			\begin{enumerate}
% 				\item $g_{\alpha \beta} F_\beta = F_\alpha$, that is $g_{\alpha \beta}$ gives the isomorphism to the the fiber corresponding to $U_\beta$ to the fiber corresponding to $U_\alpha$.
% 				\item $\psi_{\beta} (p, f) = \psi_{\alpha} (p, g_{\alpha \beta} f)$
% 			\end{enumerate}
%
%
% 		\begin{proof}
% 			These are all easy to check just by the definition of $g_{\alpha \beta}$ as a composition of the $\psi_\alpha$ and by invoking the cartesian properties of local trivialization.
% 		\end{proof}
		The equivalence class of a set of coordinate bundles on $M$ gives the corresponding fiber bundle over $M$.
		
		Fiber bundles whose fibers are vector spaces are called \textbf{vector bundles}. The \textbf{rank} of a vector bundle is the dimension of the vector space fiber. A rank $n$ vector bundle over a field $\kk$ will have its structure group $G \subseteq \GL_n(\kk)$. Examples are the tangent/cotangent bundles of a manifold, and any tensor/symmetric/exterior powers thereof. We will see that we can view vector fields, $p$-forms, and many other interesting and physically-relevant objects as \textbf{sections} of fiber bundles, to be described in the later sections.
		
		\subsection{Morphisms and Extensions}
		The morphisms in the category of fiber bundles are called \textbf{bundle maps}:
		
		\begin{defn}[Bundle Map]
			For two fiber bundles $\pi: E \rightarrow M, \pi': E' \rightarrow M'$ a bundle map is a smooth map $\overline f: E \rightarrow E'$ that naturally induces a smooth map on the base spaces so that the following diagram commutes:
			\[
			\begin{tikzcd}
			E \arrow{r}{\overline f} \arrow{d}{\pi} & E' \arrow{d}{\pi'} \\
			M \arrow{r}{f} & M.
			\end{tikzcd}
			\]
		\end{defn}
		From this we obtain the way which we will identify two bundles as identical.
		\begin{defn}[Equivalence of fiber bundles]
			Two bundles are equivalent if there is a bundle map so that both $\overline f$ and $f$ are diffeomorphisms. 
		\end{defn}
		If we have a fiber bundle $\pi: E \rightarrow M$ and $\varphi: N \rightarrow M$ for another manifold $N$, then we can pull back $E$ to form a bundle over $N$
		\[
			\varphi^* E = \{(y,[f,p]) \in N \times E \mid \varphi(y) = p \}.
		\]
		We have projection on the second factor of $\varphi^* E$ as a map $g: \varphi^* E \rightarrow E$. 
		This is the \textbf{pullback bundle} $\varphi^* E$.
		\begin{defn}[Pullback Bundle]
			For a map $\varphi: N \rightarrow M$ and $E$ a fiber bundle over $M$ so that $\pi: E \rightarrow M$, we define the pullback bundle $\varphi^* E$ so that the following diagram commutes:
			\[
			\begin{tikzcd}
			\varphi^* E \arrow{r}{g} \arrow{d}{\pi'} & E \arrow{d}{\pi} \\
			N \arrow{r}{\varphi} & M.
			\end{tikzcd}
			\]
		\end{defn}
		Let us consider an example which will appear later in the context of studying a monopole placed at the origin of $\rr^3$.
		\begin{eg}
			Consider a vector bundle $E  \to \rr^3 \backslash \{0\}$. The pullback  gives rise to a vector bundle on $S^2$. This should be thought of as the restriction of $E$ to the sphere $S^2$
		\end{eg}
		
		We can take products of fiber bundles as topological spaces in the obvious way to obtain a fiber bundle over $M \times M'$,
		\[
			E \times E' \xrightarrow{\pi \times \pi'} M \times M'.
		\]
		In the special case where $M = M'$ we can also define 
		\begin{defn}[Whitney Sum of Vector Bundles]
			For $E, E'$ vector bundles over $M$ with structure groups $G, G'$ respectively, we can define their sum as $E \oplus E'$ to be pullback bundle $E \times E'$ along the diagonal map $\Delta: M \rightarrow M \times M$.
			
			More explicitly, this is a fiber bundle over $M$ with $F \oplus F'$ fibered over every point. 
			The structure group of $E \oplus E'$ is the product $G \times G'$ of the structure groups of the original bundles and it acts diagonally on their sum.
			\[
				G^{E \oplus E'} = \left \{ \begin{pmatrix}
					g^E & 0 \\
					0 & g^{E'}
				\end{pmatrix} : g^E \in G, g^{E'} \in G' \right \}
			\]
			and the transition functions act diagonally in the same way. 
		\end{defn}
		Similarly, we can define arbitrary direct sums of bundles recursively:
		\[
			E_1 \oplus \dots \oplus E_r.
		\]
		
		For some intuition about when fiber bundles are \emph{nontrivial}, consider the following theorem which we state without proof but refer to \cite{bott1982} chapter 3. Stated simply: taking the pullback of a bundle along a map is topologically invariant under homotopy of the map. 
		\begin{theorem}
			Let $\pi:E \rightarrow M$ be a fiber bundle over $M$ and consider maps $f, g$ from $N \rightarrow M$ so that $f, g$ are homotopic, then the pullback bundles are equivalent: $f^* E \cong g^* E$ over $N$.
		\end{theorem}
		An important fact is the following corollary:
		\begin{cor}
			If $M$ is contractible, every fiber bundle $\pi: E \rightarrow M$ is topologically trivial\footnote{In the language of classifying spaces, $M$ being trivial implies there is only one homotopy class of map $M \to BG$, so that consequently the only fiber bundle over $M$ is the trivial one.}.
		\end{cor}
		\begin{proof}
			Let $f: pt \to M$ and $g: M \to pt$ be such that $f \circ g \sim id|_M$ and $g \circ f \sim id|_{pt}$. Then because pullback respects homotopy equivalence, we will have that $E \sim (f \circ g)^* E \sim f^* (g^*E)$ but $g^* E$ is the (necessarily trivial) bundle on a point, so this will pull back along $f$ to the trivial bundle along $f$.
		\end{proof}
		
		\subsection{Principal Bundles} % (fold)
		\label{sub:principal_bundles}
		

		
		We have seen that in general, the structure group of a fiber bundle acts effectively on the fibers. More strictly, when $G$ acts freely\footnote{A free $G$-action on $F$ is one where $\forall f \in F, gf = f \Rightarrow g = 1$ i.e. each element has only the identity fixing it. This is a more restrictive form of effective action.} and transitively\footnote{A transitive $G$-action on $F$ is one with a single $G$-orbit, i.e. any element can be taken to any other.} \emph{from the right}\footnote{The reason for defining this to be a \emph{right} action is so that it can commute with transition maps, which are taken to act from the \emph{left} \cite{stackoverflow1}.} on the fiber, we can identify $F$ with $G$. In this case, we get a \textbf{principal $G$-bundle}. This will be an object of central interest in what follows.
		
		\begin{obs}
			The fibers of a principal $G$-bundle are \emph{homeomorphic} to $G$
		\end{obs}
		\begin{proof}
			Let $P \to M$ be a principal $G$-bundle and pick a point $p \in M$. Take a point $f \in \pi^{-1}(p)$. We can construct a homeomorphism $\varphi: G \to \pi^{-1}(p)$ by sending $g \mapsto p g$. To prove that this is invertible, note that the action is transitive, so $\varphi$ is certainly surjective. Further, if $p g = p g'$ then $p = p g {g'}^{-1}$ so necessarily $g = g'$, and we have injectivity as a map between topological spaces.
		\end{proof}
		
		
		Though \emph{topologically} each fiber $F$ of a principal $G$-bundle looks like $G$, unlike $G$, $F$ need not have a canonical choice of identity element, and consequently does not generically have canonical groups structure. Indeed, if it did then the bundle would necessarily have to be the trivial $M \times G$. Such a space, that looks like $G$ after ``forgetting'' which point is the identity is called a \textbf{$G$-torsor}. 
		
		We give a an example for motivation:
		\begin{eg}[Frame Bundle]
			The fiber bundle of all \textbf{frames}, namely choices of bases in an $n$-dimensional space is a principal $\mathrm{GL}_n$ bundle.
		\end{eg}
		The frame bundle is generally nontrivial. 
		
		\begin{eg}
			As another example, taking $G$ to be a discrete group and $\tilde X \to X$ be the universal cover of a topological space $X$, we get that $\tilde X$ is a principal $G$-bundle on $X$ with $G=\pi_1(X)$.
		\end{eg}
		
		Since $G$ acts transitively on the fiber, there is only one $G$ orbit and we can form the quotient $P/G$ in a well-defined way. We then have that $P/G$ is homeomorphic to $M$.
		% We state the following theorem (c.f. Chapter 9 of \cite{lee2003}).
	% 	\begin{theorem}
	% 		When $G$ is a compact Lie Group acting smoothly and freely on a manifold $M$, the orbit space $M/G$ is a topological manifold with dimension $\dim M - \dim G$ and a unique smooth structure so that $\pi: M \rightarrow M/G$ is a smooth submersion (differential is locally surjective).
	% 	\end{theorem}
	%
		\begin{cor}
			For a $G$-principal bundle $P \to M$ we get $\dim P = \dim M + \dim G$.
		\end{cor}
		
		If $M,F$ are two manifolds and $G$ has an action $G \times F \rightarrow F$, then for an open cover $\{ U_\alpha \}$ of $M$ with a map $g_{\alpha \beta}: U_{\alpha}\cap U_{\beta} \to G$ satisfying the conditions of Proposition \ref{prop:cocycle} we can construct a fiber bundle by first building the set
		\[
			X = \bigcup_\alpha U_\alpha \times F
		\]
		and quotienting out by the relation
		\[
			(x, f) \in U_\alpha \times F \sim (x', f') \in U_\beta \times F \Longleftrightarrow x=x', f=g_{\alpha \beta}(x) f' 
		\]
		
		Then $E = X/\sim$ is a fiber bundle over $M$. We can locally denote elements of $E$ by $[x,f]$ so that
		\[
			\pi(x,f) = x, ~ \psi_\alpha(x,f) = [x,f].
		\]
		
		\begin{prop}
			For a fiber bundle $\pi: E \rightarrow M$ with overlap functions $g_{\alpha \beta}: U_{\alpha \beta} \rightarrow G$ between charts, we can form a principal bundle $P$ so that 
			\[
				P = X/\sim, ~ X = \bigcup_\alpha U_\alpha \times G
			\]
		\end{prop}
		In certain contexts that we will encounter later, the $g_{\alpha \beta}$ are referred to as \textbf{clutching functions}.		%
		% Note that there was no requirement here that $G$ be compact. \\
		
		\begin{eg}
			Take $M = \cp^1$ the Riemann sphere and consider constructing a $G$-bundle over it. The Riemann sphere can be decomposed as a union of two copies of $\cc$ with overlap exactly on the cylinder $\cc^{\times}$. On each copy of $\cc$ the $G$-bundle is trivializable since $\cc$ is contractible. A clutching function would be a map $\rho: \cc^\times \to G$, and this gives rise to a principal $G$-bundle on $M$. 
		\end{eg}
		
		This discussion leads naturally to the next subsection. 
		
		% subsection principal_bundles (end)
		
		\subsection{Associated Bundles}
		Take a principal bundle $P$ and let $F$ be a space with associated automorphism $\mathrm{Aut}(F)$ so that $\rho: G \to \mathrm{Aut}(F)$ is a \emph{faithful} representation. Then $g \cdot f$ is a well-defined faithful left $G$-action.
		\begin{defn}
			Given a principal bundle $\pi: P \to M$ and group action $\rho: G \to \mathrm{Aut}(F)$, the \textbf{associated bundle} is given by taking the product space $P \times F$ and forming the quotient space:
			\[
				(P \times F) / G
			\]
			given by identifying:
			\[
				(x g, f) \sim (x, \rho(g) f).
			\]
			This is the \textbf{fiber product} $P \times_{G, \rho} F$. The projection map:
			\[
				\pi': (P \times F) / G \to M
			\]
			given by sending $(x, f)$ to $\pi(x)$ is well-defined since $\pi(x g) = \pi(x)$.
		\end{defn}
		Note that the (equivalence classes of) a coordinate bundles in section 3.1.1 gives an associated bundle.
		
		
		% \begin{equation}
% 			g \cdot ([p, h], f) = ([p, hg], \rho(g)^{-1} f)
% 		\end{equation}
% 		This is a free action as well. Then if $G$ is compact (important) we have the orbit space
% 		\begin{equation}
% 			E_\rho = P(M,G) \times F/G
% 		\end{equation}
% 		is a manifold
%
% 		\begin{theorem}
% 			The space $E_\rho$ can be made into a fiber bundle over $M$ with fiber $F$ called the \textbf{associated fiber bundle} of $P(M,G)$.
% 		\end{theorem}
% 		\begin{proof}
% 			(Following \cite{lindenhovius2011})
% 			We make $P \times F$ into a bundle by defining the projection
% 			\begin{equation}
% 				\pi_\rho ([p, h], f) = p
% 			\end{equation}
% 			and trivializations $\psi_\alpha: U_\alpha \times F \rightarrow \pi^{-1}(U)\alpha$ by
% 			\begin{equation}
% 				(\psi_\rho)_\alpha (p, f) = ([p, s_\alpha (p)] , f)
% 			\end{equation}
% 			and inverse
% 			\begin{equation}
% 				(\psi_\rho)_\alpha^{-1} ([p, g], f) = [p, \rho(g) f]
% 			\end{equation}
% 		\end{proof}
		
		Two associated bundles that we'll care about are $P \times_{\mathrm{Ad}} G$ and $P \times_{\mathrm{ad}} \frak g$. The latter will be a vector bundle known as the \textbf{adjoint bundle}.
		
		Every fiber bundle with some structure group $G$ arises as an associated bundle to some principal $G$-bundle.
		Importantly, the study of equivalence classes of $G$-bundles can be equivalently cast as a study of certain associated bundles. 
		
		\subsection{Sections and Lifts}
		
		As mentioned before, any specific smooth vector field on a manifold $M$ can be viewed as a smooth map from $M$ to the the tangent bundle of $M$: $TM$. This motivates the notion of a \textbf{section} of a fiber bundle that associates to each base point $p\in M$ an element $f$ in the fiber $E_p$. Explicitly:
		\begin{defn}[Section of a Fiber Bundle]
			A \textbf{global section} of the fiber bundle $\pi: E \rightarrow M$ is a map $s: M \rightarrow E$ so that $\pi \circ s = \text{id}$. 
			
			When we have, $s: U \subseteq M \rightarrow E$, we call $s$ a \textbf{local section}. The set of global sections is denoted by $\Gamma(M, E)$. In different contexts, this may mean sections that are continuous, smooth, holomorphic, regular, etc. For smooth sections, this space is often denoted $\Gamma^\infty(M,E)$.
		\end{defn}
		
		
		\begin{eg}
			The set of all smooth $r$-forms on $M$ is $\Gamma^\infty(M, \Lambda^r (T^* M))$ on which the structure group $G$ of $T^*M$ acts on each component. % Note the different action of the structure group on different $r$-forms is exactly what makes the components of various $r$-forms ``$r$-times covariant''.
		\end{eg}
		% When the group is fibered over the manifold, then on the local cartesian structure, we can easily pick the section $p \mapsto [p,s(p)]$.
		
		\begin{prop}
			For a principal bundle $P$, any local trivialization $\psi: U \times G \rightarrow \pi^{-1} (U)$ defines a local section by $s: p \mapsto \psi(p, e)$ and conversely any local section defines a trivialization by $\psi(p,g) = s(p) g$
		\end{prop}
		
		% By using the language of sections, we can prove the existence of lifts. That is, for a principal bundle $P(M,G)$ over $M$, and a map $\varphi: M \to N$ we can get a principal bundle over $N$ by forming the projection $\varphi \circ \pi$.
		%
		% \textbf{FIX}
		% \begin{prop}
		% 	For a principal bundle $P(M,G)$ and $\varphi: M \rightarrow N$, then $\varphi$ is smooth iff $\varphi \circ \pi$ is smooth according to the following diagram.
		% 	\[
		% 	\begin{tikzcd}
		% 		P(M,G) \arrow{d}{\pi} \arrow{rd}{\varphi \circ \pi}\\
		% 		M \arrow{r}{\varphi} & N
		% 	\end{tikzcd}
		% 	\]
		% \end{prop}
		% \begin{proof}
		% 	If $\varphi$ is smooth, then $\varphi \circ \pi$ is a composition of smooth maps. On the other hand, if $\varphi \circ \pi$ is smooth, then for each point $p$ there is a coordinate neighborhood $U_\alpha$ on which we have trivial fiber structure. Take a local section $s_\alpha$ so that $\varphi \circ \pi \circ s = \varphi|_{U_\alpha}$.
		% \end{proof}
		%
		% \begin{prop}
		% 	For $P(M,G)$ principal and $\tilde \varphi: P(M,G) \rightarrow N$ a smooth $G$-invariant map so that
		% 	\begin{equation}
		% 		\tilde \varphi(x g) = \tilde \phi (x), ~ x \in P(M,G)
		% 	\end{equation}
		% 	then there is a unique map $\phi$ induced on the base space so that the following diagram commutes:
		% 	\[
		% 	\begin{tikzcd}
		% 		P(M,G) \arrow{d}{\pi} \arrow{rd}{\tilde \varphi}\\
		% 		M \arrow[r,dashed,"\varphi"] & N
		% 	\end{tikzcd}
		% 	\]
		% 	and is given by $\tilde \phi([x, g]) = \varphi(x)$. This is well-defined.
		% \end{prop}
		
		\section{Lie Theory}
		
		Although standard knowledge on the definition of a Lie Group/Algebra is assumed, let's try to motivate the ideas within this field in a more geometric way than is often done. \\
		
		Consider a manifold $M$, and consider $\text{Vect}(M)$, the space of all smooth vector fields on $M$. For a map $\varphi: M \rightarrow N$ we have a notion of \textbf{pushforward} $\varphi_*: \text{Vect}(M) \rightarrow \text{Vect}(N)$ on vector fields given by their actions on functions as
		\[
			[\varphi_* (v)] (f) = v (\varphi^* (f))
		\]
		A smooth vector field $X$ on $M$ gives rise to \textbf{flows} that are solutions to the differential equation of motion
		\[
			\frac{d}{dt} f(\gamma(t)) = X f.
		\]
		One could argue, more strongly, that in fact the \emph{entire field} of ordinary differential equations has an interpretation as equations of motion along flows of vector fields. Such a viewpoint has brought forward the lucrative insights of symplectic geometry. 
		
		The motion along this flow is expressed as the exponential:
		\[
			f(\gamma(t)) = e^{t X} f(p), ~ p = \gamma(0)
		\]
		
		Now consider two vector fields $X,Y$ on $M$. Let $Y$ flow along $X$ so we move along $X$ giving:
		\[
			e^{tX} Y = Y(\gamma(t)) \in T_{\gamma(t)}M
		\]
		Note that the reverse flow $e^{-t X}$ maps $T_{\gamma(t)} M \to T_{\gamma(0)}M = T_pM$, so acts by pushforward on $e^{tX} Y$ equivalent to:
		\[
			e^{tX} Y e^{-tX} \in T_p
		\]
		We can compare this to $Y$ and take the local change by dividing through by $t$ as $t \to 0$, giving the Lie derivative
		\begin{equation}
			\mathcal L_X Y := \frac{e^{tX} Y e^{-tX} - Y}{t}
		\end{equation}
		It is easy to check that this is in fact antisymmetric and gives rise to a bilinear form on $\mathrm{Vect}(M)$
		\begin{equation}
			[X,Y]:= L_X Y
		\end{equation}
		A vector space endowed with such a bilinear form and satisfying the Jacobi identity is a \textbf{Lie algebra}.
		
		Most important is when $M$ itself has group structure, so is a Lie group, which we will denote by $G$. Then the vector fields on $G$ of course also form a Lie algebra, just by virtue of the manifold structure of $G$. 
		
		We state the following proposition without proof
		\begin{prop}
			Let $\varphi: G_1 \rightarrow G_2$ be a homomorphism of Lie groups, then $\varphi_*: \mathrm{Vect}(G_1) \rightarrow \mathrm{Vect}(G_2)$ is a homomorphism of Lie algebras. 
		\end{prop}
		
		For a Lie group, group elements induce automorphisms on the manifold by left multiplication, denoted $L_g$ and by right multiplication $R_g$:
		\[
			\begin{aligned}
				R_g: G \rightarrow G, ~ g: h \mapsto gh\\
				L_g: G \rightarrow G, ~ g: h \mapsto hg
			\end{aligned}
		\]
		 We have that each group element induces (by pushforward) a map between tangent spaces 
		\[
		 	\begin{aligned}
		 		(L_g)_*: T_h G \rightarrow T_{gh} G\\
				(R_g)_*: T_h G \rightarrow T_{hg} G
		 	\end{aligned}
		\]
		A vector field $X$ is left-invariant if $(L_g)_* X(h) = X(gh)$. 
		By the proposition, we get that $(L_g)_* [X, Y] = [(L_g)_* X, (L_g)_* Y]$ so these left-invariant vector fields in fact form a Lie algebra for the group. Physically, this is the set of vector fields corresponding to the isometries of $G$.
		
		In local coordinates, the commutator can be written as:
		\[
			\begin{aligned}
				X = &X^\mu \partial_\mu, ~ Y = Y^\mu \partial_\mu\\
				[X,Y] = &(X^\nu \partial_\nu Y^\mu - Y^\nu \partial_\nu X^\mu) \partial_\mu.
			\end{aligned}
		\]
		Left-invariant vectors flow in a way that is consistent with the group action:
		\[
			(L_g)_* X(e) = X(g).
		\]
		The set of all left-invariant vector fields can be uniquely extracted from their value at the identity by this rule, and in fact for any vector $x \in T_e G$, there is a corresponding left-invariant vector field $X(g) = (L_g)_* x$. Therefore the tangent space to the identity gives rise to a Lie algebra which we will call \emph{the} Lie algebra of $G$ and denote by $\frak g$. The Lie algebra of $G$ is finite dimensional when $G$ is and its dimension is equal to the dimension of $G$.
		
		
		Now because we define the Lie algebra as the ``tangent space to the identity'', it is worth asking ``how does the Lie algebra appear at a generic point, $g$, on the group?''. The idea is to bring that vector back to the identity using $G$ and see what it looks like. 
		
		This is accomplished by using the \textbf{Maurer-Cartan form} $\Theta$, which is a $\frak g$-valed 1-form on $G$ so that 
		\begin{equation}
			\Theta(g) := (L_{g^{-1}})_*.
		\end{equation}
		Note that this maps from $\mathrm{Vect}(G) \rightarrow \frak g$. It takes a vector $v$ at point $g$ and traces it back to the natural vector at the identity that would have gotten pushed forward to $v$ under $g$.
		
		\begin{prop}[Properties of $\exp$]
			For $G$ a compact and connected Lie group, with Lie algebra $\frak g$, we have a map $\exp : \frak g \rightarrow G$.
			\begin{enumerate}
				\item $[X,Y] = 0 \Leftrightarrow e^X e^Y = e^Y e^X$
				\item The map $t \rightarrow \exp(t X)$ is a homomorphism from $\mathbb R$ to $G$.
				\item If $G$ is connected then $\exp$ generates $G$ as a group, meaning all elements can be written as some product $\exp(X_1) \dots \exp(X_n)$ for $X_i \in \frak g$
				\item If $G$ is connected and compact then $\exp$ is surjective. It is almost never injective.
			\end{enumerate}
		\end{prop}
		
		\begin{eg}
			The Lie algebra associated to the Lie group $\mathrm U(n)$ of unitary matrices is $\frak u(n)$ of antihermitian matrices. This is the same as the Lie algebra for the group $\mathrm{SU}(n)$
		\end{eg}
		
		\begin{defn}[Adjoint Action on $G$]
			For each $g$ we can consider the homomorphism $\mathrm{Ad}_g: h \mapsto g h g^{-1}$ or $\mathrm{Ad}_g = L_g \circ R_{g^{-1}}$. This defines a representation
			\[
				\mathrm{Ad}: g \rightarrow \mathrm{Diff}(G)
			\]
		\end{defn}
		
		\begin{defn}[Adjoint Representation of $\frak g$]
			The pushforward of this action gives rise to the \textbf{adjoint representation} of the Lie group $\frak g$ by
			\[
				(\mathrm{Ad}_g)_* = (L_g \circ R_{g^{-1}})_* 
			\]
			From the product rule, this acts as $[g, -]$ at the identity. We denote this as
			\[
				\mathrm{ad}: \frak g \rightarrow \mathrm{End}~ \frak g
			\]
		\end{defn}
		
		The Jacobi identity ensures that $\mathrm{ad}$ is a homomorphism. If the center of $G$ is zero then $\mathrm{ad}$ is faithful and we have an embedding into $\mathrm{GL}(n)$. This is nice because it also shows that after a central extension, every Lie algebra can be represented into $\mathrm{GL}(n)$, a weaker form of Ado's theorem.
		
		Moreover the adjoint representation gives rise to a natural metric on $\gg$ called the \textbf{Killing Form} given by
		\begin{equation}
			\kappa (X, Y) := \mathrm{Tr}(\mathrm{ad}(X) \mathrm{ad}(Y))
		\end{equation}
		
		\begin{prop}
			For $\frak g$ a semisimple Lie algebra, the above gives rise to a non-degenerate bilinear form. 
		\end{prop}
		For a proof see \cite{humphreys2012}.
		
		\section{The Group of Gauge Transformations}
		
		We use the ideas from the section on gauge transformation in \cite{lindenhovius2011} to build the following definition
		\begin{defn}[Gauge Transformation]
			Let $P$ be a principle bundle over $M$ with structure group $G$. A diffeomorphism $\Phi: P \to P$ is a \textbf{gauge transformation} if it satisfies the following two properties
			\begin{itemize}
				\item $\Phi$ preserves fibers so that the following diagram commutes
				\[
					\begin{tikzcd}
						P \arrow[rr,"\Phi"] \arrow[rd,"\pi"] & & P \arrow[ld,"\pi"]\\
						& M &
					\end{tikzcd}
				\]
				\item $\Phi$ commutes with the right $G$ action on $P$. 
			\end{itemize}
			Diffeomorphisms satisfying these conditions form a group referred to as the \textbf{group of gauge transformations}\footnote{Some authors may refer to this as the \emph{gauge group}. For us, the gauge group will be the $G$ we started with while this (much larger) group will be denoted $\mathcal G$.}. 
		\end{defn}
		
		\section{Connections on Principal Bundles} % (fold)
		\label{sec:connections_on_principal_bundles}
		
		There are several different and equivalent ways to characterize the notion of a \textbf{connection} on a principal $G$-bundle. We will explore two prominent ones in this section. 
		
		\subsection{The Ehresman Connection}
		
		
		Take a $G$-principal bundle $\pi: P \rightarrow M$. 
		Just as $\xi \in \frak g$ gives rise to a vector field $X_\xi$ on $G$, it also canonically gives rise to a vector field $\sigma(\xi)$ on $P$.
		\begin{defn}[\textbf{Fundamental Vector Field} of $\xi$]
			Let $\xi \in \frak g$ and consider $\exp(t \xi) \in G$ so that for $p \in P$ we get $c_p(t) = R_{\exp(t \xi)} p$ which depends smoothly on $p$. Note $c'_p(0) \in T_p P$ at each point.
			\[
			\sigma: \frak g \rightarrow \text{Vect}(P), ~ [\sigma(\xi)](p) \mapsto \left[ \frac{d}{dt} p e^{t \xi}\right]_{t=0}
			\]
		\end{defn}
		\begin{defn}[Vertical Subspace]
			The \textbf{vertical subspace} $V_p P$ at a point $p$ of a fiber bundle is the tangent space at $p$ restricted to the fiber over $x$, i.e. $T_p (\pi^{-1}(x))|_{F_x}$. Equivalently, this is $\ker \pi_*$.
		\end{defn}
		Note, by virtue of $\sigma(\xi)$ lying along the $G$-fiber. 
		\[
		\pi_* \circ \sigma(x) = \frac{d}{dt} (\pi \circ c_p(t))|_{t=0} = \frac{d}{dt} (p) = 0
		\]
		so $\sigma(x) \in V_p P$. 
		Since $P$ is a manifold of dimension $\dim M + \dim G$, $\pi_*: T_pP \rightarrow T_{\pi(p)}M$ has a kernel of dimension $\dim G = \dim \frak g$
		In fact:
		\begin{prop}
			$\sigma_p$ is a Lie algebra isomorphism between $\frak g$ and $V_pP$.
		\end{prop}
		\begin{proof}
		Since $G$ acts freely on principal bundles, $\sigma$ is injective, so in fact it must be an isomorphism.
		\end{proof}
	
		\begin{lemma}[Properties of $\sigma$]
			We get that $\sigma$ satisfies:
			\begin{enumerate}
				\item $[R_{g}]_* \sigma(x) = \sigma(\text{ad}_{g^{-1}} x)$,
				\item $[g_i]_* \sigma(x)|_p = g_i(p) x$.
			\end{enumerate}
		\end{lemma}
		\begin{proof}
			\begin{enumerate}
				\item We have
				\[
				\begin{aligned}
					\left[ R_{g} \right]_* [\sigma(x)](p)  &= \frac{d}{dt} (R_g p e^{tx}) \\
					& = \frac{d}{dt} p g \text{Ad}_{g^{-1}} e^{tx}\\ 
					& = \frac{d}{dt} p g \exp[ t (\text{ad}_{g^{-1}} x) ]\\
					& = [\sigma(\text{ad}_{g^{-1}} x)] (pg).
				\end{aligned}
				\]
				\item Secondly,
				\[
				\begin{aligned}
					\left[g_i\right]_* [\sigma(x)]|_p &= \frac{d}{dt} g_i p e^{t x}\\
										&= g_i(p) x.
				\end{aligned}	
				\]
			\end{enumerate}
		\end{proof}
	
		Now $\sigma$ respects the Lie algebra structure and forms a homomorphism from $\frak g$ to $\text{Vect}(P)$ so that in fact
		\begin{cor}\label{cor:verticalequiv}
			$(R_g)_* V_p = V_{pg}$: pushforward acts equivariantly on vertical subspaces.
		\end{cor}
		\begin{proof}
			Let $X(p) \in V_p$ pick $A \in \frak g$ so that the corresponding fundamental vector field is $\sigma(A) (p) = X(p)$. Then we just look at
			\[
				(R_g)_* \sigma(A) (p) = \sigma(\mathrm{ad}_{g^{-1}} A)(pg)
			\]
			which is vertical. It's easy to go back from $pg$ to $g$ as well by picking $A \in \frak g$ so that $X(pg) = \mathrm{ad}_{g^{-1}} A$.
		\end{proof}
	
		Now note:
		\[
		\begin{tikzcd}
			0 \arrow{r} & V_p P \arrow{r} & T_p P \arrow[r,"\pi_*"] & T_{\pi(p)}M \arrow{r} & 0
		\end{tikzcd}	
		\]
		An injection of $T_{\pi(p)} M$ into $T_p P$ to make the above sequence split is called a \textbf{horizontal subspace} at $p$ $H_pP$. 
		\begin{defn}[Horizontal Subspace]
			A horizontal subspace is a subspace $H_p P$ of $T_p P$ such that
			\[
				T_p P = V_p P \oplus H_p P.
			\]
		\end{defn}
		We'll abbreviate this by $H_p$ and the vertical subspace by $V_p$ when our principal bundle is unambiguous.
	 
		Crucially, there is \emph{no canonical choice of} $H_p$, reflecting the physical fact there is no ``god-given'' way to compare local gauges between different points. For a given gauge, a vector on $T_x M$ should lift to a vector on $T_p P$ for some $p$ corresponding to that gauge choice. The lift will lie in a horizontal subspace which will depend on the gauge choice in an infinitesimal neighborhood around $x$. A global choice of horizontal subspace gives rise to the following:
		\begin{defn}\label{def:ehresman}
		An \textbf{Ehresmann connection} is a choice of horizontal subspace at each point $p \in P$ so that
			\begin{enumerate}
				\item Any smooth vector field $X$ splits as a sum of two smooth vector fields: a \textbf{vertical field} $X_V$ and a \textbf{horizontal field} $X_H$ so that at each point $p \in P$ we have $X_V \in V_p$, $X_H \in H_p$. That is, the choice of $H_p$ varies smoothly.
				\item $G$ acts equivariantly on $H_{pg}$:
				\[
					H_{pg} = (R_{g})_* H_p.
				\]
			\end{enumerate}
		\end{defn}
	
		We will denote the collection of our choice of $H_p P$ by $HP$ and similarly define $VP$ to be the (always canonical) collection of vertical subspaces. We say any vector field can be split into a vector field $X^H \in HP$ and $X^V \in VP$.
	
		Naturally, for any choice of $HP$, we have a corresponding projection operator $\pi_H$ on vector fields $\pi_H: \mathrm{Vect}P \rightarrow HP$  and similarly $\pi_V = id - \pi_H$, both with corresponding equivariance conditions.
		
		Note that $1 - \pi_*$ acts on $T_p P$ as a projection operator onto the vertical subspace $V_p P$. A choice of horizontal subspace gives us an analogous projection operator $H$ acting on $T_p P$, mapping to the horizontal subspace. Moreover it is easy to check that (by the equivariance of the horizontal subspace), we must have
		\[
			[R_g]_* \circ H = H \circ [R_g]_*.
		\]
		
		Just as $\pi_*$ killed the vertical subspace, given a horizontal subspace, we would like to construct a similar operator that kills off the horizontal component and acts as a projection onto the vertical component. This role will be played by a \textbf{connection 1-form}

		\begin{defn}[Connection 1-form on Principal Bundle]\label{def:connection}
			A connection 1-form, $\omega$, is an element of $\Omega^1(P, \frak g)$ satisfying:
			\begin{itemize}
				\item $\omega \circ \sigma = id$, that is $\sigma \circ \omega$ is a projection onto $V_p$.
				\item $R_g^* \omega = \ad_{g^{-1}} \omega$.
			\end{itemize}
		\end{defn}
		The first condition says that, after identifying the vertical subspace with $\frak g$, $\omega$ acts trivially. The second condition is standard equivariance, since the right ad-action of $G$ on $\omega$ would exactly look like $g^{-1} \omega g$. 
	
		We now have the following equivalence:
		\begin{prop}
			A choice of Ehresman connection on $P$ is in one-to-one correspondence with the choice of a connection 1-form on $P$
		\end{prop}
		\begin{proof}
			Given a projection operator $H$ to the horizontal subspace satisfying the equivariance properties of \ref{def:ehresman}, $H_p P$, $1-H$ would be a projector onto the vertical subspace. Then $\sigma^{-1} (1- H)$ gives us a functional on $T_{p}$ valued in $\frak g$ that satisfies the properties of Definition \ref{def:connection}. This is our 1-form.
			
			Conversely, given a connection 1-form $\omega$, $1-\sigma \circ \omega$ gives us a projection onto a subspace trivially intersecting the vertical subspace. By the equivariance properties of $\omega$, this subspace satisfies the equivariance conditions in \ref{def:ehresman} and so we are done. 
		\end{proof}
		
	
		We thus have the following correspondence:
		\emph{
		\begin{center}
			\begin{tabular}{c c c c}
				\shortstack{Ehresman\\ Connections $HP$} & $\longleftrightarrow$ \shortstack{Horizontal\\Projection Operators $H$} & $\longleftrightarrow$ & \shortstack{Connection\\1-forms $\omega$}
			\end{tabular}
		\end{center}}
		Each of the above are smooth on $E$, and have appropriate equivariance conditions:
		\begin{itemize}
			\item $R_g H_p = H_{pg}$: Horizontal subspaces are $G$-equivariant,
			\item $[R_g]_* H = H [R_g]$: Horizontal projection commutes with $G$ action of changing gauge,
			\item $\omega(pg) = R_g^* \omega = g^{-1} \omega(p) g$: The 1-form is $G$-covariant.
		\end{itemize}

		\subsection{Differential Forms on Principal Bundles}

		\begin{defn}
			Given two $\gg$-valued differential forms $\alpha, \beta$ of ranks $p$ and $q$ respectively their wedge product is defined as
						\[
							(\alpha \wedge \beta)(v_1, \dots, v_{p+q}) = \frac{1}{(p+q)!} \sum_{\sigma \in S_n} \mathrm{sgn}(\sigma)\, [\alpha(v_{\sigma(1)}, \dots, v_{\sigma(p)}), \beta(v_{\sigma(p+1)}, \dots, v_{\sigma(p+q)})].
						\]
		\label{defn:wedgeforms}
		\end{defn}
		
		
		The following lemmas, which we take from \cite{kobayashi1963}, are important in translating from a picture of $k$-forms on $P$ and $k$-forms on $M$.
		\begin{lemma}
			Let $\alpha$ be a $k$ form on a $G$-principal bundle $P \to M$. $\alpha$ will descend to a unique $k$-form $\overline \alpha$ on $M$ if the following are satisfied:
			\begin{itemize}
				\item $\alpha(v_1, \dots, v_k) = 0$ if $v_i$ is vertical for any $i$,
				\item $R_g^* \alpha = \alpha$, i.e. $\alpha(R_g v_1, \dots, R_g v_k)=\alpha(v_1, \dots, v_k)$.
			\end{itemize}
			In this case, we will have $\alpha = \pi^* \overline \alpha$.
			\label{lem:pushdown1}
		\end{lemma}
		\begin{proof}
			Let $\{\overline v_i\}_{i=1}^k$ be set of $k$ vectors in $T_p M$ and $\{v_i\}_{i=1}^k$ be a set of $k$ vectors in $T_x M$ for any $x \in \pi^{-1} (p)$ so that $\pi_* v_i = \overline v_i$. We define
			\[
				\overline \alpha(\overline v_1, \dots, \overline v_k) := \alpha(v_1, \dots, v_k)
			\]
			This is well-defined regardless of the choice of $\{v_i\}$ for given $\{\overline v_i\}$ since by hypothesis $\alpha$ is zero on the kernel of $\pi_*$. It is also independent of the choice of $x \in \pi^{-1}(p)$ by the hypothesis of $\alpha$'s invariance under right $G$ action.
		\end{proof}
		
		\begin{lemma}
			If $\alpha \in \Omega^1 (P, \frak g)$ descends to a form $\overline \alpha$ on $M$, then we have:
			\begin{equation}
				\dd_\omega \alpha = \dd \alpha
			\end{equation}
			\label{lem:pushdown2}
		\end{lemma}
		\begin{proof}
			This follows from the following manipulation:
			\[
				\begin{aligned}
					(\dd_\omega \alpha) (v_1, \dots, v_k) &= (\dd \alpha) (h v_1, \dots, h v_n)\\
					& = (\dd \pi^* \overline \alpha) (h v_1, \dots, h v_n)\\
					& = (\pi^* \dd \overline \alpha) (h v_1, \dots, h v_n)\\
					& = (\dd \overline \alpha) (\pi_* h v_1, \dots, \pi_* h v_n)\\
					& = (\dd \overline \alpha) (\pi_* v_1, \dots, \pi_* v_n)\\
					& = (\pi^* \dd \overline \alpha) (v_1, \dots, v_n)\\
					& = (\dd \alpha) (v_1, \dots, v_n).
				\end{aligned}
			\]
		\end{proof}

		\subsection{Holonomy} % (fold)
		\label{sub:holonomy}
		
		% subsection holonomy (end)
		
		A particularly important aspect of this thesis will be the action of Wilson loops when inserted into gauge theories. Wilson loops are defined in terms of something known as the \textbf{Holonomy}.
		
		The \textbf{concatenation} of two paths $\gamma_1, \gamma_2$ such that $\gamma_1(1) = \gamma_2(0)$ is the (piecewise smooth) curve given by
		\[
			\gamma'(t) := \begin{cases}
				\gamma_(2t)\text{ if } t \leq 1/2\\
				\gamma_(2t-1) \text{ if } 1/2 \leq t \leq 1.
			\end{cases}
		\]
		
		\begin{prop}
			Given a principal $G$-bundle $\pi:P\to M$, consider a smooth path $\gamma: [0, 1] \to M$. Given a point $p \in \pi^{-1}(\gamma(0))$, there is a unique lift $\tilde \gamma$ so that $\pi(\tilde \gamma) = \gamma$ and $\tilde \gamma'(t) \in H_{\tilde \gamma'(t)}P$. This is called the \textbf{horizontal lift} of $\gamma$
		\end{prop}
		\begin{proof}
			The result follows by noting that the condition that the lift be horizontal is a first order differential equation with unique specified initial conditions. By smoothness, there exists a unique solution.
		\end{proof}
		
		This can be generalized to piecewise smooth curves similarly.
		
		\begin{defn}
			The \textbf{holonomy group} for the connection $\omega$ at point $p \in P$, denoted $\mathrm{Hol}_p(\omega)$,  is the subgroup of $G$ consisting of elements that are holonomies around some loop $\gamma \subseteq M$.
			
			The \textbf{restricted holonomy group} $\mathrm{Hol}_p^0(\omega)$ is analogous, but considers only curves that are \emph{contractible}.
		\end{defn}
		Note that both of these are indeed subgroups, with multiplication of elements corresponding to the concatenation of the associated loops.
		
		A connection is called \textbf{irreducible} if the centralizer of the holonomy group in $G$ is precisely the center $Z(G)$.
		
		\section{Chern-Weil Theory}
		
		In physics, relevant quantities such as the action, the instanton number, and the gauge field Lagrangian are expressed in terms of polynomials of the field strength $F$. Mathematically, Chern-Weil theory is concerned with the study of polynomials of the curvature form $\Omega$ on the associated principal $G$-bundle. These can be related to the cohomology classes of $M$.
		
		\subsection{Symmetric Invariant Polynomials on $\frak g$}
		
		Consider $\gg$ as an affine algebraic variety ($\cong \cc^{\dim \gg}$), and consider the ring of functions $\cc [\gg]$. Since $G \lacts \gg$ by $\Ad_G$-action, we naturally have a $G$-action on this space of polynomials
		\[
			\cc [\gg] \racts G.
		\]
		Taking $f(x) \to f(\Ad_g x)$. Polynomials that are fixed by this action are called \textbf{invariant polynomials} on $\gg$, and are denoted by $\cc[\gg]^G$.
		
		\begin{eg}
			Take $\gg = \gl_n$. The following are invariant polynomials on $\gg$:
			\begin{itemize}
				\item $\tr x^n$ for any $n \in \zz^+$,
				\item $\det (x - \lambda \cdot 1)$ for any  $\lambda \in \cc$.
			\end{itemize}
			Invariance under $x \to gxg^{-1}$ follows from the cyclic properties of the trace in the first case and the fact that the determinant map is a homomorphism in the second case.
		\end{eg}
		
		\begin{defn}
			A polynomial $f$ on $\cc[\gg]$ is called \textbf{homogenous} of degree $k$ if $f(a x) = a^k f(x)$ for $x \in \gg, a \in \cc$.
		\end{defn}
		
		\begin{obs}
			A homogenous degree $k$ polynomial corresponds to an element of $\mathrm{Sym}^k (\gg^*)$: a $k$-linear symmetric functional $f: \prod_{i=1}^k \gg \to \cc$.
		\end{obs}
		
		We ask what it would mean to apply $f$ to the $\gg$-valued 2-form $\Omega$. By using Definition \ref{defn:wedgeforms} to construct a $k$-fold wedge products of 2-forms, we get a $2k$ form:
		\[
			f(\Omega)(v_1, \dots, v_{2k}) = \frac{1}{(2k)!} \sum_{\sigma \in S_n} \mathrm{sgn}(\sigma)\, f\left(\Omega(v_{\sigma(1)}, \dots, v_{\sigma(2)}), \dots ,\Omega(v_{\sigma(2k-1)}, \dots, v_{\sigma(2k)})\right).
		\]
		We now note that $f(\Omega)$ satisfies the requirements of Lemmas \ref{lem:pushdown1} and \ref{lem:pushdown2} so that 
		\[
			\dd f(\Omega) = \dd_\omega f(\Omega).
		\]
		Since $\dd_\omega$ acts as a graded derivation
		\[
			\dd_\omega (\alpha \wedge \beta) = (\dd_\omega \alpha) \wedge \beta + (-1)^{|\alpha|} \alpha \wedge (\dd_\omega \beta),
		\]
		and since $\dd_\omega \Omega = 0$ we get that $f(\Omega)$ is closed. Further, since $f(\Omega)$ descends to a $2k$-form $\overline{f(\Omega)}$, we get a closed $2k$ form on $M$, so that
		\begin{equation}
			[\overline{f(\Omega)}] \in H^{2k} (M).
		\end{equation}
		We formulate the following proposition:
		\begin{theorem}[Chern-Weil]
			Let $f$ be an invariant homogenous polynomial of degree $k$ on $\frak g$ and $\Omega$ be the curvature 2-form associated to some connection $\omega$ on a principle bundle $P$. Then $\overline{f(\Omega)}$ is a representative of a cocycle class in $H^{2k}(M)$ independent of the choice of connection.
		\label{thm:chern-weil}
		\end{theorem}
		\begin{proof}(Adopted from \cite{matthew2012})
			We have proved everything other than connection independence. For this, let $\omega_0, \omega_1$ be two different connection 1-forms on $P$. We can perform a homotopy and use the fact that cohomology is homotopy invariant. Consider $P$ as a principal $G$-bundle on $M \times [0, 1]$ and let $\omega':= t p^* \omega_0 + (1-t)  p^* \omega_1$ be the 1-form given by pulling back the appropriate combination of $\omega_0$ and $\omega_1$. Then using $\iota_t: M \to M \times [0,1]$ sending $p \to (p, t)$, $f(\Omega')$ can be pulled back from $M \times [0, 1]$ to a $2k$-form on $M$. Since $\iota_0$ and $\iota_1$ are homotopic:
			\[
				\iota_0^* \overline{f(\Omega')} \sim \iota_1^* \overline{f(\Omega')}
			\]
			must lie in the same cohomology class. This are easily seen to be equal to $\overline{f(\Omega_0)}$ and $\overline{f(\Omega_1)}$, respectively.
		\end{proof}

		We have the following corollary. 
		\begin{cor}
			For a manifold $M$, $\overline{f(\Omega_1)}$ is locally exact on each coordinate patch. The form $K$ so that $\dd K = \overline{f(\Omega_1)}$ on a given $U_\alpha$ is the \textbf{Chern-Simons} form.
		\end{cor}
		
		\subsection{Chern Classes}
		
		Let $P$ be a principal $G$ bundle and $E$ be an associated complex vector bundle on which $G$ acts nontrivially. For $G$ semisimple, this can be taken to be the adjoint bundle, but also for $G$ a classical, linear algebraic group, we can take the bundle to be associated to the defining representation. Let $n$ denote the rank of $E$.
		
		In either case, the curvature form $F \in \Omega^2(M, \gg)$ corresponding to a connection on $E$ gives rise to the following polynomial in $F$ that is easily seen to be symmetric-invariant:
		\begin{equation}
			c(F) := \det(1 - \frac{t F}{2\pi i})
		\end{equation}
		This polynomial is not homogenous, but rather splits into a sum of homogenous polynomials in even degree:
		\begin{equation}
			c(F) = 1 + t c_1(F) + t^2 c_2(F) + \dots = \sum_{k=1}^n t^k c_k(F)
		\end{equation}
		where $c_k \in \Omega^{2k}(M)$. Clearly $c_k(F) = 0$ if $2k > \dim M$.
		% where $c_i(F) \in H^{2i}(M)$ .
		
		By using simple matrix identities such as $\exp{\tr A} = \det \exp A$ one can arrive at a more explicit form of the first few of these polynomials
		\[
			c(F) = 1 + i \frac{\tr(F)}{2\pi} t + \frac{\tr(F \wedge F) - \tr(F) \wedge \tr(F)}{8\pi^2} t^2 + \dots + \frac{i \det F}{2\pi} t^n
		\]
		
		 By theorem by Theorem \ref{thm:chern-weil}, the cohomology classes $[c_i(F)]$ are independent of the connection used to define $F$. Consequently, we can define 
		 \begin{defn}[Chern class] \label{defn:chern}
		 	The \textbf{Chern classes} for the bundle $E$ are the cohomology classes in $H^*(M)$ associated with each $c_i(F)$. We write 
			\[
				c_i(E) := [c_i (F)].
			\]
			The \textbf{Chern numbers} for the bundle $E$ are given by
			\[
				c_i(E) := \int_{M} c_i(F)
			\]
			and are again independent of the connection. 
		 \end{defn}
		 
		
 		\begin{prop}
 			Let $E = \bigoplus_{j=1}^m E_j$. Then
 			\[
 				c(E) = c(E_1) \cdot \dots \cdot c(E_m)
 			\]
 		\end{prop}
 		\begin{proof}
 			Because the gauge group acts as block matrices, the field strength tensor can be decomposed into blocks acting separately on each $E_j$ so that the determinant factors:
 			\[
 				\det\left(I - \frac{t F}{2\pi i}\right) = \det\left(I - \frac{t F_1}{2\pi i}\right) \wedge \dots \wedge \det \left(I - \frac{t F_m}{2\pi i}\right).
 			\]
			Thus, the associated Chern class is a cup product of the cohomology classes corresponding to each differential form in this wedge.
 		\end{proof}
		 