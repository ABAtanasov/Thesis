\chapter{The Basics of Field Theory\label{ch:phys}}

	This chapter aims to give a background into the physical ideas needed to understand the remainder of this paper
	
	\section{Classical Field Theory} % (fold)
	\label{sec:classical_field_theory}
	
	% section classical_field_theory (end)
	
	
	Here is a mathematical formulation of classical field theory:
	\begin{defn}[Classical Field Theory]
		A classical field theory $\mathcal E$ is a collection of the following data:
		\begin{itemize}
			\item A manifold $M$ known as the \textbf{spacetime} of the theory.
			\item A space $\mathrm{Map}(M, X)$ of section maps $\Phi: M \to X$, where $X$ is a Riemannian manifold called the ``target space''. Any such $\Phi$ is called a \textbf{field}.
			\item An action functional $S[\Phi]$ from the space of field configurations into $\cc$ (or more generally some number field).
		\end{itemize}
		Classical field theory studies solutions to the \textbf{classical equations of motion}
		$$\{\varphi \in \mathcal F \; \text{s.t.} \; \delta S(\varphi) = 0 \}.$$
	\end{defn}
	\begin{eg}
		When $X = \mathbb R$, we get a single scalar field $\phi$ (here $\Phi$ is $\phi$). An action for this field theory is often given by:
		$$S[\phi] = \int_M |\partial_\mu \phi|^2.$$
	\end{eg}
	\begin{eg}
		Classical electromagnetism is defined by $X=T^* M$ with an action given by:
		$$S[A] = \int_M F \wedge \star F, \qquad F := \mathrm dA.$$
		Here $F = dA$ is the \emph{curvature form} or \emph{electromagnetic field-strength tensor}.
		
		More generally, Yang-Mills theory (to be more thoroughly defined and discussed in the next section) takes $X=T^* M \otimes \frak g$ and given 
		$$S[A] = \int_M \mathrm{Tr}\left( F \wedge \star F \right), \qquad F := \mathrm dA + A \wedge A.$$
		where the trace is taken over the Lie algebra using the Killing form. 
	\end{eg}
	
	\section{Quantum Field Theory and the Operator-Product Expansion} % (fold)
	\label{sec:quantum_field_theory_and_the_operator_product_expansion}
	
		Though we do not know how to make sense of quantum field theory, the intuitive picture that we have of it is given by the \textbf{Feynman Path Integral}. For a given quantum field theory, there is quantity known as the \textbf{partition function}, defined as:
		\begin{equation}
			\mathcal Z = \int \mathcal D\Phi\, e^{- S[\Phi]}.
		\end{equation}
		This is an integral taken over the space of all fields. The measure on this space is mathematically ill-defined in general. 
		\begin{defn}[Classical Observable]
			A classical observable is a functional from the set of field configurations to the ground field $\mathbf k$.
		\end{defn}
		\begin{defn}[Observable]
			A \textbf{quantum observable} (which we will refer to as just an \emph{observable} in these lectures) is a functional from the a field theory into the ground field $\mathbf k$. In the Feynman picture, it can be seen as a statistical average of classical observables over all field configurations.
		\end{defn}
		The partition function is an observable, as is the \textbf{1-point correlation function} at a point $x_1$:
		$$\left< \Phi(x_1) \right> := \frac{1}{\mathcal Z} \int \mathcal D\Phi \, \Phi(x_1) e^{-S[\Phi]}.$$
		In this example, the path integral over all configurations of $\Phi$ probes $\Phi$ at this single point, giving essentially an expectation value. We can take expectation values of many different operators, e.g. $\phi(x_1), \partial_\mu \phi(x_1), \mathbf{1}, \phi(x_1) \partial_\mu \phi(x_1)$ on $X$. We denote operators by $\OO$. More generally, we define \textbf{correlation functions} as 
		$$\left< \mathcal O_1 \dots \mathcal O_n \right>_g := \frac{1}{\mathcal Z} \int \mathcal D\Phi \, \mathcal O_1 \dots \mathcal O_n e^{-S[\Phi]}.$$
		\begin{defn}[TQFT]
			If the correlation functions of a given quantum field theory are independent of the metric $g$, then the corresponding theory is called a \textbf{topological quantum field theory} (TQFT).
		\end{defn}
	\noindent 	In fact metric independence implies diffeomorphism invariance.
		\begin{eg}[Chern Simmons Theory]
			It turns out the correlation functions of Chern-Simmons theory on a 3-manifold $M$ with $\Phi$ being the field $A: M \to T^* M \otimes \frak g$ and the action given by
			$$S[A] \, \propto\,  \int_{M} \mathrm{Tr}\left(A \wedge dA + \frac23 A \wedge A \wedge A \right)$$
			This is clear because the metric has no role in defining the action.
		\end{eg}

		\begin{prop}[Operator Product Expansion]
			 Within the path integral, a product of two local fields can be replaced by a (possibly infinite) sum over individual fields. Namely, given two operators $\OO_a, \OO_b$ and evaluation points $x_1, x_2$, there is an open neighborhood $U$ around $x_2$ such that
			\begin{equation}
				\OO_a (x_1) \OO_b(x_2) = \sum_c C_{ab}^c(x_1-x_2) \OO_c(x_2)
			\end{equation}
		where $\OO_c$ are other operators in the quantum field theory, and the $C_{ab}^c$ are analytic functions on $U \backslash \{ x_2 \}$ (that often become singular as $x_1 \to x_2$).
		\end{prop}
	
		In the 2D case, this yields the (possibly familiar) Laurent series associated with CFT. The structure constants contain valuable information about the QFT that allow onw to view it \emph{algebraically}. In particular, they satisfy \textbf{associativity conditions}. The philosophy of the OPE is as follows: % \textbf{(elaborate Phil's point here)}.
% 		This leads naturally to the next idea
		\begin{idea}
			The OPE coefficients, together with the 1-point correlation functions completely determine the $n$-point correlation functions in a quantum field theory. 
		\end{idea}
	\noindent 	For example, a two-point function is simply given by:
		\begin{equation}
			\left< \OO_a(x_1) \OO_b(x_2) \right> = \sum_c C^c_{ab}(x_1 - x_2) \left< \mathcal O_c (x_2) \right>
		\end{equation}
	
	% section quantum_field_theory_and_the_operator_product_expansion (end)
	
	
	\section{Topological Quantum Field Theory} % (fold)
	\label{sec:topological_quantum_field_theory}
	
	An understanding of topological quantum field theory. 
	
	In categorical language, we say:
	\begin{defn}
		A \textbf{$n$-dimensional topological quantum field theory} is a symmetric monoidal functor:
			\[
				\mathcal Z: \mathrm{Bord_n} \to \mathrm{Vect}_{\kk}
			\]
	\end{defn} 
	
	\begin{theorem}
		The category of 2-dimensional topological quantum field theories is the same as the category of Frobenius algebras.
	\end{theorem}
	
	In general, besides just considering $n$-bordisms between $n-1$ manifolds, one might also be inclined to consider the  \textbf{extended} topological quantum field theory in $n$-dimensions.
	These are difficult to define, and would in principle rely on the language of $n$-categories to give a satisfactory definition. W 
	
	% section topological_quantum_field_theory (end)

	\section{Supersymmetry} % (fold)
	\label{sec:supersymmetry}

	\begin{defn}
	A \textbf{Lie superalgebra} is a $\mathbb Z_2$-graded Lie algebra with a commutator bracket satisfying:
		$$[x, y]= -(-1)^{|x||y|} [y, x]$$
		In our case, we will be extending the familiar \emph{Poincare algebra} of $\mathrm{Lie}\{ \mathrm{SO}(3, 1) \ltimes \mathbb R^4  \}$ by $\mathcal N$ ``odd'' vectors, which transform in the fundamental representation of $\mathrm{SL(2, \mathbb C)}$, which is a projective \emph{spinor} representation of the Lorentz group. 
	\end{defn}
	The brackets between the odd vectors $\{Q^A_\alpha, Q^B_\beta \} $ give rise to various central elements $Z^{AB}$ in the algebra. These are called \emph{supercharges}:
	$$\{Q^A_\alpha, Q^B_\beta \} = \epsilon_{\alpha \beta} Z^{AB}$$
	and they satisfy
	$$Z^{AB} = -Z^{BA}$$
	So that there are a total of $\mathcal N (\mathcal N - 1)/2$ distinct supercharges in a theory with $\mathcal N$ supersymmetry generators. 
	\begin{defn}[R-symmetry group]
		The group of transformations exchanging the supercharges. For the case of $\mathcal N = 4$ this is $\mathrm{Spin}(6)$
	\end{defn}
	
	\begin{defn}[Subsector]
		Given a supersymmetry operator $Q$ s.t. $Q^2 = \frac{1}{2} [Q, Q] = 0$, we define the subsector of our theory $\mathcal E$ by the set of $Q$ invariants, and denote this as $(\mathcal E, [Q, -])$.
		
		Slightly more precisely, $[Q, -]$ defines a differential operator, and the ``observables'' become exactly those gauge-invariant quantities annihilated by $Q$ modulo those that are $Q$-exact.
	\end{defn}

	% section supersymmetry (end)
	