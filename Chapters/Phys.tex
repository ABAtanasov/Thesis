\chapter{Physics\label{ch:phys}}

	
	\section{Classical Physics and Symplectic Geometry} % (fold)
	\label{sec:classical_physics_and_symplectic_geometry}
	
	% section classical_physics_and_symplectic_geometry (end)
	
	\section{Classical Field Theory} % (fold)
	\label{sec:classical_field_theory}
	
	% section classical_field_theory (end)
	
	\section{Gauge Theory} % (fold)
	\label{sec:gauge_theory}
	
	Gauge theory will play a central role in understanding the geometric Langlands correspondence physically. The role of the group $G$ in the langlands correspondence is played by the gauge group in the physical theory. 
	
	To understand gauge theory from a mathematical perspective, let us take a manifold $M$ together with a vector bundle $E \to M$. We ask the question:
	\begin{ques}
		
	\end{ques}
	
	\begin{nb}
		Just because a bundle is topologically trivial does not mean it is flat, nor vice versa. Flatness is an algebraic condition on the curvature 2-form $F = 0$ while triviality is a global topological condition on the vector bundle in question. 
	\end{nb}
	
	% section gauge_theory (end)
	
	\section{Supersymmetry} % (fold)
	\label{sec:supersymmetry}
	
	% section supersymmetry (end)
	
	\section{The Process of Quantization} % (fold)
	\label{sec:the_process_of_quantization}
	
	% section the_process_of_quantization (end)
	
	\section{Sketch of Quantum Field Theory} % (fold)
	\label{sec:sketch_of_quantum_field_theory}
	
	% section sketch_of_quantum_field_theory (end)

	\section{Conformal Field Theory} % (fold)
	\label{sec:conformal_field_theory}
	
	% section conformal_field_theory (end)
	
	\section{$\mathcal N = 4$ Supersymmetric Yang-Mills in 4D} % (fold)
	\label{sec:n4susy}
	
	% section n4susy (end)