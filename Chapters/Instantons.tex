\chapter{Instantons and the ADHM Construction\label{ch:instantons}}

Instantons are objects of significant interest to both physicists and mathematicians. 
For physicists, they represent \emph{classical solutions to the equations of motion}. In the context of field theory, and more specifically \emph{Yang-Mills Field Theory}, instantons correspond to nontrivial field configurations on a given spacetime manifold. 

Donaldson used the interesting mathematical properties of Yang-Mills instantons on $\mathbb R^4$ to prove novel and extremely surprising statements about the nontrivial smooth structures that can be associated to $\mathbb R^4$ uniquely among all Euclidean spaces\cite{donaldson1984}. 

A useful picture comes from quantum mechanics, of a particle in a double-well potential. Having a particle localized at the bottom of either well gives rise to a classical solution. Perturbative corrections around this minimum due to the quantum theory may give rise to harmonic-oscillator-type structure within the well, but is completely unable to account for the possibility of \emph{quantum tunneling} across the barrier into the second well of the potential. To account for this, we must understand the space of classical solutions in addition to performing perturbation theory. 

Mathematically, one way that this can manifest itself is in the fact that $e^{-1/x}$ has every higher derivative vanish as $x \to 0^+$. It is the same phenomenon that allows for the existence of \emph{bump functions} in real analysis and also for \emph{asymptotic expansions} in various areas of physics and engineering.

For the purposes of this thesis, instantons will not themselves play a central role, but their close relatives in three dimensions will: magnetic monopoles. In order to understand the construction of monopoles, however, it will be important to first understand the famous self-duality equation and ADHM construction of instantons. 


\section{Instantons in Classical Yang-Mills Field Theory} % (fold)
\label{sec:instantons_in_classical_yang_mills_field_theory}

\subsection{The Equations of Motion} % (fold)
\label{sub:the_equations_of_motion}

Yang-Mills gauge theory is a gauge theory with gauge group $G = \mathrm{SU}(n)$. In four dimensions, the objects of study are $G$-bundles and associated $G$-bundles on Euclidean 4-space $M = \rr^4$. $\rr^4$ has a Riemannian metric, so we have a Hodge-star operator giving an isomorphism:
\[
	\star: \Omega^k \to \Omega^{n-k}.
\]
From the prior section, gauge theory on $\mathbb R^4$ involves a connection $1$-form $A$ transforming in the $\ad\gg$ representation. From this, we obtain the field-strength $F$, again transforming in the adjoint action, by applying the covariant exterior derivative:
\begin{equation}
	F = \dd_A A = \dd A + [A, A]
\end{equation}
Both $F$ and $\star F$ are $\frak g$-valued 2-forms. On the other hand $F \wedge \star F$ is a $\gg$-valued 4-form. Taking the trace of this over the Lie algebra gives a 4-form that can be integrated over $M$, $\tr F \wedge \star F$. This is equivalently denoted by $||F||^2$ since $\tr (F \wedge \star  F)$ corresponds exactly to the inner product norm on $\frak g$-valued 2-forms induced by the killing form.

\begin{prop}
	$\tr(F \wedge \star F)$ is gauge independent and globally defined.
\end{prop}
\begin{proof}
	Since $F$ transforms in the adjoint representation, the cyclic property of the trace gives:
	\[
		\tr(F \wedge \star F) \to \tr(g F g^{-1} \wedge g \star F g^{-1}) = \tr(F \wedge \star F).
	\]
\end{proof}
It is important to recall that the field strength corresponds to a curvature 2-form on some principal $\SU(n)$-bundle, $P$. Given such a field strength 2-form on $M$, it can be pulled back to any bundle $E$ associated to $P$. 

In Yang-Mills theory, the action is given by:
\begin{equation}
	S[A] := \frac{1}{8\pi} \int_M \tr(F \wedge \star F)
\end{equation}
We aim to find $A$ so that $S_E[A]$ is a minimum. To do this, we use standard calculus of variations. Consider a small perturbation $A + t \alpha$.
		\begin{equation*}
			\begin{aligned}
				\mathcal F[A + t \alpha] &= \dd (A + t \alpha) + A \wedge A + t [A, \alpha] + O(t^2) \\
				&= \mathcal F[A] + t (d \alpha + A \wedge \alpha)\\
				&= \mathcal F[A] + \dd_{A} \alpha
			\end{aligned}
		\end{equation*}
		so that to order $t$:
		\begin{equation*}
			\begin{aligned}
				||\mathcal F[A + t \alpha]||^2 &= ||\mathcal F[A + t \alpha]||^2 + 2 t (\mathcal F[A], \dd_{A} \alpha)\\
				&\Rightarrow (\mathcal F[A], \dd_{A} \alpha) = 0 ~ \forall \alpha.
			\end{aligned}
		\end{equation*}
		The adjoint of the covariant derivative is the codifferential $\star \dd_A \star$, so that we can equivalently write this as:
		\[
			\forall \alpha \; (\star \dd_A \star \mathcal F[A], \alpha) = 0 \Rightarrow \dd_A \star F = 0.
		\]
		Except for the case of an abelian gauge theory, these will in general give second-order nonlinear differential equations in the connection that are difficult to solve for explicitly. Though we will not be able to easily talk about general field configurations, we \emph{will} be able to talk about field configurations that are minima for the action on the principal $\SU(n)$ bundle $P$ that the theory is defined on. To do this, we must first understand a connection between a certain integral of the field strength and the topology of $P$.

% section instantons_in_classical_yang_mills_field_theory (end)

\subsection{The Instanton Number} % (fold)
\label{sub:the_instanton_number}

		Though the action is defined by $\int_M \tr(F \wedge \star F)$, we define
		\begin{defn}[Instanton Number]
			The \textbf{instanton number} $k$ for a given field configuration is given by
			\begin{equation}
				k := \int_M \tr(F \wedge F).
			\end{equation}
		\end{defn}

		Recall from the definition of Chern classes in \ref{defn:chern} that the Chern numbers are independent of the choice of connection. Recall further that the first few Chern numbers were given by:
		\[
			c_1(E) := \frac{i}{2\pi} \int_{M} \tr(F) \qquad c_2(E) := \frac{1}{8\pi^2} \int_M \left[\tr(F \wedge F) - \tr(F) \wedge \tr(F)\right]
		\]
		Note that since $\su(n)$ consists of only traceless matrices, $c_1$ vanishes, and thus for any associated bundle $\su(n)$-bundle $E$ we have:
		\[
			c_1(E) = 0 \qquad c_2(E) = \frac{1}{8\pi^2} \int_M \tr(F \wedge F) = k.
		\]
		Thus in our case, the instanton number is simply the second Chern class, and in particular is a \emph{topological invariant of the bundle $E$, independent of the connection}.
		
% subsection the_instanton_number (end)

\subsection{The ASD Equations} % (fold)
\label{sub:the_asd_equations}

	We are now in a place where we can understand the equations defining the local minima of the action. Note first by basic properties of $\star$ that
	\begin{equation}
		\star \star: \Omega^2(M, \gg) \to \Omega^2(M, \gg)
	\end{equation}
	is equal to $1$ for $M = \rr^4$. This means that this operator has two eigenspaces corresponding to $+1$ and $-1$, giving a decomposition
	\begin{equation}
		\Omega^2(M, \gg) = \Omega^2(M, \gg)^+ \oplus \Omega^2(M, \gg)^-.
	\end{equation}
	So in general $F$ can be expressed as a sum $F = F_+ + F_-$ of 2-forms in these two spaces. Moreover since these two spaces are orthogonal by the Hermiticity of $\star$, $(F_+, F_-) = 0$. On one hand, then:
	\[
	\begin{aligned}
		S[A] &= \int_M \tr(F \wedge \star F)\\
			 &= \int_M \tr((F_+ + F_-) \wedge \star(F_+ + F_-))\\
			 &= \int_M \tr(F_+ \wedge \star F_+) + \int_M \tr(F_+ \wedge \star F_+)\\
	\end{aligned}
	\]
	Note that the action integral is the integral of $||F||^2$ is is necessarily positive. Now consider the following manipulation:
	\[
	\begin{aligned}
		8\pi^2 k &= \int_M \tr(F \wedge F) \\
		  &= \int_M \tr((F_+ + F_-)\wedge (F_+ + F_-)) \\
		  &= \int_M \tr(F_+ \wedge F_+) + \int_M \tr(F_- \wedge F_-) \\
		  &= \int_M \tr(F_+ \wedge F_+) + \int_M \tr(F_- \wedge F_-) \\
		  &= \int_M \tr(F_+ \wedge \star F_+) - \int_M \tr(F_- \wedge \star F_-)\\
		  &= \int_M ||F_+||^2 - \int_M ||F_-||^2.
	\end{aligned}
	\]
	Using the triangle inequality we get:
	\begin{equation}
		S[A] \geq |8 \pi^2 k|.
	\end{equation}
	It is also easy to see that equality will be satisfied iff $F = F_+$ or $F=F_-$.
	
	We thus have the \textbf{anti-self-dual equations} for instantons:
	\begin{equation}\label{eq:asd1}
		\star F = - F,
	\end{equation}
	or component-wise:
	\begin{equation}
		\begin{aligned}
			F_{12} + F_{34} &= 0\\
			F_{14} + F_{23} &= 0\\
			F_{14} + F_{32} &= 0.
		\end{aligned}
	\end{equation}
	% Since $F_{12} := \dd_A A = [\dd_A, \dd_A]$
	
	We see that the instanton number depends on the principal bundle, and that the instanton number of the trivial bundle is zero. 
	
	\begin{nb}
		$\su(n)$-instantons do not exist in Minkowski space $\rr^{3,1}$, since $\star^2 = -1$ would have eigenvalues $\pm i$ and $F = \pm i F$ would contradict that $F$ is a real object as an $\su(n)$-valued 2-form.
	\end{nb}

% subsection the_asd_equations (end)


	\subsection{Classifying Principal Bundles over $S^4$} % (fold)
	\label{sub:classifying_principal_bundles_over_s_4}

		In our above analysis, and the construction of instantons that is to follow, we make several assumptions about $F$ and $A$.
		\begin{itemize}
			\item For the above integrals to have made sense, we must require that $F(\vec x)$ decays ``sufficiently quickly'' as $|\vec x| \to \infty$.
			\item Consequently we must also have $A$ ``tend to a constant''. In the language of gauge theory, $A$ must become ``pure gauge'' $g \dd g^{-1}$ as $|\vec x| \to \infty$.
			\item We thus restrict the gauge group to consist of only \textbf{framed} gauge transformations, defined below.
		\end{itemize}
		\begin{defn}
			A framed gauge transformation on $\rr^4$ is one that tends to a constant group element as $|\vec x| \to \infty$.
		\end{defn}

		We first change the setting from $\rr^4$ to $S^4$. Because of the decay of the fields, this will still give a well-defined field strength and vector potential on $S^4$. The following argument is directly from \cite{lindenhovius2011}.
	
		We will now understand the instanton number in terms of a \emph{clutching function} defined on $S^3$ connecting the two hemispheres of $S^4$. First, note that on an open disk, the form $\tr (F \wedge F)$ (by virtue of being locally exact) can be written as 
		\[
			\dd \tr \left[F \wedge A - \frac13 A^3 \right] = \tr( F \wedge F)
		\]
		where $A^3 = A \wedge A \wedge A$.
		Now take $D_N$ and $D_S$ be two disks overlapping on $S^3$. The $G$-bundle must have an overlap function $\rho: S^3 \to G$.
	
		Now the integral becomes:
		\[
		\begin{aligned}
			8\pi k &= \int_{S^4} \tr (F \wedge F)\\
				   &= \int_{D_S} \tr (F_S \wedge F_S) + \int_{D_N} \tr (F_N \wedge F)\\
				   &= \int_{\partial D_S} \tr \left[F_S \wedge A_S - \frac13 A_S^3 \right] 
				   + \int_{\partial D_N} \tr \left[F_N \wedge A_N - \frac13 A_N^3 \right]\\
				   &= \int_{S^3} \left( \tr \left[F_S \wedge A_S - \frac13 A_S^3 \right] + \tr \left[F_N \wedge A_N - \frac13 A_N^3 \right] \right).
		\end{aligned}
		\]
		After some manipulations, changing $A_N, F_N$ to $A_S, F_S$ by transforming according to $\rho$, this all reduces to:
		\[
			k = - \frac{1}{24\pi^2} \int_{S^3} \tr((\rho \dd \rho)^3)
		\]
		and this can now be expressed as the pullback of $\rho$ acting on the Mauer-Cartan form on some $\SU(2)$-homotopic subgroup of $G$ by Bott's theorem. Hence,
		\[
			k = - \frac{1}{24\pi^2} \int_{S^3} \rho^* \tr(\Theta^3) = \frac{\deg \rho}{24} \int_{\SU(2)} \tr(\Theta^3).
		\]
		On $\SU(2)$, the triple wedge of the Mauer-Cartan form gives a volume form whose integral is exactly $24\pi^2$.

		\begin{prop}
			The homotopy classes of maps $S^3 \to \mathrm{SU}(2)$ are classified by integers.
		\end{prop}
		\begin{proof}
			This follows from noting that $\SU(2) \cong S^3$ and $\pi_3(S^3) = \zz$.
		\end{proof}
		\noindent Consequently, we have our result.
		\begin{prop}
			 The instanton number $k$ must be an integer equal to the negative of the degree of the clutching function $\rho$ defining the principal $G$-bundle on $S^4$.
		\end{prop}
		
		\noindent With the stage set, will now discuss the method for constructing \emph{all} instantons on $\rr^4$. This is the \textbf{ADHM construction} of Atiyah, Hitchin, Drinfeld, and Mannin \cite{atiyah65}.

	% subsection classifying_principal_bundles_over_s_4 (end)


\section{Construction of Instantons} % (fold)
\label{sec:construction_of_instantons}
	
	In the ADHM construction, we make use of an identification $\rr^4 \cong \cc^2$. 
	
	We will show how this construction will give a bundle $E$ over $S^4$ with topological charge $-k$. The proof that this exhaustively gives \emph{all} instantons can be found in \cite{donaldson1988}.
	
	\subsection{Holomorphic and Hermitian Vector Bundles} % (fold)
	\label{sub:holomorphic_and_hermitian_vector_bundles}
	
	\textbf{FINISH THIS}
	
	% subsection holomorphic_and_hermitian_vector_bundles (end)
	
	\subsection{The Data}
	
	Let $x_1, x_2, x_3, x_4$ parameterize a $\mathbb R^4$, and write this as $\mathbb C^2$ using $z_1 = x_2 + i x_1, z_2 = x_4 + i x_3$. In terms of the complex coordinates, we get
	\begin{equation}
		\begin{aligned}
			D_1 &= \frac{1}{2} ({\dd_A}_2 - i {\dd_A}_1)\\
			D_2 &= \frac{1}{2} ({\dd_A}_4 - i {\dd_A}_3)
		\end{aligned}
	\end{equation}
	We can express anti-self duality of $\mathcal F_{\mu \nu}$ in terms of these $D_\mu$ through two equations:
	\begin{equation}
		\begin{aligned}
			\left[ D_1, D_2 \right] &= 0\\
			[D_1, D_1^\dagger] + [D_2, D_2^\dagger] &= 0
		\end{aligned}
	\end{equation}
	
	% The idea behind ADHM is to take ``Fourier transforms'' of these $D_i$ to matrices $B_i$.
	
	\begin{defn}[ADHM System]
		Let $U$ be a $4$-dimensional space with complex structure. An \textbf{ADHM System} on $\cc^2$ is a set of linear data:
		\begin{enumerate}
			\item Vector spaces $V,W$ over $\mathbb C$ of dimensions $k,n$ respectively.
			\item Complex $k \times k$ matrices $B_1, B_2$, a $k\times n$ matrix $I$, and an $n\times k$ matrix $J$.
		\end{enumerate}
		
		We can see this diagrammatically by the following quiver:
		\[
			\begin{tikzcd}
				W \arrow[r,bend left,"I"] & V \arrow[l,bend left,"J"] \arrow[out=30,in=90,loop,swap,"B_1"] \arrow[out=330,in=270,loop,"B_2"]
			\end{tikzcd}
		\]
		
		A set of ADHM Data is an ADHM system if it satisfies the following constraints:
		\begin{enumerate}
			\item The ADHM equations:
			\begin{equation}
				\begin{aligned}
					[B_1, B_2] + IJ&=0\\
					[B_1, B_1^\dagger] + [B_2, B_2^\dagger] + II^\dagger - J^\dagger J &= 0
				\end{aligned}
			\end{equation}
			\item For $(x,y) \in  \cc^2$ with $x = (z_1, z_2), y = (w_1, w_2)$, the map:
			\begin{equation}
				\alpha_{x,y} = 
				\begin{pmatrix}
					w_2 J - w_1 I^\dagger \\
					-w_2 B_1 - w_1 B_2^\dagger - z_1 \\
					w_2 B_2 - w_1 B_1^\dagger + z_2
				\end{pmatrix}
			\end{equation}
			is injective from $V$ to $W \oplus (V\otimes \cc^2)$ while
			\[
			\beta_{x,y} = \begin{pmatrix}
				w_2 I + w_1 J^\dagger & w_2 B_2 - w_1 B_1^\dagger + z_2 & w_2 B_1 + w_1 B_2^\dagger + z_1
			\end{pmatrix}	
			\]
			is surjective from $W \oplus (V \otimes \cc^2)$ to $V$.
		\end{enumerate}
	\end{defn}
	
	It is an easy check to see
	\begin{obs}
		If $B_1, B_2, I, J$ satisfy the above conditions, then for $g \in \mathrm(k), h \in \mathrm{SU}(n)$, 
		\[
			(g B_1 g^{-1}, g B_2 g^{-1}, gI, Jg^{-1})
		\]
		also satisfies the ADHM equations.
	\end{obs}
	
	We can recast the ADHM equations into a more succinct form.
	\begin{prop}
		The ADHM equations are satisfied iff
		\[
		\begin{tikzcd}
			0 \arrow[r]&V \arrow[r,"\alpha_{x,y}"] & W \oplus (V \otimes \cc^2) \arrow[r,"\beta_{x,y}"] & V \arrow[r] & 0
		\end{tikzcd}
		\]
		is a complex, namely $\beta \circ \alpha = 0$.
	\end{prop}
	\begin{proof}
		We need both $\beta \alpha = 0$ as well as surjectivity of $\beta$ and injectivity of $\alpha$. The equation $\beta \alpha = 0$ reduces to a quadratic polynomial in the $w_1, w_2$ with the two ASD equations emerging as coefficients.
	\end{proof}
	
	% \begin{prop}
	% 	The topological charge of $E \to S^4$ is $-k$.
	% \end{prop}
	
	\begin{theorem}[ADHM construction]
		There is a one-to-one correspondence between equivalence classes of solutions to the ADHM system and gauge equivalence classes of anti-self-dual $\mathrm{SU}(n)$-connections $\mathcal A$ with instanton number $k$.
	\end{theorem}
	
	A full proof of this theorem is beyond the scope of this thesis. Nonetheless, we show how such a set of data gives rise to a 2-dimensional $\SU(n)$-associated bundle $E$ over $S^4$. 
	
	Succinctly: the only nontrivial cohomology group of this complex is $\ker \beta_{x,y}/ \mathrm{im}\, \alpha_{x,y}$. This gives a vector bundle over $\cc^2 \times \cc^2$ which can be identified with $\mathbb H^2$. An equivariance condition on the data under quaternionic action will let this descend to a vector bundle on $\mathbb{HP}^1 \cong S^4$. This 2D complex vector bundle will be associated to some appropriate principal bundle and have instanton number $k$.
	
	In quaternionic language, the ADHM equations become easier to work with. To each $x = (q_1, q_2) \in \cc^2$, we can associate a quaternionic operator acting on $\cc^2$ as:
	\begin{equation}
		(q_1, q_2) \mapsto z = \begin{pmatrix}
			\bar q_2 & -q_1\\
			\bar q_2 & q_2
		\end{pmatrix}.
	\end{equation}
	For $(q_1, q_2) \neq 0$ this is a rank two linear operator.
	
	We can we write the ADHM equations by defining an operator:
	\begin{equation}
		\Delta_{x,y} := \begin{pmatrix}
			\beta^\dagger_{x,y} & \alpha_{x,y}
		\end{pmatrix}.
	\end{equation}
	Then it is easy to see that (with $x=(z_1, z_2)$ and $y=(w_1, w_2)$)
	\begin{equation}
		\Delta_{x, y} = a w + b z
	\end{equation}
	where $w, z$ are the quaternionic matrices corresponding to the complex pairs $(w_1, w_2), (z_1, z_2)$ and 
	\begin{equation}
		a = \begin{pmatrix}
			I^\dagger & J\\
			B_2^\dagger & -B_1\\
			B_1^\dagger & B_2
		\end{pmatrix}, \qquad b = \begin{pmatrix}
			0 & 0\\
			I_k & 0\\
			0 & I_k
		\end{pmatrix}
	\end{equation}
	are the by $n+ 2n$ by $2k$ matrices, with $I_k$ here denoting the identity. We similarly have \footnote{The notation here is suggestive. $\Delta^\dagger$ is a Dirac operator, and solutions to the ADHM equations are $\Psi(x,y)$ so that $\Delta^\dagger \Psi = 0$.}
	\begin{equation}
		\Delta^\dagger_{x,y} = \begin{pmatrix}
			\beta_{x,y}\\
			\alpha^\dagger_{x,y}
		\end{pmatrix} =
		(a w  + b z)^\dagger.
	\end{equation}
	Importantly, the kernel of this operator is $\ker \beta_{x,y} \cap \ker \alpha^\dagger_{x,y}$ which can be rewritten as $\ker \beta_{x,y} \cap \mathrm{im}(\alpha)^\perp_{x,y}$. By the definition of orthogonal complement together with $\beta_{x,y} \circ \alpha_{x,y} = 0 \Rightarrow \im \alpha_{x,y} \subseteq \ker \beta_{x,y}$, this intersection is seen to be isomorphic to $\ker \beta_{x,y} / \mathrm{im}\, \alpha_{x,y}$.
	
	We see that $x, y$ can be interpretted as two quaternions on $\mathbb H^2$. We have an action of the quaternionic operators on this space by $(x, y) \to (x q, y q)$. The space $\ker \Delta^\dagger_{x, y} \to (x, y)$ gives rise to a rank two vector bundle $\tilde E$ on $\mathbb H^2$. Observe of the following equivariance condition:
	\begin{equation}
		\Delta^\dagger_{xq, yq} = (a w q + b z q)^\dagger = q^\dagger \Delta^\dagger.
	\end{equation}
	For $q \neq 0$, $q^\dagger$ maintains full rank, so the kernel of $\Delta^\dagger_{xq, yq}$ is the same as the kernel of $\Delta^\dagger_{x,y}$. This means that $\tilde E$ descends to a vector bundle on $\mathbb{HP}^1 \cong S^4$.

% section construction_of_instantons (end)


