\chapter{Instantons and the ADHM Construction\label{ch:instantons}}

Instantons are objects of significant interest to both physicists and mathematicians. 
For physicists, they represent \emph{classical solutions to the equations of motion}. In the context of field theory, and more specifically \emph{Yang-Mills Field Theory}, instantons correspond to nontrivial field configurations on a given spacetime manifold. 

Donaldson used the interesting mathematical properties of Yang-Mills instantons on $\mathbb R^4$ to prove novel and extremely surprising statements about the nontrivial smooth structures that can be associated to $\mathbb R^4$ uniquely among all Euclidean spaces\cite{donaldson1984}. 

A useful picture comes from quantum mechanics, of a particle in a double-well potential. Having a particle localized at the bottom of either well gives rise to a classical solution. Perturbative corrections around this minimum due to the quantum theory may give rise to harmonic-oscillator-type structure within the well, but is completely unable to account for the possibility of \emph{quantum tunneling} across the barrier into the second well of the potential. To account for this, we must understand the space of classical solutions in addition to performing perturbation theory. 

Mathematically, this often manifests itself in the fact that $e^{-1/x}$ has every higher derivative vanish as $x \to 0^+$. It is also the same phenomenon that allows for the existence of \emph{bump functions} in real analysis and also for \emph{asymptotic expansions} in various areas of physics and engineering.


\section{Instantons in Classical Yang-Mills Field Theory} % (fold)
\label{sec:instantons_in_classical_yang_mills_field_theory}

Yang-Mills gauge theory is a gauge theory with gauge group $G = \mathrm{SU}(n)$. In four dimensions, the objects of study are $G$-bundles and associated $G$-bundles on Euclidean 4-space $M = \rr^4$. $\rr^4$ has a Riemannian metric, so we have a Hodge-star operator giving a (metric-dependent) canonical isomorphism:
\[
	\star: \Omega^k \to \Omega^{n-k}.
\]
From the prior section, gauge theory on $\mathbb R^4$ involves a connection $1$-form $A$ transforming in the $\ad\gg$ representation. From this, we obtain the curvature form $F$, again transforming in the adjoint action, by applying the covariant exterior derivative:
\begin{equation}
	F = \dd_A A = \dd A + [A, A]
\end{equation}
Note that 

In this case, the action of the theory is given by:
\[
	S_E[A] := \frac{1}{8\pi} \int \tr(F \wedge \star F)
\]
\begin{prop}
	$\tr(F \wedge \star F)$ is gauge independent and globally defined.
\end{prop}
\begin{proof}
	Since $F$ transforms in the adjoint representation, the cyclic property of the trace gives:
	\[
		\tr(F \wedge \star F) \to \tr(g F g^{-1} \wedge g \star F g^{-1}) = \tr(F \wedge \star F)
	\]
\end{proof}
We aim to find $A$ so that $S_E[A]$ is a minimum. To do this, we use standard calculus of variations. Consider a local perturbation A + tα



\begin{defn}
	A Hermitian vector bundle $\pi: E \to M$ over a base space $M$ is a complex vector bundle $E$ over $M$ equipped with a Hermitian inner product on each fiber.
\end{defn}

\begin{defn}[Connection on a Vector Bundle]
	
\end{defn}

% section instantons_in_classical_yang_mills_field_theory (end)