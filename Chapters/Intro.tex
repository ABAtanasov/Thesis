\chapter{Introduction\label{ch:intro}}

The aim of this chapter is to give a both a conceptual and historical overview of both the Langlands program and the development of quantum and conformal field theory. The goal is not so much to develop any mathematical background so much as to illustrate to the reader \emph{why} this great web of ideas is important.

\section{The Langlands Program} % (fold)
\label{sec:the_langlands_program}

One of the greatest problems in mathematics, known as \emph{Fermat's Last Theorem}, conjectured there does not exist any integer solution to the following equation:
$$a^n + b^n = c^n, \qquad n > 2$$
that is nontrivial. A solution is nontrivial is none of $a, b, c$ is zero.

The proof of Fermat's last theorem relied on some of the most intricate mathematics developed at the end of the 20th century. The central theorem necessary for the proof of Fermat's last theorem\footnote{In fact, this theorem is strictly stronger than is necessary. It was enough for Wiles et al.\ to prove that a special family of elliptic curves is modular. The case for general elliptic curves has since been proven.} is as follows.
\begin{theorem}[Modularity Theorem for Elliptic Curves]
	Every elliptic curve is modular.
\end{theorem}

Fermat's last theorem follows from a special case of the modularity conjecture. The modularity conjecture for elliptic curves turns out to follow from a special case of a special case of the Langlands Conjectures.

% section the_langlands_program (end)

\section{The Development of Modern Field Theory} % (fold)
\label{sec:the_development_of_modern_field_theory}

% section the_development_of_modern_field_theory (end)

% Class Field Theory

% Weil conjectures

% Langlands Correspondence 

% Proofs of 


% Planck, Quantum Theory

% Dirac, electron

% Feynman, Path integral, QED

% Gell-Mann and others, Gauge Theory

% Wilson, Renormalization

% CFT in 2D, statistical systems, BPZ

% Modern CFT (Poland et al.), QFT as a renormalization flow 
% "This new modern perspective on quantum field theory has still not yet fully touched the mathematical world, whose developments have owed most of their fruit to the study of CFT alone. It is exciting and interesting to think about the role that a refined view of renormalization will play in future mathematics. That renormalization is of mathematical interest should be apparent to one who looks at the Fields Medals from 2006 to the present year. One every year has gone for using the ideas of renormalization to develop our understanding of the interplay of statistics and geometry."


