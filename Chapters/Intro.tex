\chapter{Introduction and Overview of the Langlands Program\label{ch:intro}}

The aim of this chapter is to give a conceptual and historical overview of the Langlands program from both it original number-theoretic setting as well as its geometric analogue. The goal is not so much to develop any mathematical background so much as to illustrate to the reader \emph{why} this great web of ideas is important. This chapter is more technical than those that follow.

The following two sections are adopted from the lectures and notes of \cite{Yoo18}. The third and fourth are motivations adopted from the first lecture of \cite{Yoo17} together with various ideas of \cite{Yoo18}.

\section{The Langlands Program in Number Theory} % (fold)
\label{sec:the_langlands_program_in_number_theory}

\emph{Fermat's Last Theorem}, once known as the ``greatest unsolved problem in mathematics,'' asserts that there does not exist an integral solution to
\begin{equation}
\label{eqn:fermat-last-theorem}
a^n + b^n = c^n, \qquad n > 2
\end{equation}
with $abc\neq 0$. 

The proof of Fermat's last theorem relied on some of the most intricate mathematics developed at the end of the 20th century. A crucial step towards its completion was put forward by Frey and made rigorous by Ribet and Serre. They showed that if the triple $(a,b,c)$ was a solution to (\ref{eqn:fermat-last-theorem}) for an odd prime $n=p$ (which one might assume without loss of generality), then the so-called Frey curve $y^2=x(x-a^p)(x+b^p)$ contradicted Taniyama--Shimura--Weil conjecture, now referred to as the Modularity Theorem. 
\begin{theorem}[Modularity Theorem for Elliptic Curves]
\label{thm:modularity-theorem}
	Every elliptic curve is modular.
\end{theorem}

Fermat's last theorem follows from a special case of the modularity conjecture\footnote{In fact, the modularity theorem is strictly stronger than necessary. It was enough for Wiles and Taylor to prove that a special family (the so-called semistable ones) of elliptic curves is modular. The case for general elliptic curves has since been proven by Breuil, Conrad, Diamond, and Taylor \cite{breuil2001}.}. The modularity conjecture for elliptic curves turns out to follow from a special case of a special case of the \emph{Langlands conjectures}, originally formulated by Robert Langlands in a letter to Andre Weil in 1967 \cite{langlands1967}.
More precisely, it is a corollary of the Langlands correspondence for $G = \mathrm{GL}_2$ over $\qq$. This part of the Langlands conjecture remains unproven as of May 2018. 

We give a sketch of the statement of the number-theoretic Langlands correspondence, intended towards an audience with some background in \emph{Galois theory} and the language of \emph{adeles}. 

Begin by considering the \textbf{absolute Galois group} of the rationals:
\[
	\mathrm{Gal}(\overline{\qq}/\qq),
\]
where $\overline{\qq}$ is the algebraic closure of $\qq$, consisting of all algebraic numbers. This Galois group is tremendously large. It is the profinite group obtained as an inverse limit over all finite Galois extensions of $\qq$. It is an open conjecture whether every finite group appears as a Galois group of some Galois extension. 
\begin{conj}[Inverse Galois]
	Every finite group is contained in $\mathrm{Gal}(\overline{\qq}/\qq)$.
\end{conj}
The number theoretic Langlands correspondence considers the $n$-dimensional representations of the absolute Galois group (called \emph{Galois representations}) and relates them to certain representations known as \emph{automorphic representations}. To define these latter types of representations, we first make the definition
\begin{defn}[Ring of adeles]
	The \textbf{ring of adeles} of $\qq$ is defined as 
	\[
		\mathbb A_{\qq} := \rr \times \prod_{p\, \mathrm{prime}}^{res} \qq_p,
	\]
	where $\qq_p$ denotes the $p$-adic completion of the rationals (for an introductory text to the $p$-adic numbers and valuation theory, see \cite{bachman1964}). Here $\rr$ can be viewed as the completion at $p=\infty$ and the above product is \emph{restricted} in the sense that:
	\[
		\prod_{p\, \mathrm{prime}}^{res} \qq_p := \left\{ (x_p) \in \prod_{p\, \mathrm{prime}} \qq_p \mid x_p \in \mathbb Z_p \text{ for all but finitely many } p\right\}.
	\]
\end{defn}
Let $\GL_n(\mathbb A_\qq)$ denote the set of $n \times n$ matrices with entries in $\mathbb A_\qq$. Further, because $\qq \hookrightarrow \mathbb A_\qq$ diagonally, we also have 
\[
	\GL_n(\qq) \hookrightarrow \GL_n(\mathbb A_\qq)
\]
which yields a left (and right) action\footnote{In this paper we shall use $G\lacts X$ to denote left action of $G$ on $X$ and $X \racts G$ to denote right action.}:
\[
	\GL_n(\qq) \lacts  \GL_n(\mathbb A_\qq) \racts  \GL_n(\qq).
\]
The left quotient space $\GL_n(\qq) \backslash \GL_n(\mathbb A_\qq)$ is well-defined in this case. Since $\GL_n(\qq)$ still acts by right action on this space, functions of this space form a (left) representation of $\GL_n(\qq)$
\[
	\GL_n(\qq) \lacts \mathrm{Fun}\left( \GL_n(\qq) \backslash \GL_n(\mathbb A_\qq) \right)
\]
This can be decomposed into irreducible representations, which are known as the \textbf{automorphic representations} of $\GL_n(\qq)$.
Though not absolutely precise, this is a good first-order description of what an automorphic representation is. 

\begin{idea}
	The Langlands correspondence associates to each $n$-dimensional representation of the absolute Galois group $\mathrm{Gal}(\overline \qq/\qq)$ an automorphic representation of $\GL_n(\qq)$.
\end{idea}
More than just an equivalence of sets, though, the Langlands correspondence states that a certain set of \emph{eigenvalue data} must agree on both sides. 

From the perspective of the absolute Galois group (henceforth referred to as the \emph{Galois side}), this eigenvalue data is called the \textbf{Frobenius eigenvalues} of this representation. The Frobenius automorphism $x \to x^p$ is the generator of the Galois group of any finite extension $\mathrm{Gal}(\mathbb F_q/\mathbb F_p)$. Given a finite-dimensional representation of $\mathrm{Gal}(\overline \qq/\qq)$ as well as a conjugacy class, for $p$ an \emph{unramified prime}, one can lift the Frobenius automorphism to a conjugacy class. The eigenvalues (well-defined for a given conjugacy class) of these elements are the Frobenius eigenvalues of that representation. 

From the perspective of the automorphic representations (henceforth referred to as the \emph{automorphic side}), the eigenvalue data is more difficult to describe. It relies on the construction of linear operators on the space of automorphic representations known as \textbf{Hecke Operators}. Though a full description of the Hecke eigendata is beyond the scope of this paper, we can give a rough and ``cartoonish'' picture of the most basic case of Hecke eigenvalues (c.f. \cite{miyake1971, kudla2004} for a deeper exposition). In the $\GL_2$ case, the space of automorphic representations is related to the space of modular forms on the upper half plane corresponding to quotients $\Gamma \backslash \mathbb H$ with $\Gamma$ a special type of discrete subgroup of $\SL_2(\mathbb Z)$. 

When $\Gamma = \SL_2(\mathbb Z)$, a modular form of weight $k$ can be interpretted as a function $f$ on the set of lattices in $\mathbb R^2$ so that $f(a\Lambda) = a^{-k} f(\Lambda)$.
The $m$th Hecke operator is then defined as:
\[
	T_m f (\Lambda) := m^{k-1} \sum_{[\Lambda' : \Lambda] = m} f(\Lambda')
\]
These are pairwise-commuting linear operators, and can thus be simultaneously diagonalizable. 
The modular forms that are eigenvectors for this operator are known as \textbf{Hecke eigenforms}, and their eigenvalue data is what we define as the \textbf{Hecke eigenvalues} of that representation. The story for more general subgroups $\Gamma$ gives an analogous construction but the story becomes much more involved beyond the $\GL_2$ case.


With this bare background laid out, we can make at least a parsable statement of the Langlands conjecture.

\begin{conj}[Langlands]
	To each $n$-dimensional representation of the absolute Galois group, there is a corresponding automorphic representation of $\GL_n(\qq)$ so that the Frobenius eigenvalues of the Galois representation agree with the Hecke eigenvalues of the automorphic representation.
\end{conj}

It is worth mentioning that the Langlands conjecture over $G = \mathrm{GL}_1$ is the same as what is known in number theory as \emph{class field theory} \cite{66515}. 


Many questions in number theory can be formulated in terms of questions about the nature of the absolute Galois group. On the other hand, automorphic representations can be studied using analytic methods, which would imply that deep number-theoretic data can be made accessible by studying these analytic objects.

The eigenvalue data plays a particularly important role both in the Langlands correspondence and its geometric analogue. The study of this eigenvalue data will become the study of the \emph{geometric Satake} symmetries acting on both sides of the geometric Langlands equivalence, and this thesis will explore how ideas from physics can give a concrete realization of the eigenvalue data in terms of \emph{operator insertions} in quantum field theory \cite{kapustin2006}. 

% section the_langlands_program (end)

\section{Weil's Rosetta Stone} % (fold)
\label{sec:weil_s_rosetta_stone}

The Langlands correspondence in number theory also has a close analogy for curves defined over finite fields $\mathbb F_q$. Indeed, translating the number theoretic statements of the Langlands program has over the past fifty years led to an extremely fruitful set of developments in the field of \emph{Arithmetic Geometry}. These developments have led to the famous proofs of the \emph{Weil Conjectures} and the \emph{Riemann Hypothesis over Finite Fields}. We will not discuss these developments here but refer the reader to \cite{osserman2008}. 

We \emph{will}, however, illustrate this anology to function fields over $\mathbb F_q$ to motivate the translation of the Langlands program to a more geometric setting. Consider the 1-dimensional affine space $\mathbb A^1 (\mathbb F_q)$. We have $F := \mathbb F_q(t)$ the function field on $\mathbb A^1 (\mathbb F_q)$. This will play the role analogous to the role of $\qq$ before. Before, we could complete $\qq$ at each prime $p$ to get the $p$-adics. For each point $x \in \mathbb A^1 (\mathbb F_q)$, there is a notion of a \emph{completion} for $\mathbb F_q(t)$ at $x$, and also a notion of a \emph{ring of integers} corresponding to the localization $\mathcal O_x$ at $x$.

To understand these completions, we make the following definitions.
\begin{defn}[Formal Power Series]
	Let $\kk[t]$ be a polynomial ring in one variable over a field $\kk$. The \textbf{ring of formal power series} around $x$, $\kk[[t-x]]$, is defined as the ring of all (possibly infinite) series of the form
	\[
		\sum_{n=0}^\infty a_n (t-x)^n,
	\]
\end{defn}
\begin{nb}
	There is no restriction in this ring that only finitely many $a_n$ are nonzero. 
\end{nb}

\begin{defn}[Formal Laurent Series]
	Let $\kk[t]$ be a polynomial ring in one variable over a field $\kk$. The \textbf{ring of formal Laurent series} around $x$, $\kk((t-x))$, is defined as the ring of all (possibly infinite) series of the form
	\[
		\sum_{n=-\infty}^\infty a_n (t-x)^n,
	\]
	where \emph{only finitely many $a_n, n<0$ can be nonzero}.
\end{defn}

The field $F_x$ corresponding to the completion of $F$ at $x$ can be viewed as the field of Laurent series around $x$, denoted $\mathbb F_q((t-x))$. $\mathcal O_x$ can similarly be viewed in terms of formal power series at $x$, $\mathbb F_q [[t-x]]$. With these definitions in place, we can define the ring of adeles analogously to before.

\begin{defn}[Adele Ring for $\mathbb F_q(t)$]
	The ring of adeles of $\mathbb F_q(t)$ is defined as 
	\[
		\mathbb A_{\mathbb F_q(t)} := \prod_{x \in \mathbb P^1(\mathbb F_q)}^{res} \mathbb F_{q_x}((t-x))
	\]
	and the above product is restricted as before in the sense that all but finitely many terms in this product over $x$ lie in $\mathbb F_q [[t-x]]$. Here the completion at the point at infinity corresponds to $\mathbb F_q ((1/t))$.
\end{defn}
We naturally have that 
\[
	\mathbb O_{\mathbb F_q(t)} := 	\prod_{x \in \mathbb P^1(\mathbb F_q)} \mathbb F_{q_x}[[t-x]]
\]
sits inside $\mathbb A_{\mathbb F_q(z)}$.

All of this can be generalized to the function field $F$ for a curve $C$ over $\mathbb F_p$. Here, ramification of various points on the curve becomes an issue and there is more subtlety in defining many of the above concepts. Working over a curve $C$ in this picture would correspond to working in some number field setting in the original Langlands conjecture. 

Already, for a function field of a curve $C$, the analogue of the Galois group in the unramified case is known to be the \textbf{\'etale fundamental group}, and a Galois representation would be a representation of $\pi_1^{\text{\'et}}(C,x) \to \GL_n$ in the unramified case. In analytic language for $C$ a complex curve, the \'etale fundamental group becomes the usual $\pi_1$ and a Galois representation becomes a representation of the fundamental group $\pi_1(C) \to \GL_n$. 

In the unramified case, automorphic representations correspond exactly the $\GL_n(\mathbb O_F)$-invariant functions on $\GL_n (F)\backslash \GL_n(\mathbb A_F)$. This means that the space of automorphic representations corresponds to:
\[
	\mathrm{Fun} \left(\GL_n (F)\backslash \GL_n(\mathbb A_F)/\GL_n(\mathbb O_F) \right).
\]

It is the following theorem of Weil that will be crucial to us in making a connection between with the geometric setting over $\cc$.
\begin{theorem}[Weil Uniformization]
	Take $F$ the function field for a curve $C$ over $\fq$. There is a canonical bijection as sets between
	\[
		G (F)\backslash G(\mathbb A_F)/G(\mathbb O_F)
	\]
	and the set of $G$-bundles\footnote{$G$-bundles are discussed in Section~\ref{sub:principal_bundles}.} over $C$. Moreover, there exists an \emph{algebraic stack} denoted by $\mathrm{Bun}_G(C)$ whose set of $\fq$ points are in canonical bijective correspondence with this set.
\end{theorem}

So (in the unramified case), the automorphic side is captured by functions on $\mathrm{Bun}_G(C, \fq)$. 
This set of functions admits an action by the \textbf{spherical Hecke algebra} at every closed point $x \in C$, defined as the space of compactly supported functions on the double coset space:
\[
	\mathcal H_x := \mathrm{Fun}_c (\GL_n (\mathcal O_x)\backslash \GL_n(F_x)/\GL_n(\mathcal O_x))
\]
with multiplication given by an operation known as a \textbf{convolution product} of functions. These algebras correspond to the Hecke operators described earlier. The actions of these algebras at different $x$ commute with one another, just like the Hecke operators in the first column. Consequently, they can be simultaneously diagonalized to give rise to eigenfunctions generalizing the notion of Hecke eigenforms in the modular form setting of the $\GL_2$ case. These operators yield \textbf{Hecke eigenfunction} objects on $\mathrm{Bun}_G$. In a more formal algebraic setting, related objects known as \textbf{Hecke-eigensheaves} are the associated objects of study\footnote{For an explanation about the transition between functions on this coset space and sheaves, see a reference on the \emph{function-sheaf correspondence}, e.g. \cite{shin2005}.}. This thesis will aim to explore the corresponding interpretation of this action in the context of topological field theory in physics.

\begin{table}[t]
	\centering
	\setlength\tabcolsep{3pt} 
\begin{tabularx}{\linewidth}{|>{\hsize=0.8\hsize}Y|>{\hsize=1.2\hsize}Y|Y|}
	\hline
	Number Theory & Curves over $\ff_q$ & Riemann Surfaces\\
	\hline
	\hline
	$\zz \subset \qq$ & $\ff_q[t] \subset \ff_q(t)$ & $\OO^{hol}_\cc \subset \OO^{mer}_\cc$ \\
	\hline
	$\spec \zz$ & $\mathbb A^1_{\ff_q}$ & $\cc$\\
	\hline
	$\spec \zz \cup \{\infty\}$ & $\mathbb P^1_{\ff_q}$ (projective line) & $\cp^1$ (Riemann sphere)\\
	\hline
	$p$ prime number & $x \in \mathbb A^1_{\ff_q}$ & $x \in \cc$\\
	\hline
	\hline
	$\zz_p$ ($p$-adic integers) & $\ff_q[[t-x]]$ power series around $x$ & $\cc[[z-x]]$ holomorphic on formal disk around $x$\\
	\hline
	$\qq_p$ ($p$-adic numbers) & $\ff_q((t-x))$ Laurent series around $x$ & $\cc((z-x))$ holomorphic on punctured formal disk around $x$\\
	\hline
	$\mathbb A_\qq$ (adeles) & $\mathbb A_{\ff_q}$ function field adeles & $\prod_{x \in \cc}^{res} \cc((z-x))$ restricted product of functions on all punctured disks, with all but finitely many extending to the unpunctured disk\\
	\hline
	\hline
	$F/\qq$ (number fields) & $F/\ff_q(t)$ or $\ff_q(C)/\ff_q(\mathbb P^1)$ & $C \to \cp^1$ (branched covers)\\
	\hline
	$\mathrm{Gal}(\overline{F}/F)$ & $\mathrm{Gal}(\overline{F}/F) = \pi_1^{\text{\'et}}(\spec F, \spec \overline F)$ & \\
	\hline
	& $\twoheadrightarrow \mathrm{Gal}(F^{\text{unr}}/F) = \pi_1^{\text{\'et}}(C, x)$ & $\pi_1(C, x)$\\ 
	\hline
	\hline
\end{tabularx}
\caption{Weil's \emph{Rosetta stone}}
\label{tab:rosetta}
\end{table}

Table~\ref{tab:rosetta}, based off of \cite{Yoo18} and \cite{nlab:function_field_analogy}, captures the analogy described above. This is the \emph{function field analogy}, otherwise known as Weil's \emph{Rosetta stone}. 

It is the hope and goal of this correspondence that the extremely difficult number-theoretic Langlands program might become more accessible when phrased in the language of the second or third columns of Table~\ref{tab:rosetta}. A reason to believe this might be so is because in this setting, there is powerful machinery stemming from the algebraic geometry developed by Grothendieck, Serre, Deligne, and others. This becomes a prominent force for driving our understanding of columns two and three. 

The analogy between columns one and two is especially strong, and in many cases a statement about the second column can be exactly translated over into a statement about the first. % The Langlands program for the second column would relate an $n$-dimensional representation of $\Gal(\overline F/F)$ to some special ``automorphic'' subrepresentation of $\mathrm{Fun}(\GL_n (\mathbb F)\backslash \GL_n(\mathbb A_F))$ with some associated eigenvalue data on both sides corresponding appropriately.


We are now in a place where we can attempt to discuss and motivate the third column: the geometric Langlands correspondence over $\cc$. To do this, we will begin with motivation from a different direction, namely Fourier analysis.


% section weil_s_rosetta_stone_and_geometric_langlands (end)

\section{The Fourier Transform and Pontryagin Duality} % (fold)
\label{sec:the_fourier_transform_and_pontryagin_duality}
		
In this section, we will attempt to give an alternative motivation for the geometric Langlands program as a generalized non-abelian analogue of the Fourier transform. 

First let us begin by working with a locally-compact abelian group $G$. Recall that these possess a unique (normalized) Haar measure. We make the following definition:
\begin{defn}[Unitary Character]
	For $G$ locally-compact and abelian, a \textbf{unitary character} of $G$ is a group homomorphism $\chi: G \to U(1)$.
\end{defn}
Using this definition, we define the following group, which plays a role as a \emph{dual} to $G$. It is called the \textbf{Pontryagin dual}.
\begin{defn}
	The set of all unitary characters $\chi$ together with multiplication  $\chi_1 \cdot \chi_2 \in \mathrm{Hom}(G, U(1))$ given by pointwise multiplication of characters, form an abelian group, denoted by $\widehat{G}.$
\end{defn}

\begin{eg}
	We have the following examples:
	\begin{enumerate}
		\item Let $G = S^1$, then the space of unitary characters consists precisely of these of the form $e^{inx}: G \to U(1)$. This makes $\widehat G = \mathbb Z$.
		\item Let $G = \mathbb Z$, then $\chi(1)$ determines the representation uniquely, and so $\widehat G = U(1)$.
		\item 			Let $G = \mathbb R$, then $e^{ikx} : \mathbb R \to U(1)$ is free to have $k$ vary over $\mathbb R$ so $\widehat G = \mathbb R$.
	\end{enumerate}

\end{eg}

Notice in all these cases that $\widehat{\widehat G} \cong G$. This is in fact true more generally, and we have the following theorem:
\begin{theorem}[Pontryagin Duality]
For any locally-compact abelian topological group $G$, the canonical map 
\begin{equation*}
\begin{split}
G &\to \widehat{\widehat G}\\
g &\mapsto [\chi \mapsto \chi (g)]
\end{split}
\end{equation*}
is an isomorphism.
\end{theorem}

\begin{obs}
	The space of functions\footnote{By this, we don't mean $L^2(G)$. $\mathrm{Fun}(G)$ can be taken to mean the space of \emph{tempered distributions} on $G$, defined as the continuous linear dual of the Schwartz space of functions. See \cite{arthur1989}.} on $G$, $\mathrm{Fun}(G)$ has a basis given by characters. 
\end{obs}
\begin{eg}
	We have the following examples:
	\begin{enumerate}
		\item $f: S^1 \to \mathbb C$ has $f(\theta) = \sum_{n} a_n e^{i n \theta}$. This is known as the \textbf{Fourier series}.
		\item $f: \mathbb Z \to \mathbb C$ has $f(n) = \int_{0}^{2\pi} F(\theta) e^{i n \theta}$. This is known as the \textbf{discrete time series}.
		\item $f: \mathbb R \to \mathbb R$ has $f(x) = \int_{-\infty}^\infty \widehat{f(k)} e^{ikx}$. This is known as the \textbf{Fourier transform}.
	\end{enumerate}
\end{eg}

Let us now try to generalize the ideas of the Fourier transform to a more direct case. It is useful to view the Fourier transform as letting us see two different sides of the same object. Let that object be the direct product of the group $G$ and $\hat G$. 
The reason this space is worth considering is by noting that there is a unique function on this space, which we can call the \textbf{kernel} $K: G \times \hat G \to \mathbb C$ defined by $K(g, \chi) = \chi (g)$. In the case of  $G=\mathbb R$, this function is exactly $e^{i k x}, x \in \mathbb R, k \in \widehat{ \mathbb R} = \mathbb R$, that is viewed as a function on \emph{both} time and frequency space.

This space comes with two obvious projections.
\[
\begin{tikzcd}
  & G \times \hat G \arrow[ld,"\pi_G"] \arrow[rd,swap,"\pi_{\hat G}"]&\\
G & & \hat G
\end{tikzcd}
\]
Any function $f$ on $G$ can be ``pulled back'' to a function on $G \times \hat G$, namely by ignoring the second component $f'(g, \hat g) = f(g)$. We will denote this pulled back function by $\pi_{G}^* f = f \circ \pi_G$.

Further, a suitable distribution on $G \times \hat G$ can be ``pushed forward'' to either $G$ or $\hat G$ by integrating it over $\hat G$ or $G$ respectively. We will denote these by $(\pi_G)_*$ and $(\pi_{\hat G})_*$, again respectively.

Now if $\hat f$ is a distribution on $\hat G$, we get that $\pi_{\hat G}^* \hat f$ is a distribution on $G \times \hat G$. This can be pushed forward to a function on $G$ by integrating over the $\hat G$ coordinates, but because $\pi_{\hat G}^* \hat f$ is constant on the $G$-coordinate, this function will just be a constant independent of $G$.

On the other hand, if we look at:
\begin{equation}
	f (g) := {(\pi_{G})}_* ([{\pi_{\hat G}}^* \hat f] K) = \int_{\chi \in \hat G} [(\hat f \circ \pi_{\hat G}) (g, \chi)] K(g, \chi)\, d\chi
	\label{eq:fourier}
\end{equation}
we obtain exactly the Fourier transform. For $G = \mathbb R$ this gives us:
\begin{equation}
	f(x) = \int_{\mathbb R} \widehat{f(k)} e^{ikx} dk.
\end{equation}

The reason that the Fourier transform finds so much use in practice is that it serves as an eigendecomposition for the derivative operator. More broadly, on $\mathbb R^n$, the eigenfunctions are plane waves $e^{i\vec k \cdot \vec x}$, which yield eigenvalues both under $\partial_x$ and also under the translation operator more generally $\vec x \mapsto \vec x + \vec y$. Any abelian group acts on itself by translation\footnote{Note that right and left action coincide for an abelian group.}. Consequently, it acts on the functions living on it, $\mathrm{Fun}(G)$, by translation $f(x) \to f(x - y)$. Note however that the unitary characters satisfy:
\[
	y \cdot \chi(x) = \chi(x - y) = \chi(-y) \chi(x)
\]
so that the characters \emph{diagonalize} the translation operator as an eigenbasis, exactly as $e^{ikx}$ did on the real line.

\begin{fact}
	The Fourier transform diagonalizes the action of $G$ on the space of functions $L^2(G) \cong L^2(\hat G)$.
\end{fact}

We have just treated Fourier analysis successfully for the category of locally-compact abelian groups. A natural next question is:
\begin{ques}
	How could we build upon the ideas Fourier analysis to generalize to non-abelian groups? That is, what could be the non-abelian analogue of the Fourier transform?
\end{ques}
Already, one can see that the naive ideas from before will not hold up as well. For one, translation operators no longer commute, and hence cannot be simultaneously diagonalizable with an eigenbasis of unitary characters. As we move to explore the continuous non-abelian setting, the role of the Pontryagin dual group $\hat G$ will be replaced by the Langlands dual group $\check G$.
% , and of course Pontryagin duality will become a very special case of Langlands duality.

It will turn out that to understand the Fourier transform from an algebraic perspective, we will have to appeal to \emph{categorification}, which in recent years have proved crucial in many fruitful applications.

% section the_fourier_transform_and_pontryagin_duality (end)

\section{Categorical Harmonic Analysis and Geometric Langlands} % (fold)
\label{sec:categorical_harmonic_analysis_and_geometric_langlands}

As a motivating example of both the algebraic perspective the idea of categorification mentioned in the previous chapter, we will illustrate the \textbf{Fourier-Mukai} transform. We will assume basic familiarity with the language of line bundles. 

When viewing $G$ as an object in a topological setting, namely as a topological space equipped with Haar measure, we consider the space of functions on $G$, $\mathrm{Fun}(G)$. In an algebraic setting, the study of functions on $G$ is often replaced by instead studying \emph{line bundles}, \emph{vector bundles}, or more generally \emph{(quasi-coherent) sheaves}\footnote{We will not attempt to give proper exposition to quasi-coherent sheaves or related objects. For a physicist, a good first-order intuition for coherent sheaves is a generalization of vector bundle, where rank is no longer assumed to be constant but can increase on certain sub-manifolds. Quasi-coherent sheaves then include the possibility for infinite rank of these bundles.}.

Let $A$ be an abelian variety, namely a complex torus of the form $A = \cc^g/\Lambda$ such that $A$ is also a projective variety. $A$ is called abelian because it is endowed with the group structure of this torus. We thus have a multiplication operation (along with the two projections):
\[
	\begin{tikzcd}
		& A \times A \arrow[ld,swap,"\pi_1"] \arrow[d,"\mu"] \arrow[rd,"\pi_2"]&\\
		A & A & A
	\end{tikzcd}
\]
Just like functions, line bundles can be pulled back along map between varieties. Given a line bundle $\mathcal L$ on $A$, $\mu^* \mathcal L, \pi^*_1 \mathcal L, \pi_2^* \mathcal L$ all give line bundles on $A \times A$.

A \textbf{geometric character} is a line bundle $\mathcal L$ on $A$ such that $\mu^* \mathcal L = \pi_1^* \mathcal L \otimes \pi_2^* \mathcal L$.
For geometric characters, there is a canonical isomorphism between $\mathcal L_{x+y}$ and $\mathcal L_x \otimes \mathcal L_y$ given by restricting $\mu^* \mathcal L$ to $(x,y) \in A \times A$ and noting that by definition, this must equal $\mathcal L_x \otimes \mathcal L_y$.

Further, multiplication by an element $x$ gives a map $\mu_x: A \to A$ which is the same as restricting $\mu$ to $\{x\} \times A$. Consequently, for a geometric character
\[
	\mu_x^* \mathcal L = \mathcal L_x \otimes \mathcal L.
\]
That is, the group action acts on geometric characters by tensoring each fiber with the 1D vector space of $\mathcal L$ at $x$, $\mathcal L_x$. Equivalently, it acts on the line bundle by tensoring it with the trivial line bundle with fiber canonically isomorphic to $\mathcal L_x$. Note the similarity between this property of \emph{geometric characters} and the property of ordinary \emph{characters} from before, namely $y \cdot e^{ikx} = e^{ik(x-y)} = e^{-iky} e^{ikx}$.

It turns out that the set of geometric characters together with the commutative operation $\otimes$ themselves form an abelian variety known as the \textbf{dual abelian variety} to $A$. This is denoted by
\[
	A^\vee := (\{\text{geometric characters}\}, \otimes)
\]
From our birds-eye view of what is going on, it looks like $A^\vee$ is playing an analogous role to $\hat G$ of the previous chapter. We have as before the simple diagram
\[
	\begin{tikzcd}
		& A \times A^\vee \arrow[ld,"\pi_1"] \arrow[rd,"\pi_2"]&\\
		A & & A^\vee
	\end{tikzcd}
\]
Just as on $G \times \hat G$ there was a universal function $K$ called the kernel from which the Fourier transform was defined, on $A \times A^\vee$ there is a \emph{universal bundle} known as the \textbf{Poincare line bundle} $\mathcal P$ so that:
\[
	\mathcal P_{(x, \mathcal L)} = \mathcal L_x.
\]
Note that a geometric character $\mathcal L$ on $A$ would not correspond to a line bundle on $A^\vee$ but instead to an object with a single fiber at $\mathcal L \in A^\vee$ that is zero at all other points. In more precise language, this would be the \textbf{skyscraper sheaf}\footnote{A skyscraper sheaf $\OO_x$ at a point $x$ can naively be thought of as a vector bundle that is rank zero everywhere except for a single point $x$ where it has rank $1$. It plays the same role in the algebraic setting as the Dirac delta does in the analytic one.} of $\mathcal L$ on $A^\vee$. Indeed, the natural objects to consider in place of \emph{functions/distributions on $G, \hat G$} are not line bundles on $A, A^\vee$ but rather objects known as \emph{quasi-coherent sheaves} on these spaces. For a reference about these objects, see \cite{hartshorne1977}.
\begin{concept}[Fourier-Mukai Transform]
	The Fourier-Mukai Transform is a map between the categories of quasi-coherent sheaves:
	\[
		\mathcal{FM}: \mathcal{QC}(A^\vee) \to \mathcal{QC}(A).
	\]
	In terms of the language above, it is given by:
	\[
		\mathcal F \mapsto (\pi_1)_* ([\pi_2^* \mathcal F] \otimes \mathcal P) .
	\]
	Note the similarity between this and the ``classical'' or ``decategorified'' Equation~\eqref{eq:fourier}. 
	In particular the skyscraper sheaf of $\mathcal L$ in $A^\vee$, denoted $\mathcal O_{\mathcal L}$, is mapped to 
	\[
		(\pi_1)_* ([\pi_2^* \mathcal O_{\mathcal L}] \otimes \mathcal P) = \mathcal L.
	\]
\end{concept}

The correspondences of this categorification are given in Table~\ref{tab:fourier_mukai}. Note in particular how scalars become vector spaces in this categorification, and how vector spaces become categories. 
\begin{table}
	\centering
	\begin{tabular}{c c  c}
		number & $\to$ & line (vector space in general)\\
		functions on $G$ & $\to$ & line bundles on $A$\\
		\emph{vector space} of functions/distributions & $\to$ & \emph{category} of quasi-coherent sheaves\\
		translations $g: G \to G$ & $\to$ & translations $\mu_x: A \to A$\\
		$\{e^{ikx}\}_{k \in \hat G}$ eigenbasis for translations & $\to$ & $
		\{\mathcal L\}_{\mathcal L \in A^\vee}$ eigenbasis for translations\\
		eigenvector multiplied by a number & $\to$ & eigen-bundle tensored with a line bundle\\
		$e^{ik(x+y)} = e^{ikx} e^{iky}$ & $\to$ & $\mathcal L_{x+y} \cong \mathcal L_x \otimes \mathcal L_y$\\
		delta function & $\to$ & skyscraper sheaf\\
		$\{e^{ikx}\}$ on $G$ is a delta function on $\hat G$ & $\to$ & $\mathcal L$ on $G$ is a skyscraper sheaf on $A^\vee$
	\end{tabular}
	\caption{The categorification associated to the Fourier-Mukai transform}
	\label{tab:fourier_mukai}
\end{table}
 
Everything so far discussed has been about abelian groups, though we have managed to use this categorified language to arrive at an algebraic picture of the Fourier transform. This will at least give us some motivation to give a statement of the categorical geometric Langlands conjecture. In the Langlands program, we have $G$ a reductive algebraic group. 

Our discussion of the Fourier-Mukai transform would naively lead us to formulate some sort of duality transformation taking us from quasi-coherent sheaves on $G$ to quasi-coherent sheaves on some dual group $\check G$. Because the group multiplication is not abelian, the above categorification will not make sense. The correct generalization is more subtle, and the principal geometric objects of study are not $G$ and $\check G$.

 
\begin{table}[h!]
	\centering
\begin{tabularx}{\textwidth}{|>{\hsize=0.8\hsize}Y|Y|>{\hsize=1.2\hsize}Y|}
	\hline
	& Abelian (classical) & Non-abelian (categorified)\\
	\hline
	Space of ``functions'' & $\mathrm{Fun}(G) \cong \mathrm{Fun}(\hat G)$ & $\mathcal D (\mathrm{Bun}_G) \cong \, \mathcal{QC}(\mathrm{Flat}_{\check G})$\\
	Symmetries acting & $G \lacts \mathrm{Fun}(G)$ & $\mathrm{Sat}_G \lacts \mathcal D (\mathrm{Bun}_G)$\\
	Eigenbasis & $\{e^{ikx}\}_{t \in \hat G}$ & Hecke Eigensheaves\\
	\hline
	\end{tabularx}
\caption{A loose analogy between the Fourier transform and the geometric Langlands correspondence}
\label{tab:geometric_langlands}
\end{table}


Taking a hint from the last section, we recall that the Langlands duality for function fields would take a representation of $\pi_1^{\text{\'et}}(C)$ and relate it to a eigenfunction for Hecke operators on the double coset space, corresponding to $\mathrm{Bun}_G(C)$ by the Weil's uniformization theorem. 
In the geometric picture, we should expect to take a representation of the fundamental group of our complex curve $C$, $\rho: \pi_1(C) \to \check G$ and obtain some sort of eigen-object on the space (more technically, moduli stack) of $G$ bundles over $C$. 

On the other hand, our discussion of the Fourier-Mukai transform gave us an equivalence of categories of quasi-coherent sheaves on two (dual) algebraic varieties. It can be seen that a representation of the fundamental group $\pi_1 (C) \to \check G$, known also as a \textbf{local system}, gives rise to a flat connection on a $\check G$-principal bundle over $C$, to be discussed in Section~\ref{sub:holonomy}. Viewing the set of flat connections on $\check G$-bundles $C$ as an algebraic space denoted $\mathrm{Flat}_{\check G} (C)$, a flat connection would correspond to a skyscraper sheaf at a given point $x \in \mathrm{Flat}_{\check G} (C)$. 

On the other side, we expect to be studying some categorical generalization functions on $\mathrm{Bun}_G(C)$. The appropriate object turns out to be \textbf{$\mathcal D$-modules} on this space.
A full discussion of $\mathcal D$-modules over a space $X$ is beyond the scope of this thesis, though they play a very important role in modern geometry and representation theory. In the case that the background of the reader is physics, it might be worthwhile to point out that $\mathcal D$-modules on $X$ are related to \emph{quantization} of quasi-coherent sheaves on $T^* X$. This idea is similar in character to the Hilbert space setting for quantum mechanics.

The full ``meta-conjecture'' of geometric Langlands is then:
\begin{equation}
	\mathcal D (\mathrm{Bun_G}(C)) \cong \mathcal{QC} (\mathrm{Flat}_{\check G} (C))
\end{equation}
where Satake symmetries act naturally on both sides. This is supposed to be a nonabelian analogue of the Fourier-Mukai transform, so in particular it should take a skyscraper sheaves on the right (i.e. flat $\check G$-connections on $C$) to eigenobjects on the left that are again called Hecke eigensheaves in this setting. Again, the left-hand side will be called the \emph{automorphic side} and the right-hand side will be called the \emph{Galois side}.

This original ``meta-conjecture'' was formulated by Beilinson and Drinfeld based off of the work of \cite{beilinson1991}, though even then it was not a full-fledged conjecture as it was not believed to be true in the general setting. It turns out to hold for $G$ an abelian torus, as shown in \cite{laumon1996}.
An explicit counterexample for more general $G$ was constructed by V. Lafforgue in \cite{lafforgue2009}.

As a final remark in this story: a refined version of this conjecture is given by Arinkin and Gaitsgory in \cite{arinkin2015}, involving a refinement of the quasi-coherent sheaves on the Galois side to objects known as \emph{ind-coherent} sheaves with a certain support condition. 
\begin{equation}
	\mathcal D (\mathrm{Bun_G}(C)) \cong \mathcal{IC}_{N} (\mathrm{Flat}_{\check G} (C))
\end{equation}
Though this may seem more complicated, there is reason to believe that these objects can be derived as the right ones to consider on the basis of physical grounds, c.f. \cite{elliott2017}. 

\begin{table}[h!]
	\centering
\begin{tabularx}{\textwidth}{|>{\hsize=0.8\hsize}Y|>{\hsize=1.2\hsize}Y|Y|}
	\hline
	Classical Picture & Geometric Langlands & Topologically twisted $\mathcal N=4$ theory\\
	\hline
	Space of ``functions'' & $\mathcal D (\mathrm{Bun}_G) \cong \, \mathcal{QC}(\mathrm{Flat}_{\check G})$ & \emph{Category} of boundary conditions \\
	Symmetries acting & $\mathrm{Sat}_G \lacts \mathcal D (\mathrm{Bun}_G)$ & Insertions of `t Hooft line defects\\
	Eigenbasis & Hecke Eigensheaves & Electric/Magnetic Eigenbranes\\
	\hline
	\end{tabularx}
\caption{The connection between the ideas in geometric Langlands and supersymmetric field theory, to be discussed in this thesis.}
\label{tab:langlands_and_physics}
\end{table}


Although a full discussion of the concepts that appear in Table~\ref{tab:geometric_langlands} is beyond the scope of this thesis, we can at least give the reader one ``final column'', yielding Table~\ref{tab:langlands_and_physics}. This column is intended to highlight some key points in the relationship between the concepts of geometric Langlands and physics. The original idea and motivation for a connection between the geometric Langlands program and the physics of gauge theory was first investigated by Anton Kapustin and Edward Witten in \cite{kapustin2006}.


The action of Wilson loops on the Galois side can be very easily understood using the language of holonomy and flat connections. These rely on the language of gauge theory, to be defined in Chapter~\ref{ch:gauge}. On the other hand, the action of the `t Hooft operators is much more subtle and involved. To be able to fully appreciate this, we must understand the nature of these so-called ``disorder operators'' by first understanding the well-known picture of instantons on $\mathbb R^4$ in Chapter~\ref{ch:instantons} and then restricting this to an understanding of monopoles on $\mathbb R^3$ in Chapter~\ref{ch:monopoles}. Finally, we work in the spirit of Edward Witten's paper \cite{witten2010nahm} we will make use of our understanding of monopoles and use this to understand the action of line defect operators on boundary conditions in the topologically twisted $\mathcal N=4$ theory that will be studied in Chapter~\ref{ch:finale}. 

% section categorical_harmonic_analysis_and_geometric_langlands (end)
