% ****************************************************************************************** % Dissertation template and document class for Princeton University
% Author  : Jeffrey Scott Dwoskin <jdwoskin@princeton.edu>
% Adapted from: http://www.math.princeton.edu/graduate/tex/puthesis.html
% ****************************************************************************************** %


%%% For print copies
%% set 'singlespace' option to set entire thesis to single space, and define "\printmode" to remove all hyperlinks for printed copies of the thesis. Delete all output files before changing this mode -- it will turn hyperref package on and off
%\documentclass[12pt,lot, lof, singlespace]{puthesis}
%\newcommand{\printmode}{}

%%% For the electronic copy, use doublespacing, define "\proquestmode" to use outlined links, instead of colored links. 
\documentclass[12pt,lot, lof]{puthesis}
\newcommand{\proquestmode}{}
% I prefer proquestmode to be off for electronic copies for normal use, since the colored links are less distracting. However when printed in black and white, the colored links are difficult to read. 

%%% For early drafts without some of the frontmatter
% Also see the "ifodd" command below to disable more frontmatter
%\documentclass[12pt]{puthesis}

%%%%%%%%%%%%%%%%%%%%%%%%%%%%%%%%%%%%%%%%%%%%%%%%%%%%%%%%%%%%%\
%%%% Author & title page info

\title{\textbf{
% Gauge Theory and the
$\mathcal N=4$ Supersymmetric Yang-Mills Theory
\\\vspace{-.1in} {\Large and the}
\vspace{-.1in}
\\ Geometric Langlands Correspondence}}

\submitted{Spring 2018}  % degree conferral date (January, April, June, September, or November)
\copyrightyear{2018}  % year in which the copyright is secured by publication of the dissertation.
\author{Alexander B. Atanasov}
\adviser{Philsang Yoo}  %replace with the full name of your adviser
%\departmentprefix{Program in}  % defaults to "Department of", but programs need to change this.
\department{Mathematics}

%%%%%%%%%%%%%%%%%%%%%%%%%%%%%%%%%%%%%%%%%%%%%%%%%%%%%%%%%%%%%\
%%%% Tweak float placements
% From: http://mintaka.sdsu.edu/GF/bibliog/latex/floats.html "Controlling LaTeX Floats"
% and based on: http://www.tex.ac.uk/cgi-bin/texfaq2html?label=floats
% LaTeX defaults listed at: http://people.cs.uu.nl/piet/floats/node1.html

% Alter some LaTeX defaults for better treatment of figures:
    % See p.105 of "TeX Unbound" for suggested values.
    % See pp. 199-200 of Lamport's "LaTeX" book for details.
    %   General parameters, for ALL pages:
    \renewcommand{\topfraction}{0.85}	% max fraction of floats at top
    \renewcommand{\bottomfraction}{0.6}	% max fraction of floats at bottom
    %   Parameters for TEXT pages (not float pages):
    \setcounter{topnumber}{2}
    \setcounter{bottomnumber}{2}
    \setcounter{totalnumber}{4}     % 2 may work better
    \setcounter{dbltopnumber}{2}    % for 2-column pages
    \renewcommand{\dbltopfraction}{0.66}	% fit big float above 2-col. text
    \renewcommand{\textfraction}{0.15}	% allow minimal text w. figs
    %   Parameters for FLOAT pages (not text pages):
    \renewcommand{\floatpagefraction}{0.66}	% require fuller float pages
	% N.B.: floatpagefraction MUST be less than topfraction !!
    \renewcommand{\dblfloatpagefraction}{0.66}	% require fuller float pages

% The documentclass already sets parameters to make a high penalty for widows and orphans. 

%%%%%%%%%%%%%%%%%%%%%%%%%%%%%%%%%%%%%%%%%%%%%%%%%%%%%%%%%%%%%\
%%%% Use packages

\usepackage{amsmath,amssymb,amsfonts,amsthm}
\usepackage{gensymb}
\usepackage{tikz-cd}

%%% For figures
\usepackage{graphicx}
%\usepackage{subfig,rotate}

%%% for comments
\usepackage{verbatim}

%%% For tables
\usepackage{multirow}
% Longtable lets you have tables that span multiple pages.
\usepackage{longtable}

% Booktabs produces far nicer tables than the standard LaTeX tables.
%   see: http://en.wikibooks.org/wiki/LaTeX/Tables
\usepackage{booktabs}

%set parameters for longtable:
% default caption width is 4in for longtable, but wider for normal tables
\setlength{\LTcapwidth}{\textwidth}



%%%%%%%%%%%%%%%%%%%%%%%%%%%%%%%%%%%%%%%%%%%%%%%%%%%%%%%%%%
%%% Printed vs. online formatting
\ifdefined\printmode

% Printed copy
% url package understands urls (with proper line-breaks) without hyperlinking them
\usepackage{url}


\else

\ifdefined\proquestmode
%ProQuest copy -- http://www.princeton.edu/~mudd/thesis/Submissionguide.pdf

% ProQuest requires a double spaced version (set previously). They will take an electronic copy, so we want links in the pdf, but also copies may be printed or made into microfilm in black and white, so we want outlined links instead of colored links.
\usepackage{hyperref}
\hypersetup{bookmarksnumbered}

% copy the already-set title and author to use in the pdf properties
\makeatletter
\hypersetup{pdftitle=\@title,pdfauthor=\@author}
\makeatother

\else
% Online copy

% adds internal linked references, pdf bookmarks, etc

% turn all references and citations into hyperlinks:
%  -- not for printed copies
% -- automatically includes url package
% options:
%   colorlinks makes links by coloring the text instead of putting a rectangle around the text.
\usepackage{hyperref}
\hypersetup{colorlinks,bookmarksnumbered}

% copy the already-set title and author to use in the pdf properties
\makeatletter
\hypersetup{pdftitle=\@title,pdfauthor=\@author}
\makeatother

% make the page number rather than the text be the link for ToC entries
%\hypersetup{linktocpage}
\fi % proquest or online formatting
\fi % printed or online formatting


%%%%%%%%%%%%%%%%%%%%%%%%%%%%%%%%%%%%%%%%%%%%%%%%%%%%%%%%%%%%%\
%%%% Define commands

% Define any custom commands that you want to use.
% For example, highlight notes for future edits to the thesis
%\newcommand{\todo}[1]{\textbf{\emph{TODO:}#1}}


% create an environment that will indent text
% see: http://latex.computersci.org/Reference/ListEnvironments
% 	\raggedright makes them left aligned instead of justified
\newenvironment{indenttext}{
    \begin{list}{}{ \itemsep 0in \itemindent 0in
    \labelsep 0in \labelwidth 0in
    \listparindent 0in
    \topsep 0in \partopsep 0in \parskip 0in \parsep 0in
    \leftmargin 1em \rightmargin 0in
    \raggedright
    }
    \item
  }
  {\end{list}}

% another environment that's an indented list, with no spaces between items -- if we want multiple items/lines. Useful in tables. Use \item inside the environment.
% 	\raggedright makes them left aligned instead of justified
\newenvironment{indentlist}{
    \begin{list}{}{ \itemsep 0in \itemindent 0in
    \labelsep 0in \labelwidth 0in
    \listparindent 0in
    \topsep 0in \partopsep 0in \parskip 0in \parsep 0in
    \leftmargin 1em \rightmargin 0in
    \raggedright
    }

  }
  {\end{list}}



%%%%%%%%%%%%%%%%%%%%%%%%%%%%%%%%%%%%%%%%%%%%%%%%%%%%%%%%%%%%%\
%%%% Front-matter

% For early drafts, you may want to disable some of the frontmatter. Simply change this to "\ifodd 1" to do so.
\ifodd 0
% front-matter disabled while writing chapters
\renewcommand{\maketitlepage}{}
\renewcommand*{\makecopyrightpage}{}
\renewcommand*{\makeabstract}{}

% you can just skip the \acknowledgements and \dedication commands to leave out these sections.

\else


\abstract{
% Abstract can be any length, but should be max 350 words for a Dissertation for ProQuest's print indicies (150 words for a Master's Thesis) or it will be truncated for those uses.
The aim of this thesis is to give the reader a gentle but thorough introduction to the vast web of ideas underlying the realization of the geometric Langlands correspondence in the physics of quantum field theory (QFT). It begins with a pedagogically-motivated introduction to the relevant concepts in the Langlands program, quantum field theory, and gauge theory for an audience of mathematicians or physicists. With this machinery in place, the more complicated phenomena associated with gauge theory is explored, specifically instantons, topological operators, and electric-magnetic duality. We conclude by connecting the ideas of the Langlands correspondence discussed at the start with phenomena in topologically twisted $\mathcal N = 4$ supersymmetric Yang-Mills theory (SYM) which exhibits a striking property known as $S$-duality. A large part of the goal of this thesis is to give an exposition to the language and techniques that the literature related to this topic already assumes familiarity with, so that an advanced undergraduate or early graduate student might have a good exposition into this field. 
}

\acknowledgements{
%I would like to thank...
I would like to firstly thank Professor Yoo for agreeing to be my advisor and for dedicating much of his time (not to mention his Monday lunches) to help guide me along as I worked through understanding the many ideas necessary to write this thesis. I especially want to thank him for his patience when I was struggling with the more difficult parts of this program, and for his excellent ability to get to the heart of matters in pointing out what was important for me to understand.

I would also like to thank the past professors that I have had at Yale for building my background up so that I could complete this thesis. In particular, I would like to thank Prof. David Poland for giving me the unique opportunity to get my hands dirty working with conformal theory early in my undergraduate career, and for his guidance throughout my time at Yale. Second, I would like to thank Professors Igor Frenkel and Walter Goldberger for providing me the courses that challenged me and pushed me to the furthest limits of any other academic experience at Yale.

Lastly, I would like to thank my friends and colleagues, Aaron Hillman and Andrew Saydjari, for being part of this great journey with me. I want to thank them for the numerous and deeply meaningful conversations that they engaged me with. I wish them all the best in their future scientific pursuits, and look forward to seeing them fly.
}

\dedication{To my parents.}

\fi  % disable frontmatter


%%%%%%%%%%%%%%%%%%%%%%%%%%%%%%%%%%%%%%%%%%%%%%%%%%%%%%%%%%%%%\
%%%% Hide some chapters

%%% If you want to produce a pdf that includes only certain chapters, specify them with includeonly, in addition to including all chapters below.
%\includeonly{ch-intro/chapter-intro}
%%% You can also specify multiple chapters.
%\includeonly{ch-intro/chapter-intro,ch-usage/chapter-usage}
%\includeonly{chap1,chap2,chap3}


\newtheorem{theorem}{Theorem}[section]
\newtheorem{lemma}[theorem]{Lemma}
\newtheorem{prop}[theorem]{Proposition}
\newtheorem{cor}[theorem]{Corollary}
\newtheorem{obs}[theorem]{Observation}

\theoremstyle{remark}
\newtheorem*{review}{Review}
\newtheorem*{ques}{Question}
\newtheorem*{nb}{Note}

\theoremstyle{definition}
\newtheorem{defn}[theorem]{Definition}
\newtheorem{concept}[theorem]{Concept}
\newtheorem{view}{View}
\newtheorem{cons}[theorem]{Construction}
\newtheorem{eg}[theorem]{Example}

\def\lacts{\ensuremath{%
  \reflectbox{\rotatebox[origin=c]{180}{$\circlearrowleft$}}}}
\def\racts{\ensuremath{%
    \reflectbox{\rotatebox[origin=c]{180}{$\circlearrowright$}}}}

\makeatletter
\def\@tempa#1{\@xp\@tempb\meaning#1\@nil#1}
\def\@tempb#1>#2#3 #4\@nil#5{%
  \@xp\ifx\csname#3\endcsname\mathaccent
    \@tempc#4?"7777\@nil#5%
  \else
    \PackageWarningNoLine{amsmath}{%
      Unable to redefine math accent \string#5}%
  \fi
}
\def\@tempc#1"#2#3#4#5#6\@nil#7{%
  \chardef\@tempd="#3\relax\set@mathaccent\@tempd{#7}{#2}{#4#5}}
\@tempa\widehat
\makeatother

%%%%%%%%%%%%%%%%%%%%%%%%%%%%%%%%%%%%%%%%%%%%%%%%%%%%%%%%%%%%%
%%%% Notes:

% Footnotes should be placed after punctuation.\footnote{place here.}
% Generally, place citations before the period~\cite{anotherauthor}.
% The proper usage for i.e., and e.g., include commas ``(e.g., option A, option B)''

%%%%%%%%%%%%%%%%%%%%%%%%%%%%%%%%%%%%%%%%%%%%%%%%%%%%%%%%%%%%%
%%%% Import chapters

\begin{document}

\makefrontmatter


% If you've disabled frontmatter, you can insert the toc manually
%\tableofcontents\clearpage

% \include lets us split up the document (and each include starts a new page):
\chapter{Introduction and Overview of the Langlands Program\label{ch:intro}}

The aim of this chapter is to give a gentle conceptual and historical overview of both the Langlands program and the development of quantum and conformal field theories. The goal is not so much to develop any mathematical background so much as to illustrate to the reader \emph{why} this great web of ideas is important.

The following two sections are adopted from the lectures and notes of \cite{Yoo18}. The third and fourth are motivations adopted from the first lecture of \cite{Yoo17} together with various ideas of \cite{Yoo18}.

\section{The Langlands Program in Number Theory} % (fold)
\label{sec:the_langlands_program_in_number_theory}

\emph{Fermat's Last Theorem}, once known as the ``greatest unsolved problem in mathematics,'' asserts that there does not exist an integral solution to
\begin{equation}
\label{eqn:fermat-last-theorem}
a^n + b^n = c^n, \qquad n > 2
\end{equation}
with $abc\neq 0$. 

The proof of Fermat's last theorem relied on some of the most intricate mathematics developed at the end of the 20th century. A crucial step towards its completion was put forward by Frey and made rigorous by Ribet and Serre. They showed that if the triple $(a,b,c)$ was a solution to (\ref{eqn:fermat-last-theorem}) for and odd prime $n=p$ (which one might assume without loss of generality), then the so-called Frey curve $y^2=x(x-a^p)(x+b^p)$ contradicted Taniyama-Shimura-Weil conjecture, now referred to as the Modularity Theorem. 
\begin{theorem}[Modularity Theorem for Elliptic Curves]
\label{thm:modularity-theorem}
	Every elliptic curve is modular.
\end{theorem}

Fermat's last theorem follows from a special case of the modularity conjecture. The modularity conjecture for elliptic curves turns out to follow from a special case of a special case of the \emph{Langlands conjectures}, originally formulated by Robert Langlands in a letter to Andre Weil in 1967 \cite{langlands1967}.
More precisely, it is a corollary of the Langlands correspondence for $G = \mathrm{GL}_2$ over $\qq$ \footnote{In fact, the modularity theorem is strictly stronger than necessary. It was enough for Wiles and Taylor to prove that a special family (the so-called semistable ones) of elliptic curves is modular. The case for general elliptic curves has since been proven by Breuil, Conrad, Diamond, and Taylor \cite{breuil2001}.}. This part of the Langlands conjecture remains unproven as of May 2018. 

We give a sketch of the statement of the number-theoretic Langlands correspondence, intended towards an audience with some background in \emph{Galois theory} and the language of \emph{adeles}. 

Begin by considering the \textbf{absolute Galois group} of the rationals:
\[
	\mathrm{Gal}(\overline{\qq}/\qq),
\]
where $\overline{\qq}$ is the algebraic closure of $\qq$, consisting of all algebraic numbers. This Galois group is tremendously large. It is the profinite group obtained as an inverse limit over all finite Galois extensions of $\qq$. It is an open conjecture whether every finite group appears as a Galois group of some Galois extension. 
\begin{conj}[Inverse Galois]
	Every finite group is contained in $\mathrm{Gal}(\overline{\qq}/\qq)$.
\end{conj}
The number theoretic Langlands correspondence considers the $n$-dimensional representations of the absolute Galois group (called \emph{Galois representations}) and relates them to certain representations known as \emph{automorphic representations}. To define these latter types of representations, we first make the definition
\begin{defn}[Ring of adeles]
	The \textbf{ring of adeles} of $\qq$ is defined as 
	\[
		\mathbb A_{\qq} := \rr \times \prod_{p\, \mathrm{prime}}^{res} \qq_p,
	\]
	where $\qq_p$ denotes the $p$-adic completion of the rationals \cite{bachman1964} ($\rr$ can be viewed as the completion at $p=\infty$) and the above product is \emph{restricted} in the sense that:
	\[
		\prod_{p\, \mathrm{prime}}^{res} \qq_p := \left\{ (x_p) \in \prod_{p\, \mathrm{prime}} \qq_p \mid x_p \in \mathbb Z_p \text{ for all but finitely many } p\right\}.
	\]
\end{defn}
Let $\GL_n(\mathbb A_\qq)$ denote the set of $n \times n$ matrices with entries in $\mathbb A_\qq$. Further, because $\qq \hookrightarrow \mathbb A_\qq$ diagonally, we also have 
\[
	\GL_n(\qq) \hookrightarrow \GL_n(\mathbb A_\qq)
\]
which yields a left (and right) action\footnote{In this paper we shall use $G\lacts X$ to denote left action of $G$ on $X$ and $X \racts G$ to denote right action.}:
\[
	\GL_n(\qq) \lacts  \GL_n(\mathbb A_\qq) \racts  \GL_n(\qq).
\]
The left quotient space $\GL_n(\qq) \backslash \GL_n(\mathbb A_\qq)$ is well-defined in this case. Since $\GL_n(\qq)$ still acts by right action on this space, functions of this space form a (left) representation of $\GL_n(\qq)$
\[
	\GL_n(\qq) \lacts \mathrm{Fun}\left( \GL_n(\qq) \backslash \GL_n(\mathbb A_\qq) \right)
\]
This can be decomposed into irreducible representations, which are known as the \textbf{automorphic representations} of $\GL_n(\qq)$.
Though not absolutely precise, this is a good first-order description of what an automorphic representation is. 

\begin{idea}
	The Langlands correspondence associates to each $n$-dimensional representation of the absolute Galois group $\mathrm{Gal}(\bar \qq/\qq)$ an automorphic representation of $\GL_n(\qq)$.
\end{idea}
More than just an equivalence of sets, though, the Langlands correspondence states that a certain set of \emph{eigenvalue data} must agree on both sides. 

From the perspective of the absolute Galois group (henceforth referred to as the \emph{Galois side}), this eigenvalue data is called the \textbf{Frobenius eigenvalues} of this representation. For $p$ an unramified prime, the Frobenius automorphism $x \to x^p$ is the generator of the Galois group of any finite extension $\mathrm{Gal}(\mathbb F_q/\mathbb F_p)$. Given a finite-dimensional representation of $\mathrm{Gal}(\bar \qq/\qq)$ as well as a conjugacy class, one can lift the Frobenius automorphism to a conjugacy class. The eigenvalues (well-defined for a given conjugacy class) of these elements are the Frobenius eigenvalues of that representation. 

From the perspective of the automorphic representations (henceforth referred to as the \emph{automorphic side}), the eigenvalue data is more difficult to describe. It relies on the construction of linear operators on the space of automorphic representations known as \textbf{Hecke Operators}. Though a full description of the Hecke eigendata is beyond the scope of this paper, we can give a rough and ``cartoonish'' picture of the most basic case of Hecke eigenvalues (c.f. \cite{miyake1971, kudla2004} for a deeper exposition). In the $\GL_2$ case, the space of automorphic representations is related to the space of modular forms on the upper half plane corresponding to quotients $\Gamma \backslash \mathbb H$ with $\Gamma$ a special type of discrete subgroup of $\SL_2(\mathbb Z)$. 

When $\Gamma = \SL_2(\mathbb Z)$, a modular form of weight $k$ can be interpretted as a function $f$ on the set of lattices in $\mathbb R^2$ so that $f(a\Lambda) = a^k f(\Lambda)$.
The $m$th Hecke operator is then defined as:
\[
	T_m f (\Lambda) := m^{k-1} \sum_{[\Lambda' : \Lambda] = m} f(\Lambda')
\]
These are pairwise-commuting linear operators, and can thus be simultaneously diagonalizable. 
The modular forms that are eigenvectors for this operator are known as \textbf{Hecke eigenforms}, and their eigenvalue data is what we define as the \textbf{Hecke eigenvalues} of that representation. The story for more general subgroups $\Gamma$ gives an analogous construction but the story becomes much more involved beyond the $\GL_2$ case.


With this bare background laid out, we can make at least a parsable statement of the Langlands conjecture.

\begin{conj}[Langlands]
	To each $n$-dimensional representation of the absolute Galois group, there is a corresponding automorphic representation of $\GL_n(\qq)$ so that the Frobenius eigenvalues of the Galois representation agree with the Hecke eigenvalues of the automorphic representation.
\end{conj}

It is worth mentioning that the Langlands conjecture over $G = \mathrm{GL}_1$ is the same as what is known in number theory as \emph{class field theory} \cite{frenkel2007}. 


Many questions in number theory can be formulated in terms of questions about the nature of the absolute Galois group. On the other hand, automorphic representations can be studied using analytic methods, which would imply that deep number-theoretic data can be made accessible by studying these analytic objects.

The eigenvalue data plays a particularly important role both in the Langlands correspondence and its geometric analogue. The study of this eigenvalue data will become the study of the \emph{geometric Satake} symmetries acting on both sides of the geometric Langlands equivalence, and this thesis will explore how ideas from physics can give a concrete realization of the eigenvalue data in terms of \emph{operator insertions} in quantum field theory \cite{kapustin2006}. 

% section the_langlands_program (end)

\section{Weil's Rosetta Stone and Geometric Langlands} % (fold)
\label{sec:weil_s_rosetta_stone_and_geometric_langlands}

The Langlands correspondence in number theory also has a close analogy for curves defined over finite fields $\mathbb F_q$. Indeed, translating the number theoretic statements of the Langlands program has over the past fifty years led to an extremely fruitful set of developments in the field of \emph{Arithmetic Geometry}. These developments have led to the famous proofs of the \emph{Weil Conjectures} and the \emph{Riemann Hypothesis over Finite Fields}. We will not discuss these developments here but refer the reader to \cite{osserman2008}. 

We \emph{will}, however, illustrate this anology to function fields over $\mathbb F_q$ to motivate the translation of the Langlands program to a more geometric setting. Consider the 1-dimensional affine space $\mathbb A^1 (\mathbb F_q)$. We have $F := \mathbb F_q(t)$ the function field on $\mathbb A^1 (\mathbb F_q)$. This will play the role analogous to the role of $\qq$ before. Before, we could complete $\qq$ at each prime $p$ to get the $p$-adics. For each point $x \in \mathbb A^1 (\mathbb F_q)$, there is a notion of a \emph{completion} for $\mathbb F_q(t)$ at $x$, and also a notion of a \emph{ring of integers} corresponding to the localization $\mathcal O_x$ at $x$.

To understand these completions, we make the following definitions.
\begin{defn}[Formal Power Series]
	Let $\kk[t]$ be a polynomial ring in one variable over a field $\kk$. The \textbf{ring of formal power series} around $x$, $\kk[[t-x]]$, is defined as the ring of all (possibly infinite) series of the form
	\[
		\sum_{n=0}^\infty a_n q^n,
	\]
	where here there is no restriction that only finitely many $a_n$ are nonzero. 
\end{defn}

\begin{defn}[Formal Laurent Series]
	Let $\kk[t]$ be a polynomial ring in one variable over a field $\kk$. The \textbf{ring of formal Laurent series} around $x$, $\kk((t-x))$, is defined as the ring of all (possibly infinite) series of the form
	\[
		\sum_{n=-\infty}^\infty a_n q^n,
	\]
	where here there is no restriction that only finitely many $a_n, n \geq 0$ are nonzero but \emph{only finitely many $a_n, n<0$ can be nonzero}.
\end{defn}

The field $F_x$ corresponding to the completion of $F$ at $x$ can be viewed as the field of Laurent series around $x$, denoted $\mathbb F_q((t-x))$. $\mathcal O_x$ can similarly be viewed in terms of formal power series at $x$, $\mathbb F_q [[t-x]]$. With this machinery in place, we can define the ring of adeles analogously to before.

\begin{defn}[Adele Ring for $\mathbb F_q(t)$]
	The ring of adeles of $\mathbb F_q(t)$ is defined as 
	\[
		\mathbb A_{\mathbb F_q(t)} := \prod_{x \in \mathbb P^1(\mathbb F_q)}^{res} \mathbb F_{q_x}((t-x))
	\]
	and the above product is restricted as before in the sense that all but finitely many terms in this product over $x$ lie in $\mathbb F_q [[t-x]]$. Here the completion at the point at infinity corresponds to $\mathbb F_q ((1/t))$.
\end{defn}
We naturally have that 
\[
	\mathbb O_{\mathbb F_q(t)} := 	\prod_{x \in \mathbb P^1(\mathbb F_q)} \mathbb F_{q_x}[[t-x]]
\]
sits inside $\mathbb A_{\mathbb F_q(z)}$.

All of this can be translated more generally to the function field $F$ for a curve $C$ over $\mathbb F_p$. This would correspond to a number field and its ring of integers in the original Langlands conjecture. Here, ramification of various points on the curve becomes an issue and there is more subtlety in defining many of the above concepts.

Already, for a function field of a curve $C$, the analogue of the Galois group is known to be the \textbf{etale fundamental group}, and a Galois representation would be a representation of $\pi_1^{\text{\'et}}(C,x) \to \GL_n$ in the unramified case. In analytic language for $C$ a complex curve, the \'etale fundamental group becomes the usual $\pi_1$ and a Galois representation becomes a representation of the fundamental group $\pi_1(C) \to \GL_n$. 

In the unramified case, automorphic representations correspond exactly the the $\GL_n(\mathbb O_F)$-invariant functions on $\GL_n (\mathbb F)\backslash \GL_n(\mathbb A_F)$. This means that the space of automorphic representations corresponds to:
\[
	\mathrm{Fun} \left(\GL_n (\mathbb F)\backslash \GL_n(\mathbb A_F)/\GL_n(\mathbb O_F) \right).
\]

It is the following theorem of Weil that will be crucial in making a connection between with the geometric setting over $\cc$.
\begin{theorem}[Weil Uniformization]
	Take $F$ the function field for a curve $C$ over $\fq$. There is a canonical bijection as sets between
	\[
		G (\mathbb F)\backslash G(\mathbb A_F)/G(\mathbb O_F)
	\]
	and the set of $G$-bundles over $C$. Moreover, this in fact extends to an algebraic correspondence between this space and the stack $\mathrm{Bun}_G(C, \fq)$. 
\end{theorem}

$G$-bundles are discussed in Section~\ref{sub:principal_bundles}.
So (in the unramified case), the automorphic side is captured by functions on $\mathrm{Bun}_G(C, \fq)$. 
This set of functions admits an action by the \textbf{spherical Hecke algebra} at every closed point $x \in C$, defined as the space of compactly supported functions on the double coset space:
\[
	\mathcal H_x := \mathrm{Fun}_c (\GL_n (\mathcal O_x)\backslash \GL_n(F_x)/\GL_n(\mathcal O_x))
\]
with multiplication given by an operation known as a \textbf{convolution product} of functions. These algebras correspond to the Hecke operators described earlier. The actions of these algebras at different $x$ commute with one another, just like the Hecke operators in the first column. Consequently, they can be simultaneously diagonalized to give rise to eigenfunctions generalizing the notion of Hecke eigenforms in the number-theoretic $\GL_2$ setting. More correctly, these operators yield eigen-objects called \textbf{Hecke-eigensheaves}\footnote{For an explanation about the transition between functions on this coset space and sheaves, see a reference on the \emph{function-sheaf correspondence}, e.g. \cite{shin2005}}. This thesis will aim to explore the corresponding interpretation of this action in the context of topological field theory in physics.

\begin{table}[t]
	\centering
	\setlength\tabcolsep{3pt} 
\begin{tabularx}{\linewidth}{|>{\hsize=0.8\hsize}Y|>{\hsize=1.2\hsize}Y|Y|}
	\hline
	Number Theory & Curves over $\ff_q$ & Riemann Surfaces\\
	\hline
	\hline
	$\zz \subset \qq$ & $\ff_q[t] \subset \ff_q(t)$ & $\OO^{hol}_\cc \subset \OO^{mer}_\cc$ \\
	\hline
	$\spec \zz$ & $\mathbb A^1_{\ff_q}$ & $\cc$\\
	\hline
	$\spec \zz \cup \{\infty\}$ & $\mathbb P^1_{\ff_q}$ (projective line) & $\cp^1$ (Riemann sphere)\\
	\hline
	$p$ prime number & $x \in \mathbb A^1_{\ff_q}$ & $x \in \cc$\\
	\hline
	\hline
	$\zz_p$ ($p$-adic integers) & $\ff_q[[t-x]]$ power series around $x$ & $\cc[[z-x]]$ holomorphic on formal disk around $x$\\
	\hline
	$\qq_p$ ($p$-adic numbers) & $\ff_q((t-x))$ Laurent series around $x$ & $\cc((z-x))$ holomorphic on punctured formal disk around $x$\\
	\hline
	$\mathbb A_\qq$ (adeles) & $\mathbb A_{\ff_q}$ function field adeles & $\prod_{x \in \cc}^{res} \cc((z-x))$ restricted product of functions on all punctured disks, with all but finitely many extending to the unpunctured disk\\
	\hline
	\hline
	$F/\qq$ (number fields) & $F/\ff_q(t)$ or $\ff_q(C)/\ff_q(\mathbb P^1)$ & $C \to \cp^1$ (branched covers)\\
	\hline
	$\mathrm{Gal}(\overline{F}/F)$ & $\mathrm{Gal}(\overline{F}/F) = \pi_1^{\text{\'et}}(\spec F, \spec \bar F)$ & \\
	\hline
	& $\twoheadrightarrow \mathrm{Gal}(F^{\text{unr}}/F) = \pi_1^{\text{\'et}}(C, x)$ & $\pi_1(C, x)$\\ 
	\hline
	\hline
\end{tabularx}
\caption{Weil's \emph{Rosetta stone}}
\label{tab:rosetta}
\end{table}

Table~\ref{tab:rosetta}, based off of \cite{Yoo18} and \cite{nlab:function_field_analogy}, captures the analogy described above. This is the \emph{function field analogy}, otherwise known as Weil's \emph{Rosetta stone}. 

It is the hope and goal of this correspondence that the extremely difficult number-theoretic Langlands program might become more accessible when phrased in the language of the second or third columns of Table~\ref{tab:rosetta}. A reason to believe this might be so is because the power of modern algebraic geometry, as developed by Grothendieck, Serre, Deligne, and others, becomes a prominent force in driving our understanding of columns two and three. 

The analogy between columns one and two is especially strong, and in many cases a statement about the second column can be exactly translated over into a statement about the first. % The Langlands program for the second column would relate an $n$-dimensional representation of $\Gal(\overline F/F)$ to some special ``automorphic'' subrepresentation of $\mathrm{Fun}(\GL_n (\mathbb F)\backslash \GL_n(\mathbb A_F))$ with some associated eigenvalue data on both sides corresponding appropriately.


We are now in a place where we can attempt to discuss and motivate the third column: the geometric Langlands correspondence over $\cc$. To do this, we will begin with motivation from a different direction, namely Fourier analysis.


% section weil_s_rosetta_stone_and_geometric_langlands (end)

\section{The Fourier Transform and Pontryagin Duality} % (fold)
\label{sec:the_fourier_transform_and_pontryagin_duality}
		
In this section, we will attempt to give an alternative motivation for the geometric Langlands program as a generalized non-abelian analogue of the Fourier transform. 

First let us begin by working with a locally-compact abelian group $G$. Recall that these possess a unique (normalized) Haar measure. We make the following definition:
\begin{defn}[Unitary Character]
	For $G$ locally-compact and abelian, a \textbf{unitary character} of $G$ is a group homomorphism $\chi: G \to U(1)$.
\end{defn}
Using this definition, we define the following group, which plays a role as a \emph{dual} to $G$. It is called the \textbf{Pontryagin dual}.
\begin{defn}
	The set of all unitary characters $\chi$ together with multiplication  $\chi_1 \cdot \chi_2 \in \mathrm{Hom}(G, U(1))$ given by pointwise multiplication of characters, form an abelian group, denoted by $\widehat{G}.$
\end{defn}

\begin{eg}
	We have the following examples:
	\begin{enumerate}
		\item Let $G = S^1$, then the space of unitary characters consists precisely of these of the form $e^{inx}: G \to U(1)$. This makes $\widehat G = \mathbb Z$.
		\item Let $G = \mathbb Z$, then $\chi(1)$ determines the representation uniquely, and so $\widehat G = U(1)$.
		\item 			Let $G = \mathbb R$, then $e^{ikx} : \mathbb R \to U(1)$ is free to have $k$ vary over $\mathbb R$ so $\widehat G = \mathbb R$.
	\end{enumerate}

\end{eg}

Notice in all these cases that $\widehat{\widehat G} \cong G$. This is in fact true more general, and we have the following theorem:
\begin{theorem}[Pontryagin Duality]
For any locally-compact abelian topological group $G$, the canonical map 
\begin{equation*}
\begin{split}
G &\to \widehat{\widehat G}\\
g &\mapsto [\chi \mapsto \chi (g)]
\end{split}
\end{equation*}
is an isomorphism.
\end{theorem}

\begin{obs}
	The space of functions\footnote{By this, we don't mean $L^2(G)$. $\mathrm{Fun}(G)$ can be taken to mean the space of \emph{tempered distributions} on $G$, defined as the continuous linear dual of the Schwartz space of functions. See \cite{arthur1989}.} on $G$, $\mathrm{Fun}(G)$ has a basis given by characters. 
\end{obs}
\begin{eg}
	We have the following examples:
	\begin{enumerate}
		\item $f: S^1 \to \mathbb C$ has $f(\theta) = \sum_{n} a_n e^{i n \theta}$. This is known as the \textbf{Fourier series}.
		\item $f: \mathbb Z \to \mathbb C$ has $f(n) = \int_{0}^{2\pi} F(\theta) e^{i n \theta}$. This is known as the \textbf{discrete time series}.
		\item $f: \mathbb R \to \mathbb R$ has $f(x) = \int_{-\infty}^\infty \widehat{f(k)} e^{ikx}$. This is known as the \textbf{Fourier transform}.
	\end{enumerate}
\end{eg}

Let us now try to generalize the ideas of the Fourier transform to a more direct case. It is useful to view the Fourier transform as letting us see two different sides of the same object. Let that object be the direct product of the group $G$ and $\hat G$. 
The reason this space is worth considering is by noting that there is a unique function on this space, which we can call the \textbf{kernel} $K: G \times \hat G \to \mathbb C$ defined by $K(g, \chi) = \chi (g)$. In the case of  $G=\mathbb R$, this function is exactly $e^{i k x}, x \in \mathbb R, k \in \widehat{ \mathbb R} = \mathbb R$, that is viewed as a function on \emph{both} time and frequency space.

This space comes with two obvious projections.
\[
\begin{tikzcd}
  & G \times \hat G \arrow[ld,"\pi_G"] \arrow[rd,swap,"\pi_{\hat G}"]&\\
G & & \hat G
\end{tikzcd}
\]
Any function $f$ on $G$ can be ``pulled back'' to a function on $G \times \hat G$, namely by ignoring the second component $f'(g, \hat g) = f(g)$. We will denote this pulled back function by $\pi_{G}^* f = f \circ \pi_G$.

Further, a suitable distribution on $G \times \hat G$ can be ``pushed forward'' to either $G$ or $\hat G$ by integrating it over $\hat G$ or $G$ respectively. We will denote these by $(\pi_G)_*$ and $(\pi_{\hat G})_*$, again respectively.

Now if $\hat f$ is a distribution on $\hat G$, we get that $\pi_{\hat G}^* \hat f$ is a distribution on $G \times \hat G$. This can be pushed forward to a function on $G$ by integrating over the $\hat G$ coordinates, but because $\pi_{\hat G}^* \hat f$ is constant on the $G$-coordinate, this function will just be a constant independent of $G$.

On the other hand, if we look at:
\begin{equation}
	f (g) := {(\pi_{G})}_* ([{\pi_{\hat G}}^* \hat f] K) = \int_{\chi \in \hat G} [(\hat f \circ \pi_{\hat G}) (g, \chi)] K(g, \chi)\, d\chi
	\label{eq:fourier}
\end{equation}
we obtain exactly the Fourier transform. For $G = \mathbb R$ this gives us:
\begin{equation}
	f(x) = \int_{\mathbb R} \widehat{f(k)} e^{ikx} dk.
\end{equation}

The reason that the Fourier transform finds so much use in practice is that it serves as an eigendecomposition for the derivative operator. More broadly, on $\mathbb R^n$, the eigenfunctions are plane waves $e^{i\vec k \cdot \vec x}$, which yield eigenvalues both under $\partial_x$ and also under the translation operator more generally $\vec x \mapsto \vec x + \vec y$. Any abelian group acts on itself by translation\footnote{Note that right and left action coincide for an abelian group.}. Consequently, it acts on the functions living on it, $\mathrm{Fun}(G)$, by translation $f(x) \to f(x - y)$. Note however that the unitary characters satisfy:
\[
	y \cdot \chi(x) = \chi(x - y) = \chi(-y) \chi(x)
\]
so that the characters \emph{diagonalize} the translation operator as an eigenbasis, exactly as $e^{ikx}$ did on the real line.

\begin{fact}
	The Fourier transform diagonalizes the action of $G$ on the space of functions $L^2(G) \cong L^2(\hat G)$.
\end{fact}

We have just treated Fourier analysis successfully for the category of locally-compact abelian groups. A natural next question is:
\begin{ques}
	How could we build upon the ideas Fourier analysis to generalize to non-abelian groups? That is, what could be the non-abelian analogue of the Fourier transform?
\end{ques}
Already, one can see that the naive ideas from before will not hold up as well. For one, translation operators no longer commute, and hence cannot be simultaneously diagonalizable with an eigenbasis of unitary characters. As we move to explore the continuous non-abelian setting, the Pontryagin dual group $\hat G$ will be replaced by the Langlands dual group $^L G$.
% , and of course Pontryagin duality will become a very special case of Langlands duality.

It will turn out that to understand the Fourier transform in the non-abelian case, we will have to appeal to \emph{categorification}, which in recent years have proved crucial in many fruitful applications.

% section the_fourier_transform_and_pontryagin_duality (end)

\section{Categorical Harmonic Analysis and Geometric Langlands} % (fold)
\label{sec:categorical_harmonic_analysis_and_geometric_langlands}

As a motivating example of both the algebraic perspective the idea of categorification mentioned in the previous chapter, we will illustrate the \textbf{Fourier-Mukai} transform. We will assume basic familiarity with the language of line bundles. 

When viewing $G$ as a topological category: a topological space equipped with Haar measure, we consider the space of functions on $G$ $\mathrm{Fun}(G)$. For an algebraic category $A$, the study of functions on $A$ is often replaced by instead studying \emph{line bundles}, \emph{vector bundles}, or more generally \emph{(quasi-coherent) sheaves} on $A$.

Let $A$ be an algebraic variety, namely a complex torus of the form $A = \cc^g/\Lambda$ such that $A$ is also a projective variety. $A$ is called abelian because it is endowed with the group structure of this torus. We thus have a multiplication operation (along with the two projections):
\[
	\begin{tikzcd}
		& A \times A \arrow[ld,swap,"\pi_1"] \arrow[d,"\mu"] \arrow[rd,"\pi_2"]&\\
		A & A & A
	\end{tikzcd}
\]
Just like functions, line bundles can be pulled back along map between varieties. Given a line bundle $\mathcal L$ on $A$, $\mu^* \mathcal L, \pi^*_1 \mathcal L, \pi_2^* \mathcal L$ all give line bundles on $A \times A$.

A \textbf{geometric character} is a line bundle $\mathcal L$ on $A$ such that $\mu^* \mathcal L = \pi_1^* \mathcal L \otimes \pi_2^* \mathcal L$.
For geometric characters, there is a canonical isomorphism between $\mathcal L_{x+y}$ and $\mathcal L_x \otimes \mathcal L_y$ given by restricting $\mu^* \mathcal L$ to $(x,y) \in A \times A$ and noting that by definition, this must equal $\mathcal L_x \otimes \mathcal L_y$.

Further, multiplication by an element $x$ gives a map $\mu_x: A \to A$ which is the same as restricting $\mu$ to $\{x\} \times A$. Consequently, for a geometric character
\[
	\mu_x^* \mathcal L = \mathcal L_x \otimes \mathcal L.
\]
That is, the group action acts on geometric characters by tensoring each fiber with the 1D vector space of $\mathcal L$ at $x$, $\mathcal L_x$. Equivalently, it acts on the line bundle by tensoring it with the trivial line bundle with fiber canonically isomorphic to $\mathcal L_x$. Note the similarity between this property of \emph{geometric characters} and the property of ordinary \emph{characters} from before, namely $e^{ik(x+y)} = e^{ikx} e^{iky}$.

It turns out that the set of geometric characters together with the commutative operation $\otimes$ themselves form an abelian variety known as the \textbf{dual abelian variety} to $A$. This is denoted by
\[
	A^\vee := (\{\text{geometric characters}\}, \otimes)
\]
From our birds-eye view of what is going on, it looks like $A^\vee$ is playing an analogous role to $\hat G$ of the previous chapter. We have as before the simple diagram
\[
	\begin{tikzcd}
		& A \times A^\vee \arrow[ld,"\pi_1"] \arrow[rd,"\pi_2"]&\\
		A & & A^\vee
	\end{tikzcd}
\]
Just as on $G \times \hat G$ there was a universal function $K$ called the kernel from which the Fourier transform was defined, on $A \times A^\vee$ there is a \emph{universal bundle} known as the \textbf{Poincare line bundle} $\mathcal P$ so that:
\[
	\mathcal P_{(x, \mathcal L)} = \mathcal L_x.
\]
Note that a geometric character $\mathcal L$ on $A$ would not correspond to a line bundle on $A^\vee$ but instead to an object with a single fiber at $\mathcal L \in A^\vee$ that is zero at all other points. In more precise language, this would be the \emph{structure sheaf} of $\mathcal L$ on $A^\vee$. Indeed, the natural objects to consider in place of \emph{functions/distributions on $G, \hat G$} are not line bundles on $A, A^\vee$ but rather objects known as \emph{quasi-coherent sheaves} on these spaces. For a reference about these objects, see \cite{hartshorne1977}.
\begin{concept}[Fourier-Mukai Transform]
	The Fourier-Mukai Transform is a map between the categories of quasi-coherent sheaves:
	\[
		\mathcal{FM}: \mathcal{QC}(A^\vee) \to \mathcal{QC}(A).
	\]
	In terms of the language above, it is given by:
	\[
		\mathcal F \mapsto (\pi_1)_* ([\pi_2^* \mathcal F] \otimes \mathcal P) .
	\]
	Note the similarity between this and the ``classical'' or ``decategorified'' Equation~\eqref{eq:fourier}. 
	In particular the skyscraper sheaf of $\mathcal L$ in $A^\vee$, denoted $\mathcal O_{\mathcal L}$, is mapped to 
	\[
		(\pi_1)_* ([\pi_2^* \mathcal O_{\mathcal L}] \otimes \mathcal P) = \mathcal L.
	\]
\end{concept}

The correspondences of this categorification are given in Table~\ref{tab:fourier_mukai}. Note in particular how scalars become vector spaces in this categorification, and how vector spaces become categories. 
\begin{table}
	\centering
	\begin{tabular}{c c  c}
		number & $\to$ & line (vector space in general)\\
		functions on $G$ & $\to$ & line bundles on $A$\\
		\emph{vector space} of functions/distributions & $\to$ & \emph{category} of quasi-coherent sheaves\\
		translations $g: G \to G$ & $\to$ & translations $\mu_x: A \to A$\\
		$\{e^{ikx}\}_{k \in \hat G}$ eigenbasis for translations & $\to$ & $
		\{\mathcal L\}_{\mathcal L \in A^\vee}$ eigenbasis for translations\\
		eigenvector multiplied by a number & $\to$ & eigen-bundle tensored with a line bundle\\
		$e^{ik(x+y)} = e^{ikx} e^{iky}$ & $\to$ & $\mathcal L_{x+y} \cong \mathcal L_x \otimes \mathcal L_y$\\
		delta function & $\to$ & skyscraper sheaf\\
		$\{e^{ikx}\}$ on $G$ is a delta function on $\hat G$ & $\to$ & $\mathcal L$ on $G$ is a structure sheaf on $A^\vee$
	\end{tabular}
	\caption{The categorification associated to the Fourier-Mukai transform}
	\label{tab:fourier_mukai}
\end{table}
 
Everything so far discussed has been about abelian groups, though we have managed to get a much deeper language by using the algebraic picture. This will at least give us some motivation to give a statement of the geometric Langlands conjecture. In the Langlands program, we have $G$ a reductive algebraic group. 

Our discussion of the Fourier-Mukai transform would naively lead us to formulate some sort of duality transformation taking us from quasi-coherent sheaves on $G$ to quasi-coherent sheaves on some dual group $\check G$. Because the group multiplication is not abelian, the above categorification will not make sense. The correct generalization is more subtle, and the principal geometric objects of study are not $G$ and $\check G$.

 
\begin{table}[h!]
	\centering
\begin{tabularx}{\textwidth}{|>{\hsize=0.8\hsize}Y|Y|>{\hsize=1.2\hsize}Y|}
	\hline
	& Abelian (classical) & Non-abelian (categorified)\\
	\hline
	Space of ``functions'' & $\mathrm{Fun}(G) \cong \mathrm{Fun}(\hat G)$ & $\mathcal D (\mathrm{Bun}_G) \cong \, \mathcal{QC}(\mathrm{Flat}_{\check G})$\\
	Symmetries acting & $G \lacts \mathrm{Fun}(G)$ & $\mathrm{Sat}_G \lacts \mathcal D (\mathrm{Bun}_G), \mathcal{QC}(\mathrm{Flat}_{\check G})$\\
	Eigenbasis & $\{e^{ikx}\}_{t \in \hat G}$ & Hecke Eigensheaves\\
	\hline
	\end{tabularx}
\caption{The analogy of the Fourier-Mukai transform as an abelian case of the geometric Langlands correspondence}
\label{tab:geometric_langlands}
\end{table}


Taking a hint from the last section, we recall that the Langlands duality for function fields relates certain categories of sheaves on spaces associated to $G$ and $\check G$. The automorphic side turned out to correspond to functions on the space of $G$-bundles, by Weil's uniformization theorem. The Galois side concerned itself with representations of the fundamental group of a curve $C$ into $\pi_1(C)$. 


\begin{equation}
	\mathcal D (\mathrm{Bun_G}(C)) \cong \mathcal{QC} (\mathrm{Flat}_{\check G} (C))
\end{equation}
where Satake symmetries act naturally on both sides. This is supposed to be a nonabelian analogue of the Fourier-Mukai transform, so in particular it should take a skyscraper sheaves on the right (i.e. flat $\check G$-connections on $C$) to a class of $D$ modules known as \emph{Hecke eigensheaves} on the left. 

The original conjecture was formulated by Beilinson and Drinfeld in \cite{beilinson1991}. This conjecture is true when $G$ is abelian 
In fact, this conjecture was shown to be false by V. Lafforgue \cite{lafforgue2009}.
A refined version of this conjecture is given by Arinkin and Gaitsgory in \cite{arinkin2015}, involving a refinement of the quasi-coherent sheaves on the Galois side to objects known as \emph{ind-coherent} sheaves with a certain support condition. 
\begin{equation}
	\mathcal D (\mathrm{Bun_G}(C)) \cong \mathcal{IC}_{N} (\mathrm{Flat}_{\check G} (C))
\end{equation}
Though this may seem more complicated, there is reason to believe that these objects can be derived as the right ones to consider on the basis of physical grounds, c.f. \cite{elliott2017}. 

\begin{table}[h!]
	\centering
\begin{tabularx}{\textwidth}{|>{\hsize=0.8\hsize}Y|>{\hsize=1.2\hsize}Y|Y|}
	\hline
	Classical Picture & Geometric Langlands & Topologically twisted $\mathcal N=4$ theory\\
	\hline
	Space of ``functions'' & $\mathcal D (\mathrm{Bun}_G) \cong \, \mathcal{QC}(\mathrm{Flat}_{\check G})$ & \emph{Category} of boundary conditions \\
	Symmetries acting & $\mathrm{Sat}_G \lacts \mathcal D (\mathrm{Bun}_G), \mathcal{QC}(\mathrm{Flat}_{\check G})$ & Insertions of Wilson and `t Hooft line defects\\
	Eigenbasis & Hecke Eigensheaves & Electric/Magnetic Eigenbranes\\
	\hline
	\end{tabularx}
\caption{The connection between the ideas in geometric Langlands and supersymmetric field theory, to be discussed in this thesis.}
\label{tab:langlands_and_physics}
\end{table}


Although a full discussion of the concepts that appear in Table~\ref{tab:geometric_langlands} is beyond the scope of this thesis, we can at least give the reader one ``final column'', yielding Table~\ref{tab:langlands_and_physics}. This column is intended to highlight some key points in the relationship between the concepts of geometric Langlands and physics.


The action of Wilson loops on the Galois side can be very easily understood using the language of holonomy and flat connections, both of which are explained in Section~\ref{sec:connections_on_principal_bundles}. On the other hand, the action of the `t Hooft operators is much more subtle and involved. To be able to fully appreciate this, we must understand the nature of these so-called ``disorder operators'' by first understanding the well-known picture of instantons on $\mathbb R^4$ and then restricting this to an understanding of monopoles on $\mathbb R^3$. Finally, we work in the spirit of Edward Witten's paper \cite{witten2010nahm} we will make use of our understanding of monopoles and use this to understand the action of line defect operators on boundary conditions in the topological $\mathcal N=4$ theory. 

% section categorical_harmonic_analysis_and_geometric_langlands (end)

\chapter{Mathematical Background\label{ch:math}}

	This chapter aims to give a thorough introduction to the mathematical ideas necessary to cast the geometric Langlands program and gauge theory in a coherent framework. 
	
	\section{Elementary Representation Theory} % (fold)
	\label{sec:elementary_representation_theory}
		We begin with a study of elementary representation theory of a group. The representation theory of finite (or more generally locally-compact) abelian groups is a straightforward generalization of the ideas of the Fourier transform. 
		For someone with a background in engineering, representation theory itself is one tremendous generalization of Fourier analysis. 
		Through this naive lens, many of the dualities explored in this paper are great generalizations of the duality between the domains of time and frequency, position and momentum, shape and spectrum, etc. 

		We assume the reader is familiar with elementary group theory. A \textbf{group} is 
		\begin{defn}[Group]
			A group is a set $G$ together with an operation $\cdot$ such that
			\begin{itemize}
				\item If $a, b \in G$ then $a\cdot b \in G$
				\item The group operation is associative so that $a \cdot (b \cdot c) = (a \cdot b) \cdot c$
				\item There is an identity element (necessarily unique), denoted by $1 \in G$ so that $1 \cdot g = g \cdot 1 = g$ for all elements $g$.
				\item For each element $g$, there exists a (necessarily unique) element, denoted by $g^{-1}$ such that $g \cdot g^{-1} = g^{-1} \cdot g = 1$ 
			\end{itemize}
		\end{defn}
		we will usually drop the notation for $\cdot$ and multiply the symbols denoting group elements directly. When a group is supposed to be viewed additively, we will use the symbol `$+$' instead of `$\cdot$'. A \textbf{subgroup} is a subset $H \subseteq G$ that also forms a group under this operation.
		
		A map $\phi$ between two groups $G, H$ is called a \textbf{group homomorphism} if it commutes with group action, namely if $\phi(g \cdot h) = \phi(g) \cdot \phi(h)$.
		
		Groups are meant to represent symmetries of various objects. A group $G$ is said to \textbf{act} on a space $X$ (alternatively we say that $X$ has a \textbf{group action} by $G$) if either the following are satisfied
		\begin{defn}[Left and Right Group Actions]
			A \textbf{left group action} by $G$ on a space $X$ is a map $\phi: G \times X \to X$ so that the following hold (we write $\phi(g, x)$ as $g\cdot x$):
			\begin{itemize}
				\item $e \cdot x = x$ for all $x \in X$,
				\item $(gh)\cdot x = g\cdot(h \cdot x)$. 
			\end{itemize}
			Often, we will denote right action by $G \lacts X$.
			
			A right group action by $G$ is a map $\phi: X \times G \to X$ so that the following hold (using analogous notation)
			\begin{itemize}
				\item $x \cdot e = x$ for all $x \in X$,
				\item $x \cdot gh = (x \cdot g) \cdot h$.
			\end{itemize}
			Often, we will denote right action by $X \racts G$
		\end{defn}
		The difference between these two definitions is the order in which $g$ and $h$ act. In the left action case, $h$ acts first, while in the right action case, $g$ acts first. Different actions will be more natural to consider depending on context.
		
		\begin{eg}
			Consider an equilateral triangle in the plane. Its symmetries are rotations by $60\degree$ together with flips around any one of its central axes. This forms the dihedral group $D_6$ of rotations of the $3$-gon. The general symmetry group of the $n$-gon contains $2n$ symmetry elements and is denoted by $D_{2n}$.
		\end{eg}
		\textbf{TODO: Add figure}

		When looking at group action, there are several important concepts associated to it. Given a point $x \in X$, the set of points that it can be mapped to under the action of $G$ is called the \textbf{orbit} of $X$. The set of group elements that act trivially on it (so that $g \cdot x = x$) is called the \textbf{stabilizer} of $x$. The set of points stabilized by the entire group is called the set of \textbf{fixed points} or $G$\textbf{-invariants} and is denoted $X^G$. The set of all orbits of $X$ under the action of $G$ is denoted by $X/G$ in the case of right action and $G\backslash X$ in the case of left action. This is also sometimes known as the space of $G$\textbf{-coinvariants} and denoted by $X_G$.
		
		In particular, every group acts on itself (by left/right action) and has its subgroups acting on it. When a subgroup $H$ acts on on $G$, the resulting space of orbits of equivalence classes $g H$ also forms a group when the subgroup satisfies $g H g^{-1} = H$ for all $g \in G$. Such a subgroup is called \textbf{normal} and the resulting group of orbits is known as the \textbf{group quotient}, denoted $G/H$. One can straightforwardly show that this group is the same regardless of left or right action.
		
		We aim to study groups via their actions on other objects. One of the most powerful ways to do this turns out to be by studying their actions on \textbf{vector spaces}, especially over $\mathbb C$. We assume the reader is familiar with the notion of a vector space.
		
		For any two given vector spaces, $V$ and $W$, the set of all linear transformations $T: V \to W$ is denoted by $\mathrm{Hom}(V, W)$. The set of all linear transformations $T: V \to V$ is denoted by $\mathrm{End}(V)$ (also known as endomorphisms of the vector space). Among these transformations, the invertible ones are denoted by $\mathrm{GL}(V)$, otherwise known as the \textbf{general linear group} of the space $V$. When $V$ takes the form $\mathbf k^n$ for some field $\mathbf k$ and positive integer $n$, this is denoted $\mathrm{GL}_n(\mathbf k)$. 

		 A vector space $V$ together with a group homomorphism $\rho: G \to \mathrm{GL}(V)$ is called a \textbf{representation} of $G$. Equivalently, a representation is a vector space with an action of $G$ by linear transformations. A given representation can have a \textbf{subrepresentation}, namely a subspace $V' \subseteq V$ that is preserved by the action of $G$. Every representation has itself and the $\mathbf 0$ vector as subrepresentations. These two are called the \textbf{trivial} subrepresentations. A representation with no nontrivial subrepresentations is called an \textbf{irreducible representation}. In general, we will denote the representation by $V$ alone, even though we really mean this to represent the data $(V, \rho_V)$.
		 
		A morphism between representations $V$ and $W$ is a linear map that commutes with $G$ action: $\phi(\rho_1(g)  v) = \rho_2 (g) \phi (v)$. Such maps will be called $G$\textbf{-linear}, and the set of such maps will be denoted as $\mathrm{Hom}_G(V, W)$.
		 
		We have the following lemma of central importance in representation theory: 
		\begin{lemma}[Schur's Lemma]
			If $V$ and $W$ are two different irreducible representations of a group $G$ over $\mathbb C$, and $\phi$ is a linear transformation from $V \to W$, then $\phi$ is either trivial (the zero map) or invertible.
			
			Further, if $V=W$ and $\rho_V=\rho_W$, then $\phi$ is either trivial or a scalar multiple of the identity. 
		\end{lemma}
		\begin{proof}
			This follows from noting that both the kernel and image of $\phi$ are subrepresentations, so must be either $\mathbf 0$ or the whole space. 
			
			The second part follows from the fact that every linear transformation over $\mathbb C$ has at least one eigenvalue $\lambda$, then one can see that $\phi - \lambda$ is also a $G$-linear transformation with a nontrivial kernel, so must be zero by the preceding.
		\end{proof}
		% The reader should be familiar with the fact that, in general, a vector space need not have an inner product, but there is a natural choice of inner product

		The representation theory of finite groups over $\mathbb C$ is particularly elegant. By \textbf{Mashke's Theorem}

		% section elementary_representation_theory (end)
		
		\section{The Fourier Transform and Pontryagin Duality} % (fold)
		\label{sec:the_fourier_transform_and_pontryagin_duality}
		
		In this chapter, we will be working with a locally-compact (to be defined in the next part) abelian group $G$. We make the following definition:
		\begin{defn}[Unitary Character]
			For $G$ locally-compact and abelian, a \textbf{unitary character} of $G$ is a group homomorphism $\chi: G \to U(1)$.
		\end{defn}
		From this definition, we define the following group, which plays a role as a \emph{dual} to $G$. It is called the \textbf{Pontryagin dual}.
		\begin{defn}
			The set of all unitary characters $\chi$ together with multiplication given by $\chi_1 \cdot \chi_2 \in \mathrm{Hom}(G, U(1))$.
		\end{defn}
		
		\begin{eg}
			We have the following examples:
			\begin{enumerate}
				\item Let $G = S^1$, then the space of unitary characters is precisely of the form $e^{inx}: G \to U(1)$. This makes $\widehat G = \mathbb Z$.
				\item Let $G = \mathbb Z$, then $\chi(1)$ determines the representation uniquely, and so $\widehat G = U(1)$.
				\item 			Let $G = \mathbb R$, then $e^{ikx} : \mathbb R \to U(1)$ is free to have $k$ vary over $\mathbb R$ so $\widehat G = \mathbb R$.
			\end{enumerate}
			
		\end{eg}

		Notice in all these cases that $\widehat{\widehat G} \cong G$. This is in fact true more general, and we have the following theorem:
		\begin{theorem}[Pontryagin Duality]
			$G \to \widehat{\widehat G}$ is an isomorphism of groups, given by sending $g \to  g'$ which is given by $g' (\chi) = \chi (g)$.
		\end{theorem}
		
		\begin{obs}
			The $L^2$-integrable functions on $G$ have a basis given by characters. 
		\end{obs}
		\begin{eg}
			We have the following examples:
			\begin{enumerate}
				\item $f: S^1 \to \mathbb C$ has $f(\theta) = \sum_{n} a_n e^{i n \theta}$. This is known as the \textbf{Fourier series}.
				\item $f: \mathbb Z \to \mathbb C$ has $f(n) = \int_{0}^{2\pi} F(\theta) e^{i n \theta}$. This is known as the \textbf{discrete time series}.
				\item $f: \mathbb R \to \mathbb R$ has $f(x) = \int_{-\infty}^\infty \widehat{f(k)} e^{ikx}$. This is known as the \textbf{Fourier Transform}.
			\end{enumerate}
		\end{eg}
		
		Let us now try to generalize the ideas of the Fourier transform to a more direct case. It is useful to view the Fourier transform as letting us see two different sides of the same object. Let that object be the direct product of the group $G$ and $\hat G$. 
		The reason this space is worth considering is by noting that there is a unique function on this space, which we can call the \textbf{Kernel} $K: G \times \hat G \to \mathbb C$ defined by $K(g, \chi) = \chi (g)$. In the case of  $G=\mathbb R$, this function is exactly $e^{i k x}, x \in \mathbb R, k \in \widehat{ \mathbb R} = \mathbb R$, that is viewed as a function on \emph{both} time and frequency space.
		
		This space is also endowed with two obvious projections (namely to either factor of the product).		
		\[
		\begin{tikzcd}
		  & G \times \hat G \arrow[ld,"\pi_G"] \arrow[rd,swap,"\pi_{\hat G}"]&\\
		G & & \hat G
		\end{tikzcd}
		\]
		Any function $f$ on $G$ can be ``pulled back'' to a function on $G \times \hat G$, namely by ignoring the second component $f'(g, \hat g) = f(g)$. We will denote this pulled back function by $\pi_{G}^* f = f \circ \pi_G$.
		
		Further, a suitable distribution on $G \times \hat G$ can be ``pushed forward'' to either $G$ or $\hat G$ by integrating it over $\hat G$ or $G$ respectively. We will denote these by $(\pi_G)_*$ and $(\pi_{\hat G})_*$, again respectively.
		
		Now if $\hat f$ is a distribution on $\hat G$, we get that $\pi_{\hat G}^* \hat f$ is a distribution on $G \times \hat G$. This can be pushed forward to a function on $G$ by integrating over the $\hat G$ coordiantes, but because $\pi_{\hat G}^* \hat f$ is constant on the $G$-coordinate, this function will just be a constant independent of $G$.
		
		On the other hand, if we look at:
		\begin{equation}
			f (g) := {(\pi_{G})}_* ([{\pi_{\hat G}}^* \hat f] K) = \int_{\chi \in \hat G} [(\hat f \circ \pi_{\hat G}) (g, \chi)] K(g, \chi)\, d\chi
		\end{equation}
		we obtain exactly the Fourier transform. For $G = \mathbb R$ this gives us:
		\begin{equation}
			f(x) = \int_{\mathbb R} \widehat{f(k)} e^{ikx} dk.
		\end{equation}
		
		The reason that the Fourier transform finds so much use in practice is that it serves as an eigendecomposition for the derivative operator. More broadly, on $\mathbb R^n$, the eigenfunctions are plane waves $e^{i\vec k \cdot \vec x}$, which yield eigenvalues both under $\partial_x$ and also under the translation operator more generally $\vec x \mapsto \vec x + \vec y$. Any abelian group acts on itself by translation\footnote{Keep in mind that right and left action coincide for an abelian group.}. Consequently, it acts on the functions living on it, $\mathrm{Fun}(G)$, by translation $f(x) \to f(x - y)$. Note however that the unitary characters satisfy:
		\begin{equation}
			y \cdot \chi(x) = \chi(x - y) = \chi(-y) \chi(x)
		\end{equation}
		so that the characters \emph{diagonalize} the translation operator as an eigenbasis, exactly as $e^{ikx}$ did on the real line.
		
		We have just treated Fourier analysis successfully for the category of locally-compact abelian groups. The natural next question is of course
		\begin{ques}
			How does Fourier analysis look like for more general groups? That is, what is the non-abelian analogue of the Fourier transform?
		\end{ques}
		It is this stream of thought that will take us deep into the Langlands program, and ultimately into the heart of emerging ideas in theoretical physics. Already, one can see that the naive ideas from before will not hold up as well. For one, translation operators no longer commute, and so cannot be simultaneously diagonalizable with an eigenbasis of unitary characters. As we move to explore the non-abelian setting, the Pontryagin dual group $\hat G$ will be replaced by the Langlands dual group $^L G$, and of course Pontryagin duality will become a very special case of Langlands duality.
		
		It will turn out that to understand the Fourier transform in the non-abelian case, we will have to appeal to \emph{categorification}, one of the deepest aspects of twenty-first century mathematics.
		
		
		
	% section the_fourier_transform_and_pontryagin_duality (end)
	
	\section{Elementary Topology} % (fold)
	\label{sec:elementary_topology}
	
	Before we can begin a more in-depth study of symmetry and the spaces on which it acts, it is worth understanding how to study spaces in a general setting. Firstly:
	\begin{ques}
		What do we mean when we use the word ``space''?
	\end{ques}
	In this thesis alone, this word will have many meanings. The first definition we give is the starting point of topology.
	\begin{defn}[Topological Space]
		A set $X$ together with a family $\mathcal F$ of subsets of $X$ known as the \textbf{open sets} forms a \textbf{topological space} if the following conditions hold
		\begin{enumerate}
			\item The empty set $\emptyset$ and $X$ both belong to $\mathcal F$,
			\item Any union of open sets is open\footnote{It is important to understand that this can include \emph{infinite unions}},
			\item Any \emph{finite} intersection of open sets is open.
			
		\end{enumerate}
	\end{defn}
	The complement of an open set is what defines a \textbf{closed set}. It is not difficult to show that the dual properties of the above hold for closed sets, namely
	\begin{enumerate}
		\item The empty set $\emptyset$ and $X$ are both closed
		\item Any intersection of closed sets is closed
		\item Any \emph{finite} union of closed sets is closed.
	\end{enumerate}
	\begin{eg}
		The first example that we are given in high school is the real number line. The open sets are exactly the unions of the open intervals of the form $(a, b)$ while the closed sets are exactly the intersections of closed intervals of the form $[a, b]$. Note that an open set has no maximum or minimum point (i.e. the open interval $(0, 1)$ has no greatest number).
	\end{eg}
	\begin{eg}
		The previous example generalizes to $n$-dimensional space $\mathbb R^n$. Here, the open sets are generated by unions of \emph{open balls} of the form $B_a(x) := \{ \vec y : |\vec x - \vec y| < a \}$.
		
		In both of the cases, these topologies arose from the fact that these spaces are equipped with a \textbf{metric} $d(\vec x, \vec y) = |\vec x-\vec y|$ giving a notion of distance. When a metric is given, a topology can always be defined by defining open balls as above, and defining the open sets of the topology as precisely unions of open balls.
	\end{eg}
	\begin{eg}
		If all the points of a given space $X$ are open, then any subset of $X$ is also open, by virtue of being a union of points. This gives a \textbf{trivial topology} on $X$. The other trivial topology is that which consists of only $X$ and $\emptyset$ as open sets.
	\end{eg}
	
	The open sets of a space allow us to define \textbf{neighborhoods} of points. This is what allows topology to give us structure somewhat resembling the familiar structure of our (local) spacetime
	\begin{defn}[Neighborhood]
		A neighborhood of a point $p$ is a set $V$ that contains an open set $U$ containing $p$.
	\end{defn}
	This definition does not require $V$ to be open or closed. An \textbf{open neighborhood} of $p$ is any open set $U$ containing $p$. In general, open sets are supposed to play the roles of ``neighborhoods'' while closed sets are supposed to generalize the role of ``points''.
	
	One of the most important properties of topological spaces is known as \textbf{compactness}. In early undergraduate classes, one may have been taught to think of compactness as being ``closed and bounded''. Indeed, this is the right way to think about it, but because ``bounded'' requires that there is a notion of distance on the space, and doesn't make much sense outside of euclidean spaces, we define compactness more generally as 
	\begin{defn}[Compact]
		A set $X$ in a topological is called compact if every open cover of $X$ contains a finite sub-cover.
	\end{defn}
	It is up to the reader (by appealing to Bolzano-Weierstrass) to confirm that this more difficult definition agrees with the naive one. 
	
	Compact spaces are especially easy to work with when performing integration, as any covering of the space into open neighborhoods can be turned into a finite number of integrations over the (usually finite-volume) neighborhoods themselves, so that one need not worry about integrals diverging. 
	
	Similarly we define
	\begin{defn}[Locally Compact]
		A space $X$ is \textbf{locally compact} if every point $x \in X$ has a compact neighborhood.
	\end{defn}
	
	Most spaces we are familiar with are locally compact. An example of a non-locally compact space is the set of rational numbers under the topology of the reals. Since all nontrivial open sets here all contain infinitely many rationals, each neighborhood must as well, and we can always construct a sequence of rationals that does not converge to a rational in the neighborhood that we are given. 
	
	
	% section elementary_topology (end)
	
	\section{Differential Geometry} % (fold)
	\label{sec:differential_geometry}
	
	The role of differential geometry in modern physics is similar to the role of the beating heart in a human body. It has become an essential ingredient in formulating physical law. Here, we will give an exposition to the ideas of differential geometry in a mathematical setting, with some motivation from physics to guide us along.
	
	\begin{defn}[Manifold]
		
	\end{defn}
	
	\begin{defn}[Tangent Space]
		
	\end{defn}
	
	\begin{defn}[Vector Bundle]
		
	\end{defn}
	
	\begin{defn}[Section]
		
	\end{defn}
	
	\begin{eg}[Trivial Bundle]
		
	\end{eg}
	
	\begin{eg}[Tangent Bundle]
		The first nontrivial example of a vector bundle that one comes across is the tangent bundle of a manifold $M$, denoted by $TM$.
		
		More specifically, the tangent bundle of a sphere \textbf{Todo FINISH}
	\end{eg}
	
	\begin{obs}[Parallel Transport]
		
	\end{obs}
	
	\textbf{Something about the impossibility of transporting vectors... to be attacked when we talk about Gauge theory}
	
	\begin{defn}[Lie Derivative]
		
	\end{defn}
	
	\textbf{Work out the flow along a sphere}
	
	% section differential_geometry (end)
	
	\section{Lie Theory} % (fold)
	\label{sec:lie_theory}
		
		In physics and in mathematics, both (smooth) manifolds and groups play extremely important roles. It makes sense therefore to study their intersection: smooth manifolds that are groups. This is the study of \textbf{Lie groups}.
		\begin{defn}[Lie Group]
			A Lie group is a group $G$ that is also a manifold such that the group operations of multiplication:
			$$\mu: G \times G \to G \qquad (x, y) \to xy$$
			and inversion 
			$$^{-1} : G \to G \qquad x \to x^{-1}$$
			are smooth w.r.t. the manifold topology.
		\end{defn}
		
		Lie groups are ubiquitous in physics. $U(1)$ is a Lie group, as is the group of rotations in three dimensions, $\mathrm{SO}(3)$, or the Poincare group of symmetries in Einstein's theory of special relativity, $\mathrm{SO}(3, 1) \ltimes \mathbb R^4$.
		
		In studying the flows on manifolds, we saw that the Lie derivative allows us to measure how one flow changes along another. For Lie groups, given a vector $v_g \in T_g G$, we can canonically transport it to any other point $h \in G$ by applying the (diffeomorphism) $h g^{-1}$ to $G$ and pushing the vector forward along it to $[h g^{-1}]_* v \in T_{h} G$.
		
		WLOG, take a vector $x \in T_e G$, the tangent space at the identity. This pushforward action gives us a vector field $X \in \Gamma(G, TG)$ given by $X(g) = g_* x$. Thus, every vector defines a vector field on $G$ that is \textbf{left invariant}, meaning that $g_* X_h = X_{gh}$. Indeed we see
		\begin{prop}
			There is a one-to-one correspondence between left-invariant vector fields on $G$ and the tangent space to the identity. 
		\end{prop}
		Moreover, if two vector fields $X$ and $Y$ are left-invariant, then so is their commutator bracket $[X, Y]$. This gives us a new algebraic structure on the tangent space $T_e G$\footnote{Indeed on every tangent space, but WLOG we restrict to the identity.} given by 
		\begin{equation}
			[x, y] := \frac{d}{dt} [\exp(tx), \exp(ty)] |_{t = 0}
		\end{equation}
		this is the \textbf{Lie bracket} or \textbf{commutator} on the tangent space of the Lie group at the identity. We call this vector space the \textbf{Lie algebra}, denoted in fraktur by $\frak g$. It corresponds to the infinitesimal symmetries of the system, and the commutator measures how one symmetry changes along the flow of the other. The example of the previous section is intimately related to the Lie group $\mathrm{SO}(3, \mathbb R)$ and its associated Lie algebra $\frak{so}(3, \mathbb R)$, more familiar to high-schoolers as the ``cross-product''.
		
		\begin{defn}[Ideal]
			
		\end{defn}
		
		\begin{defn}[Semisimple Lie Algebras]
			
		\end{defn}
		
		The representation theory of semisimple Lie algebras is particularly elegant, and in many ways mirrors the representation theory of finite groups, but with a much simpler and more elegant classification structure. 
	% section lie_theory (end)
	
	\section{Algebraic Topology} % (fold)
	\label{sec:algebraic_topology}
	
	Just as we have before studied groups by their actions on spaces, it is also equally fruitful to study topological spaces $X$ by finding groups associated to them. The easiest example of this comes from the theory of the fundamental group $\pi_1 (X)$ and then again from the study of the homology and cohomology of $X$. We will outline these theories briefly here and direct the reader to \cite{hatcher} for a deeper expository text.
	
	% section algebraic_topology (end)
	


	\section{Elementary Algebraic Geometry} % (fold)
	\label{sec:elementary_algebraic_geometry}

	% section elementary_algebraic_geometry (end)
	
	\section{Intermediate Algebraic Geometry} % (fold)
	\label{sec:intermediate_algebraic_geometry}
	
	% section intermediate_algebraic_geometry (end)
\chapter{The Basics of Field Theory\label{ch:phys}}

	This chapter aims to give a background into the physical ideas needed to understand the remainder of this paper
	
	\section{Classical Field Theory} % (fold)
	\label{sec:classical_field_theory}
	
	% section classical_field_theory (end)
	
	
	Here is a mathematical formulation of classical field theory:
	\begin{phys}[Classical Field Theory]
		A classical field theory $\mathcal E$ is a collection of the following data:
		\begin{itemize}
			\item A manifold $M$ known as the \textbf{spacetime} of the theory.
			\item A fiber bundle $E \to M$ (or more generally some set of fiber bundles $E_i \to M$)
			\item A space $\mathcal F$ of sections of $E \to M$ called \textbf{fields} on $M$.
			\item An action $S[\Phi]$ from the space of fields into $\cc$.
		\end{itemize}
		Classical field theory studies solutions to the \textbf{classical equations of motion}
		$$\{\varphi \in \mathcal F \mid \; \delta S(\varphi) = 0 \}.$$
	\end{phys}
	\begin{eg}
		When $X = \mathbb R$, we get a single scalar field $\phi$ (here $\Phi$ is $\phi$). An action for this field theory is often given by:
		$$S[\phi] = \int_M |\partial_\mu \phi|^2.$$
	\end{eg}
	\begin{eg}
		Classical electromagnetism is defined by $X=T^* M$ with an action given by:
		$$S[A] = \int_M F \wedge \star F, \qquad F := \dd A.$$
		Here $F = dA$ is the \emph{curvature form} or \emph{electromagnetic field-strength tensor}.
		
		More generally, Yang-Mills theory (to be more thoroughly defined and discussed in the next section) takes $X=T^* M \otimes \frak g$ and given 
		$$S[A] = \int_M \mathrm{Tr}\left( F \wedge \star F \right), \qquad F := \dd A + A \wedge A.$$
		where the trace is taken over the Lie algebra using the Killing form. 
	\end{eg}
	
	\section{Quantum Field Theory and the Operator-Product Expansion} % (fold)
	\label{sec:quantum_field_theory_and_the_operator_product_expansion}
	
		Though we do not know how to make sense of quantum field theory, the intuitive picture that we have of it is given by the \textbf{Feynman Path Integral}. For a given quantum field theory, there is quantity known as the \textbf{partition function}, defined as:
		\begin{equation}
			\mathcal Z = \int \mathcal D\Phi\, e^{- S[\Phi]}.
		\end{equation}
		This is an integral taken over the space of all fields. The measure on this space is mathematically ill-defined in general. 
		\begin{phys}[Classical Observable]
			A classical observable (which we may refer to just by the term \emph{observable}) is a function from the set of field configurations into $\cc$. The corresponding \textbf{quantum observable} is defined as a path integral of a classical observable over the space of fields. 
			\[
				\left< \OO \right> = \int \mathcal D \, \Phi \OO(\Phi) \, e^{-S[\Phi]}
			\]
			
			In the Hilbert-space language, a quantum observable is an operator-valued distribution on the space of fields. 
		\end{phys}
		The partition function is an observable, as is the \textbf{1-point correlation function} at a point $x_1$:
		$$\left< \Phi(x_1) \right> := \frac{1}{\mathcal Z} \int \mathcal D\Phi \, \Phi(x_1) e^{-S[\Phi]}.$$
		In this example, the path integral over all configurations of $\Phi$ probes $\Phi$ at this single point, giving essentially an expectation value. We can take expectation values of many different operators, e.g. $\phi(x_1), (\partial_\mu \phi)(x_1), \mathbf{1}, \phi(x_1) (\partial_\mu \phi)(x_1)$ on $X$. We denote operators by $\OO$. More generally, we define \textbf{correlation functions} as 
		$$\left< \mathcal O_1 \dots \mathcal O_n \right>_g := \frac{1}{\mathcal Z} \int \mathcal D\Phi \, \mathcal O_1 \dots \mathcal O_n e^{-S[\Phi]}.$$
		\begin{phys}[TQFT]
			If the correlation functions of a given quantum field theory are independent of the metric $g$, then the corresponding theory is called a \textbf{topological quantum field theory} (TQFT) in physics.
		\end{phys}
		As an example, consider the following.
	% \noindent 	In fact metric independence implies diffeomorphism invariance.
		\begin{eg}[Chern Simons Theory]
			It turns out the correlation functions of Chern-Simons theory on a 3-manifold $M$ with $\Phi$ being the field $A: M \to T^* M \otimes \frak g$ and the action given by
			$$S[A] \, \propto\,  \int_{M} \mathrm{Tr}\left(A \wedge dA + \frac23 A \wedge A \wedge A \right)$$
			This is clear because the metric has no role in defining the action.
		\end{eg}

		\begin{phys}[Operator Product Expansion]
			 Within the path integral, a product of two local fields can be replaced by a (possibly infinite) sum over individual fields. Namely, given two operators $\OO_a, \OO_b$ and evaluation points $x_1, x_2$, there is an open neighborhood $U$ around $x_2$ such that
			\begin{equation}
				\OO_a (x_1) \OO_b(x_2) \sim \sum_c C_{ab}^c(x_1-x_2) \OO_c(x_2)
			\end{equation}
			where $f \sim g$ implies that $f - g$ stays nonsingular as $x_1 \to x_2$.
		Here $\OO_c$ are other operators in the quantum field theory, and the $C_{ab}^c$ are analytic functions on $U \backslash \{ x_2 \}$ (that often become singular as $x_1 \to x_2$). 
		\end{phys}
	
		In the 2D case, this yields the (possibly familiar) Laurent series associated with CFT. The structure constants contain valuable information about the QFT that allow onw to view it \emph{algebraically}. In particular, they satisfy \textbf{associativity conditions}. The philosophy of the OPE is as follows: % \textbf{(elaborate Phil's point here)}.
% 		This leads naturally to the next idea
		\begin{idea}
			The OPE coefficients, together with the 1-point correlation functions completely determine the $n$-point correlation functions in a quantum field theory. 
		\end{idea}
	\noindent 	For example, a two-point function is simply given by:
		\begin{equation}
			\left< \OO_a(x_1) \OO_b(x_2) \right> = \sum_c C^c_{ab}(x_1 - x_2) \left< \mathcal O_c (x_2) \right>
		\end{equation}
	
	% section quantum_field_theory_and_the_operator_product_expansion (end)
	
	
	\section{Topological Quantum Field Theory} % (fold)
	\label{sec:topological_quantum_field_theory}
	
	An understanding of topological quantum field theory. 
	
	In categorical language, we say:
	\begin{defn}
		A \textbf{$n$-dimensional topological quantum field theory} is a symmetric monoidal functor:
			\[
				\mathcal Z: \mathrm{Bord_n} \to \mathrm{Vect}_{\kk}
			\]
	\end{defn} 
	
	\begin{theorem}
		The category of 2-dimensional topological quantum field theories is the same as the category of commutative Frobenius algebras.
	\end{theorem}
	
	In general, besides just considering $n$-bordisms between $n-1$ manifolds, one might also be inclined to consider the  \textbf{extended} topological quantum field theory in $n$-dimensions.
	These are difficult to define, and would in principle rely on the language of $n$-categories to give a satisfactory definition. 
	
	We can at least summarize 
	
	\textbf{STILL NOT FINISHED}
	
	% section topological_quantum_field_theory (end)

	\section{Supersymmetry} % (fold)
	\label{sec:supersymmetry}
	%
	\subsection{Spin Representations} % (fold)
	\label{sub:spin_representations}

		\textbf{NOT FINISHED}

		Given a special orthogonal group in Euclidean or Minkowski space, $\SO(p, q$, the Spin group is defined to be the universal cover of $\SO(p, q)$.
		
		\textbf{TALK ABOUT $S^+$ and $S^-$}
		

	% subsection spin_representations (end)

	\begin{defn}
	A \textbf{Lie superalgebra} is a $\mathbb Z_2$-graded Lie algebra with a commutator bracket satisfying:
		$$[x, y]= -(-1)^{|x||y|} [y, x]$$
	Where $|\cdot|$ is the $\zz_2$ grading.
	\end{defn}
	In our case, we will be extending the familiar \emph{Poincare algebra} of $\mathrm{Lie}\{ \mathrm{SO}(3, 1) \ltimes \mathbb R^4  \}$ by $\mathcal N$ ``odd'' vectors, which transform in the fundamental representation of $\mathrm{SL(2, \mathbb C)}$, which is a projective \emph{spinor} representation of the Lorentz group. The space of odd vectors is denoted by $\Pi S$.
	\begin{defn}[Super-Poincare Algebra]
		A \textbf{super-Poincare algebra}, $\frak{spoin}$, is a Lie superalgebra arising as an extension
		\[
			\begin{tikzcd}
				\Pi S^{\oplus \mathcal N} \arrow[r] & \frak{spoin} \arrow[r] & \frak{poin}
			\end{tikzcd}
		\]
		of the Poincare algebra $\frak{Poin}$ by the vector space of odd vectors, taken to be in odd degree.
	\end{defn}
	% Here, we have not described the nature of this extension. It suffices to say that the anti-commutator $\{Q_\alpha, \bar Q_\beta \}$
	
	The brackets between the odd vectors $\{Q^A_\alpha, Q^B_\beta \} $ give rise to  central elements $Z^{AB}$ in the algebra. These are called \emph{supercharges} and arise as:
	$$\{Q^A_\alpha, Q^B_\beta \} = \epsilon_{\alpha \beta} Z^{AB}.$$
	They satisfy
	$$Z^{AB} = -Z^{BA}.$$
	So that there are a total of $\mathcal N (\mathcal N - 1)/2$ distinct supercharges in a theory with $\mathcal N$ supersymmetry generators. 
	
	\begin{defn}[$R$-symmetry group]
		The \textbf{$R$-symmetry} group is the group of outer automorphisms of the super-Poincare group which fixes the underlying Poincare group. 
	\end{defn}
	For the case of $\mathcal N = 4$ the $R$-symmetry group turns out to be $\SU(4) \cong \mathrm{Spin}(6)$. For a deeper review of the subject, see \cite{quevedo2010}.
	
	\begin{phys}[Sector]
		Given a supersymmetry operator $Q$ such that $Q^2 = \frac{1}{2} [Q, Q] = 0$, we define the sector of our theory $\mathcal E$ associated to $Q$ to the set of $Q$ invariants, and denote this as $(\mathcal E, [Q, -])$.
		
		Slightly more precisely, $[Q, -]$ defines a differential operator, and the ``observables'' become exactly those gauge-invariant quantities annihilated by $Q$ modulo those that are $Q$-exact.
	\end{phys}

	% section supersymmetry (end)
	
\chapter{Instantons and the ADHM Construction\label{ch:instantons}}

Instantons are objects of significant interest to both physicists and mathematicians. 
For physicists, they represent \emph{classical solutions to the equations of motion}. In the context of field theory, and more specifically \emph{Yang-Mills Field Theory}, instantons correspond to nontrivial field configurations on a given spacetime manifold. 

Donaldson used the interesting mathematical properties of Yang-Mills instantons on $\mathbb R^4$ to prove novel and extremely surprising statements about the nontrivial smooth structures that can be associated to $\mathbb R^4$ uniquely among all Euclidean spaces\cite{donaldson1984}. 

A useful picture comes from quantum mechanics, of a particle in a double-well potential. Having a particle localized at the bottom of either well gives rise to a classical solution. Perturbative corrections around this minimum due to the quantum theory may give rise to harmonic-oscillator-type structure within the well, but is completely unable to account for the possibility of \emph{quantum tunneling} across the barrier into the second well of the potential. To account for this, we must understand the space of classical solutions in addition to performing perturbation theory. 

Mathematically, one way that this can manifest itself is in the fact that $e^{-1/x}$ has every higher derivative vanish as $x \to 0^+$. It is the same phenomenon that allows for the existence of \emph{bump functions} in real analysis and also for \emph{asymptotic expansions} in various areas of physics and engineering.

For the purposes of this thesis, instantons will not themselves play a central role, but their close relatives in three dimensions will: magnetic monopoles. In order to understand the construction of monopoles, however, it will be important to first understand the famous self-duality equation and ADHM construction of instantons. 


\section{Instantons in Classical Yang-Mills Field Theory} % (fold)
\label{sec:instantons_in_classical_yang_mills_field_theory}

\subsection{The Equations of Motion} % (fold)
\label{sub:the_equations_of_motion}

Yang-Mills gauge theory is a theory with gauge group $G = \mathrm{SU}(n)$. In four dimensions, the objects of study are bundles associated to a principal $G$-bundle on Euclidean 4-space $M = \rr^4$. $\rr^4$ has a Riemannian metric, so we have a Hodge-star operator giving an isomorphism:
\[
	\star: \Omega^k \to \Omega^{4-k}.
\]
From the prior section, gauge theory on $\rr^4$ involves a connection $1$-form $A$ transforming in the $\ad\gg$ representation. From this, we obtain the field-strength $F$, again transforming in the adjoint action, by applying the covariant exterior derivative:
\begin{equation}
	F = \dd_A A = \dd A + [A, A]
\end{equation}
Both $F$ and $\star F$ are $\frak g$-valued 2-forms. On the other hand $F \wedge \star F$ is a $\gg$-valued 4-form. Taking the trace of this over the Lie algebra gives a 4-form that can be integrated over $M$, $\tr F \wedge \star F$. This is equivalently denoted by $||F||^2$ since $\tr (F \wedge \star  F)$ corresponds exactly to the inner product norm on $\frak g$-valued 2-forms induced by the killing form.

\begin{prop}
	$\tr(F \wedge \star F)$ is gauge independent and globally defined.
\end{prop}
\begin{proof}
	Since $F$ transforms in the adjoint representation, the cyclic property of the trace gives:
	\[
		\tr(F \wedge \star F) \to \tr(g F g^{-1} \wedge g \star F g^{-1}) = \tr(F \wedge \star F).
	\]
\end{proof}
It is important to recall that the field strength corresponds to a curvature 2-form on some principal $\SU(n)$-bundle, $P$. Given such a field strength 2-form on $M$, it can be pulled back to any bundle $E$ associated to $P$. 

In Yang-Mills theory, the action is given by:
\begin{equation}
	S[A] := \frac{1}{8\pi} \int_M \tr(F \wedge \star F)
\end{equation}
We aim to find $A$ so that $S_E[A]$ is a minimum. To do this, we use standard calculus of variations. Consider a small perturbation $A + t \alpha$.
		\begin{equation*}
			\begin{aligned}
				\mathcal F[A + t \alpha] &= \dd (A + t \alpha) + A \wedge A + t [A, \alpha] + O(t^2) \\
				&= \mathcal F[A] + t (d \alpha + A \wedge \alpha)\\
				&= \mathcal F[A] + \dd_{A} \alpha
			\end{aligned}
		\end{equation*}
		so that to order $t$:
		\begin{equation*}
			\begin{aligned}
				||\mathcal F[A + t \alpha]||^2 &= ||\mathcal F[A + t \alpha]||^2 + 2 t (\mathcal F[A], \dd_{A} \alpha)\\
				&\Rightarrow (\mathcal F[A], \dd_{A} \alpha) = 0 ~ \forall \alpha.
			\end{aligned}
		\end{equation*}
		The adjoint of the covariant derivative is the codifferential $\star \dd_A \star$, so that we can equivalently write this as:
		\[
			\forall \alpha \; (\star \dd_A \star \mathcal F[A], \alpha) = 0 \Rightarrow \dd_A \star F = 0.
		\]
		Except for the case of an abelian gauge theory, these will in general give second-order nonlinear differential equations in the connection that are difficult to solve for explicitly. Though we will not be able to easily talk about general field configurations, we \emph{will} be able to talk about field configurations that are minima for the action on the principal $\SU(n)$ bundle $P$ that the theory is defined on. To do this, we must first understand a connection between a certain integral of the field strength and the topology of $P$.

% section instantons_in_classical_yang_mills_field_theory (end)

\subsection{The Instanton Number} % (fold)
\label{sub:the_instanton_number}

		The action is defined by $\int_M \tr(F \wedge \star F)$. Considering $F \wedge F$ gives us another important quantity. 
		\begin{defn}[Instanton Number]
			The \textbf{instanton number} $k$ for a given field configuration is given by
			\begin{equation}
				k := \int_M \tr(F \wedge F).
			\end{equation}
		\end{defn}

		Recall from the definition of Chern classes in \ref{defn:chern} that the Chern numbers are independent of the choice of connection. Recall further that the first few Chern numbers were given by:
		\[
			c_1(E) := \frac{i}{2\pi} \int_{M} \tr(F) \qquad c_2(E) := \frac{1}{8\pi^2} \int_M \left[\tr(F \wedge F) - \tr(F) \wedge \tr(F)\right]
		\]
		Note that since $\su(n)$ consists of only traceless matrices, $c_1$ vanishes, and thus for any associated bundle $\su(n)$-bundle $E$ we have:
		\[
			c_1(E) = 0 \qquad c_2(E) = \frac{1}{8\pi^2} \int_M \tr(F \wedge F) = k.
		\]
		Thus in our case, the instanton number is simply the second Chern class, and in particular is a \emph{topological invariant of the bundle $E$, independent of the connection}.
		
% subsection the_instanton_number (end)

\subsection{The ASD Equations} % (fold)
\label{sub:the_asd_equations}

	We are now in a place where we can understand the equations defining the local minima of the action. Note first by basic properties of $\star$ that
	\begin{equation}
		\star \star: \Omega^2(M, \gg) \to \Omega^2(M, \gg)
	\end{equation}
	is equal to $1$ for $M = \rr^4$. This means that this operator has two eigenspaces corresponding to $+1$ and $-1$, giving a decomposition
	\begin{equation}
		\Omega^2(M, \gg) = \Omega^2(M, \gg)^+ \oplus \Omega^2(M, \gg)^-.
	\end{equation}
	So in general $F$ can be expressed as a sum $F = F_+ + F_-$ of 2-forms in these two spaces. Moreover since these two spaces are orthogonal by the Hermiticity of $\star$, $(F_+, F_-) = 0$. On one hand, then:
	\[
	\begin{aligned}
		S[A] &= \int_M \tr(F \wedge \star F)\\
			 &= \int_M \tr((F_+ + F_-) \wedge \star(F_+ + F_-))\\
			 &= \int_M \tr(F_+ \wedge \star F_+) + \int_M \tr(F_+ \wedge \star F_+)\\
	\end{aligned}
	\]
	Note that the action integral is the integral of $||F||^2$ is is necessarily positive. Now consider the following manipulation:
	\[
	\begin{aligned}
		8\pi^2 k &= \int_M \tr(F \wedge F) \\
		  &= \int_M \tr((F_+ + F_-)\wedge (F_+ + F_-)) \\
		  &= \int_M \tr(F_+ \wedge F_+) + \int_M \tr(F_- \wedge F_-) \\
		  &= \int_M \tr(F_+ \wedge F_+) + \int_M \tr(F_- \wedge F_-) \\
		  &= \int_M \tr(F_+ \wedge \star F_+) - \int_M \tr(F_- \wedge \star F_-)\\
		  &= \int_M ||F_+||^2 - \int_M ||F_-||^2.
	\end{aligned}
	\]
	Using the triangle inequality we get:
	\begin{equation}
		S[A] \geq |8 \pi^2 k|.
	\end{equation}
	It is also easy to see that equality will be satisfied iff $F = F_+$ or $F=F_-$.
	
	Note that any solution of the self-dual equation $F=F_+$ can be obtained from a solution of the anti-self-dual equation $F=F_-$ by performing a spatial flip $x_1 \to - x_1$.
	
	We thus have the \textbf{anti-self-dual equations} for instantons:
	\begin{equation}\label{eq:asd1}
		\star F = - F,
	\end{equation}
	or component-wise:
	\begin{equation}
		\begin{aligned}
			F_{12} + F_{34} &= 0\\
			F_{14} + F_{23} &= 0\\
			F_{14} + F_{32} &= 0.
		\end{aligned}
	\end{equation}
	% Since $F_{12} := \dd_A A = [\dd_A, \dd_A]$
	
	We see that the instanton number depends on the principal bundle, and that the instanton number of the trivial bundle is zero. 
	
	\begin{nb}
		$\su(n)$-instantons do not exist in Minkowski space $\rr^{3,1}$, since $\star^2 = -1$ would have eigenvalues $\pm i$ and $F = \pm i F$ would contradict that $F$ is a real object as an $\su(n)$-valued 2-form.
	\end{nb}

% subsection the_asd_equations (end)


	\subsection{Classifying Principal Bundles over $S^4$} % (fold)
	\label{sub:classifying_principal_bundles_over_s_4}

		In our above analysis, and the construction of instantons that is to follow, we make several assumptions about $F$ and $A$.
		\begin{itemize}
			\item For the above integrals to have made sense, we must require that $F(\vec x)$ decays ``sufficiently quickly'' as $|\vec x| \to \infty$.
			\item Consequently we must also have $A$ ``tend to a constant''. In the language of gauge theory, $A$ must become ``pure gauge'' $g \dd g^{-1}$ as $|\vec x| \to \infty$.
			\item We thus restrict the gauge group to consist of only \textbf{framed} gauge transformations, defined next.
		\end{itemize}
		\begin{defn}
			A framed gauge transformation on $\rr^4$ is one that tends to a constant group element as $|\vec x| \to \infty$.
		\end{defn}

		We first change the setting from $\rr^4$ to $S^4$. Because of the decay of the fields, extending the bundle to $S^4$ with framed gauge transformation will give a well-defined field strength and vector potential on $S^4$. The following argument is directly from \cite{lindenhovius2011}.
	
		We will understand how to compute the instanton number on $S^4$ by using a \emph{clutching function} defined on $S^3$ connecting the two hemispheres of $S^4$. First, note that on an open disk, the form $\tr (F \wedge F)$ (by virtue of being locally exact) can be written as 
		\[
			\dd \tr \left[F \wedge A - \frac13 A^3 \right] = \tr( F \wedge F)
		\]
		where $A^3 = A \wedge A \wedge A$.
		Now take $D_N$ and $D_S$ two disks overlapping on $S^3$. The $G$-bundle must have an overlap function $\rho: S^3 \to G$.
	
		Now the integral becomes:
		\[
		\begin{aligned}
			8\pi k &= \int_{S^4} \tr (F \wedge F)\\
				   &= \int_{D_S} \tr (F_S \wedge F_S) + \int_{D_N} \tr (F_N \wedge F)\\
				   &= \int_{\partial D_S} \tr \left[F_S \wedge A_S - \frac13 A_S^3 \right] 
				   + \int_{\partial D_N} \tr \left[F_N \wedge A_N - \frac13 A_N^3 \right]\\
				   &= \int_{S^3} \left( \tr \left[F_S \wedge A_S - \frac13 A_S^3 \right] + \tr \left[F_N \wedge A_N - \frac13 A_N^3 \right] \right).
		\end{aligned}
		\]
		After some manipulations, changing $A_N, F_N$ to $A_S, F_S$ by transforming according to $\rho$, this all reduces to:
		\[
			k = - \frac{1}{24\pi^2} \int_{S^3} \tr((\rho \dd \rho)^3)
		\]
		and this can now be expressed as the pullback of $\rho$ acting on the Mauer-Cartan form on some $\SU(2)$-homotopic subgroup of $G$ by Bott's theorem. Hence,
		\[
			k = - \frac{1}{24\pi^2} \int_{S^3} \rho^* \tr(\Theta^3) = \frac{\deg \rho}{24} \int_{\SU(2)} \tr(\Theta^3).
		\]
		On $\SU(2)$, the triple wedge of the Mauer-Cartan form gives a volume form whose integral is exactly $24\pi^2$.

		\begin{prop}
			The homotopy classes of maps $S^3 \to \mathrm{SU}(2)$ are classified by integers.
		\end{prop}
		\begin{proof}
			This follows from noting that $\SU(2) \cong S^3$ and $\pi_3(S^3) = \zz$.
		\end{proof}
		\noindent Consequently, we have our result.
		\begin{prop}
			 The instanton number $k$ must be an integer equal to the negative of the degree of the clutching function $\rho$ defining the principal $G$-bundle on $S^4$.
		\end{prop}
		
		\noindent With the stage set, will now discuss the method for constructing \emph{all} instantons on $\rr^4$. This is the \textbf{ADHM construction} of Atiyah, Hitchin, Drinfeld, and Manin \cite{atiyah65}.

	% subsection classifying_principal_bundles_over_s_4 (end)


\section{Construction of Instantons} % (fold)
\label{sec:construction_of_instantons}
	
	In the ADHM construction, we make use of an identification $\rr^4 \cong \cc^2$. 
	
	We will show how this construction will give a vector bundle $E$ of rank $n$ over $S^4$ with topological charge $-k$. The proof that this exhaustively gives \emph{all} instantons can be found in \cite{donaldson1988}.
	
	\subsection{Holomorphic and Hermitian Vector Bundles} % (fold)
	\label{sub:holomorphic_and_hermitian_vector_bundles}
	
	\textbf{FINISH THIS}
	
	% subsection holomorphic_and_hermitian_vector_bundles (end)
	
	\subsection{The Data}
	
	Let $x_1, x_2, x_3, x_4$ parameterize a $\mathbb R^4$, and write this as $\mathbb C^2$ using $z_1 = x_2 + i x_1, z_2 = x_4 + i x_3$. In terms of the complex coordinates, we get
	\begin{equation}
		\begin{aligned}
			D_1 &:= \frac{1}{2} ({\dd_A}_2 - i {\dd_A}_1)\\
			D_2 &:= \frac{1}{2} ({\dd_A}_4 - i {\dd_A}_3)
		\end{aligned}
	\end{equation}
	We can express anti-self duality of $\mathcal F_{\mu \nu}$ in terms of these $D_1, D_2$ through two equations:
	\begin{equation}
		\begin{aligned}
			\left[ D_1, D_2 \right] &= 0\\
			[D_1, D_1^\dagger] + [D_2, D_2^\dagger] &= 0
		\end{aligned}
	\end{equation}
	
	We will now describe how to obtain a holomorphic vector bundle of rank $n$ on $S^4$ together with a connection $1$-form on this bundle that will give a solution to the ASD equations of motion. 
	% The idea behind ADHM is to take ``Fourier transforms'' of these $D_i$ to matrices $B_i$.
	
	\begin{defn}[ADHM System]
		Let $U$ be a $4$-dimensional space with complex structure. An \textbf{ADHM System} on $\cc^2$ is a set of linear data:
		\begin{enumerate}
			\item Vector spaces $V,W$ over $\mathbb C$ of dimensions $k,n$ respectively.
			\item Complex $k \times k$ matrices $B_1, B_2$, a $k\times n$ matrix $I$, and an $n\times k$ matrix $J$.
		\end{enumerate}
		
		We can see this diagrammatically by the following quiver:
		\[
			\begin{tikzcd}
				W \arrow[r,bend left,"I"] & V \arrow[l,bend left,"J"] \arrow[out=30,in=90,loop,swap,"B_1"] \arrow[out=330,in=270,loop,"B_2"]
			\end{tikzcd}
		\]
		
		A set of ADHM Data is an ADHM system if it satisfies the following constraints:
		\begin{enumerate}
			\item The ADHM equations:
			\begin{equation}
				\begin{aligned}
					[B_1, B_2] + IJ&=0\\
					[B_1, B_1^\dagger] + [B_2, B_2^\dagger] + II^\dagger - J^\dagger J &= 0
				\end{aligned}
			\end{equation}
			\item For $(x,y) \in  \cc^2$ with $x = (z_1, z_2), y = (w_1, w_2)$, the map:
			\begin{equation}
				\alpha_{x,y} = 
				\begin{pmatrix}
					w_2 J - w_1 I^\dagger \\
					-w_2 B_1 - w_1 B_2^\dagger - z_1 \\
					w_2 B_2 - w_1 B_1^\dagger + z_2
				\end{pmatrix}
			\end{equation}
			is injective from $V$ to $W \oplus (V\otimes \cc^2)$ while
			\[
			\beta_{x,y} = \begin{pmatrix}
				w_2 I + w_1 J^\dagger & w_2 B_2 - w_1 B_1^\dagger + z_2 & w_2 B_1 + w_1 B_2^\dagger + z_1
			\end{pmatrix}	
			\]
			is surjective from $W \oplus (V \otimes \cc^2)$ to $V$.
		\end{enumerate}
	\end{defn}
	
	It is an easy check to see
	\begin{obs}
		If $B_1, B_2, I, J$ satisfy the above conditions, then for $g \in \mathrm(k), h \in \mathrm{SU}(n)$, 
		\[
			(g B_1 g^{-1}, g B_2 g^{-1}, gI, Jg^{-1})
		\]
		also satisfies the ADHM equations.
	\end{obs}
	
	We can recast the ADHM equations into a more succinct form.
	\begin{prop}
		The ADHM equations are satisfied iff
		\[
		\begin{tikzcd}
			0 \arrow[r]&V \arrow[r,"\alpha_{x,y}"] & W \oplus (V \otimes \cc^2) \arrow[r,"\beta_{x,y}"] & V \arrow[r] & 0
		\end{tikzcd}
		\]
		is a complex, namely $\beta \circ \alpha = 0$.
	\end{prop}
	\begin{proof}
		We need both $\beta \alpha = 0$ as well as surjectivity of $\beta$ and injectivity of $\alpha$. The equation $\beta \alpha = 0$ reduces to a quadratic polynomial in the $w_1, w_2$ with the two ASD equations emerging as coefficients.
	\end{proof}
	
	% \begin{prop}
	% 	The topological charge of $E \to S^4$ is $-k$.
	% \end{prop}
	
	\begin{theorem}[ADHM construction]
		There is a one-to-one correspondence between equivalence classes of solutions to the ADHM system and gauge equivalence classes of anti-self-dual $\mathrm{SU}(n)$-connections $\mathcal A$ with instanton number $k$.
	\end{theorem}
	
	A full proof of this theorem is beyond the scope of this thesis. Nonetheless, we show how such a set of data gives rise to a 2-dimensional $\SU(n)$-associated bundle $E$ over $S^4$. 
	
	Succinctly: the only nontrivial cohomology group of this complex is $\ker \beta_{x,y}/ \mathrm{im}\, \alpha_{x,y}$. This gives a vector bundle over $\cc^2 \times \cc^2$ which can be identified with $\mathbb H^2$. An equivariance condition on the data under quaternionic action will let this descend to a vector bundle on $\mathbb{HP}^1 \cong S^4$. This 2D complex vector bundle will be associated to some appropriate principal bundle and have instanton number $k$.
	
	In quaternionic language, the ADHM equations become easier to work with. To each $x = (q_1, q_2) \in \cc^2$, we can associate a quaternionic operator acting on $\cc^2$ as:
	\begin{equation}
		(q_1, q_2) \mapsto z = \begin{pmatrix}
			\overline q_2 & -q_1\\
			\overline q_2 & q_2
		\end{pmatrix}.
	\end{equation}
	For $(q_1, q_2) \neq 0$ this is a rank two linear operator.
	
	We can we write the ADHM equations by defining an operator:
	\begin{equation}
		\Delta_{x,y} := \begin{pmatrix}
			\beta^\dagger_{x,y} & \alpha_{x,y}
		\end{pmatrix}.
	\end{equation}
	Then it is easy to see that (with $x=(z_1, z_2)$ and $y=(w_1, w_2)$)
	\begin{equation}
		\Delta_{x, y} = a w + b z
	\end{equation}
	where $w, z$ are the quaternionic matrices corresponding to the complex pairs $(w_1, w_2), (z_1, z_2)$ and 
	\begin{equation}
		a = \begin{pmatrix}
			I^\dagger & J\\
			B_2^\dagger & -B_1\\
			B_1^\dagger & B_2
		\end{pmatrix}, \qquad b = \begin{pmatrix}
			0 & 0\\
			I_k & 0\\
			0 & I_k
		\end{pmatrix}
	\end{equation}
	are the by $n+ 2n$ by $2k$ matrices, with $I_k$ here denoting the identity. We similarly have \footnote{The notation here is suggestive. $\Delta^\dagger$ is a Dirac operator, and solutions to the ADHM equations are $\Psi(x,y)$ so that $\Delta^\dagger \Psi = 0$.}
	\begin{equation}
		\Delta^\dagger_{x,y} = \begin{pmatrix}
			\beta_{x,y}\\
			\alpha^\dagger_{x,y}
		\end{pmatrix} =
		(a w  + b z)^\dagger.
	\end{equation}
	Importantly, the kernel of this operator is $\ker \beta_{x,y} \cap \ker \alpha^\dagger_{x,y}$ which can be rewritten as $\ker \beta_{x,y} \cap \mathrm{im}(\alpha)^\perp_{x,y}$. By the definition of orthogonal complement together with $\beta_{x,y} \circ \alpha_{x,y} = 0 \Rightarrow \im \alpha_{x,y} \subseteq \ker \beta_{x,y}$, this intersection is seen to be isomorphic to $\ker \beta_{x,y} / \mathrm{im}\, \alpha_{x,y}$.
	
	We see that $x, y$ can be interpretted as two quaternions on $\mathbb H^2$. We have an action of the quaternionic operators on this space by $(x, y) \to (x q, y q)$. The space $\ker \Delta^\dagger_{x, y} \to (x, y)$ gives rise to a rank $n$ vector bundle $\tilde E$ on $\mathbb H^2$. Observe of the following equivariance condition:
	\begin{equation}
		\Delta^\dagger_{xq, yq} = (a w q + b z q)^\dagger = q^\dagger \Delta^\dagger.
	\end{equation}
	For $q \neq 0$, $q^\dagger$ maintains full rank, so the kernel of $\Delta^\dagger_{xq, yq}$ is the same as the kernel of $\Delta^\dagger_{x,y}$. This means that $\tilde E$ descends to a vector bundle on $\mathbb{HP}^1 \cong S^4$. This is our desired construction.
	
	Moreover, if we take an orthonormal basis of $\ker \Delta^{\dagger} \subseteq W \oplus (V \otimes \cc^2)$, we can construct a $(n+2k) \times n$ matrix $U$ that satisfies the orthonormality condition $U^\dagger U = 1$. Then it can in fact be shown that our connection 1-form is defined in terms of $U$ as:
	\[
		A := U^\dagger \, \dd U.
	\]

% section construction_of_instantons (end)





% \include{ch-pastwork/chapter-pastwork}
% \include{ch-usage/chapter-usage}
% \include{ch-conclusion/chapter-conclusion}
% \appendix % all chapters following will be labeled as appendices
% \include{ch-appendicies/implementation}
% \include{ch-appendicies/printing}


% Make the bibliography single spaced
\singlespacing
\bibliographystyle{plain}

% add the Bibliography to the Table of Contents
\cleardoublepage
\ifdefined\phantomsection
  \phantomsection  % makes hyperref recognize this section properly for pdf link
\else
\fi
\addcontentsline{toc}{chapter}{Bibliography}

% include your .bib file
\bibliography{thesis}

\end{document}

