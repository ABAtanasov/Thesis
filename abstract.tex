The aim of this thesis is to give the reader a gentle but thorough introduction to the vast web of ideas underlying the realization of the geometric Langlands correspondence in the physics of quantum field theory (QFT). It begins with a pedagogically-motivated introduction to the relevant concepts in the Langlands program, quantum field theory, and gauge theory for an audience of mathematicians or physicists. With this machinery in place, the more complicated phenomena associated with gauge theory is explored, specifically instantons, topological operators, and electric-magnetic duality. We conclude by connecting the ideas of the Langlands correspondence discussed at the start with phenomena in topologically twisted $\mathcal N = 4$ supersymmetric Yang-Mills theory (SYM) which exhibits a striking property known as $S$-duality. A large part of the goal of this thesis is to give an exposition to the language and techniques that the literature related to this topic already assumes familiarity with, so that an advanced undergraduate or early graduate student might have a good exposition into this field. 